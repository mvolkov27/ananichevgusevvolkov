\documentclass[11pt]{llncs}
\usepackage{amsmath}
\usepackage{amssymb}
\usepackage{epic,gastex}
\usepackage{array}
\usepackage{enumerate}

\newcommand{\sa}{synchronizing automata}
\newcommand{\san}{synchronizing automaton}
\newcommand{\sw}{reset word}
\newcommand{\sws}{reset words}
\newcommand{\ssw}{reset word of minimum length}
\newcommand{\rl}{reset threshold}

\newcommand{\A}{\mathrsfs{A}}
\newcommand{\C}{\mathrsfs{C}}
\newcommand{\R}{\mathrsfs{R}}
\newcommand{\AD}{\mathrsfs{A_{\sigma}}}
\newcommand{\G}{\Gamma}
\newcommand{\D}{\Delta}
\newcommand{\g}{\gamma}
\newcommand{\dl}{\delta}
\newcommand{\rk}{\mathrm{rk}}
\newcommand{\rt}{\mathrm{rt}}
\newcommand{\rtc}{\mathrm{rt}_{c}}

\DeclareMathOperator{\weight}{wg}

\DeclareSymbolFont{rsfscript}{OMS}{rsfs}{m}{n}
\DeclareSymbolFontAlphabet{\mathrsfs}{rsfscript}

\begin{document}

\title{Synchronizing automata of bounded rank\thanks{Supported by the Russian Foundation for Basic Research,
grant 10-01-00793, and by the Presidential Program for young researchers, grant MK-266.2012.1. Author is also grateful to Erasmus Mundus
Action 2 Partnerships --- Triple I.}}

\titlerunning{Synchronizing automata of bounded rank}

\author{Vladimir V. Gusev}

\authorrunning{V. V. Gusev}

\tocauthor{V. V. Gusev}

\institute{Institute of Mathematics and Computer Science,\\
Ural Federal University, 620000 Ekaterinburg, Russia\\
\email{vl.gusev@gmail.com}}

\maketitle

\begin{abstract}
We reduce the problem of synchronization of an $n$-state automaton with letters of rank at most $r < n$ to the problem of synchronization
of an $r$-state automaton with constraints given by a regular language. Using this technique. we construct a series of synchronizing
$n$-state automata in which every letter has rank $r < n$ and whose reset threshold is at least $r^2 -r - 1$ Moreover, if $r >
\frac{n}{2}$, such automata are strongly connected.
\end{abstract}

\section{Introduction}
A complete deterministic finite automaton $\mathrsfs{A}$ is called \emph{synchronizing} if the action of some word $w$ resets
$\mathrsfs{A}$, that is, leaves the automaton in one particular state no matter at which state $w$ is applied. Any such word $w$ is said to
be a \emph{reset word} for the automaton. The minimum length of reset words for $\mathrsfs{A}$ is called the \emph{\rl} of $\mathrsfs{A}$
and is denoted by $\rt(\mathrsfs{A})$. Synchronizing automata often serve as natural models of error-resistant systems. For a brief
introduction to the theory of synchronizing automata we refer the reader to the recent surveys~\cite{Sa05,Vo08}. The interest to the field
is heated by the famous \v{C}ern\'{y} conjecture.

In~1964 Jan \v{C}ern\'{y}~\cite{Ce64} constructed for each $n>1$ a synchronizing automaton $\mathrsfs{C}_n$ with $n$ states whose \rl\ is
equal to $(n-1)^2$. Soon after that he conjectured that these automata represent the worst possible case. In other words, every
synchronizing automaton with $n$ states can be synchronized by a word of length $(n-1)^2$. Despite intensive research, the best upper bound
on the \rl\ of synchronizing automata with $n$ states achieved so far is cubic. Classical bound is $\frac{n^3-n}6$, see~\cite{Pi83}. A
slightly better bound $\frac{n(7n^2+6n-16)}{48}$ has been claimed in \cite{Tr11}. Though the \v{C}ern\'{y} conjecture is open in general,
it has been confirmed for some restricted classes of synchronizing automata, see~\cite{AS09,Du98,Ka03,Tr07,Vo09}. For some classes
quadratic upper bound is established, see~\cite{BBP,Rys}.
%such bound for automata that has one letter whoose underlying digraph is connected.

In the present paper we approach the following question: how does the reset threshold of an automaton $\A$ depend on the ranks of letters
in $\A$. More formally, consider a synchronizing $n$-state automaton $\A$ with $k$ letters and let $r_1, r_2, \ldots, r_k$ be their ranks.
What is the upper bound on the reset threshold of $\A$ in terms of $n$ and $r_1, r_2, \ldots, r_k$? In full generality, this question
appears to be very hard. In this paper we focus on examples with ``large'' reset threshold and bounded rank. There have already been some
related results in the literature. Synchronizing automata with a letter of deficiency 2 were considered in~\cite{AVZ} while in~\cite{AGV} a
series of slowly synchronizing automata in which all letters deficiency 2 was reported.

We say that an automaton $\A$ is \emph{of bounded rank} $r$ if every letter of $\A$ has rank at most $r$. We present a uniform approach to
deal with such automata. We illustrate this approach by constructing a series of automata of bounded rank $r$ with reset threshold at least
$r^2 -r - 1$.

\section{Preliminaries}

Let $\A = (Q, \Sigma)$ be a complete deterministic finite automaton. Let us fix some orderings of $Q = \{q_1, q_2, \ldots q_n\}$ and
$\Sigma = \{a_1, a_2, \ldots, a_k\}$. We associate with an arbitrary letter $a_\ell$ a square $(0,1)$-matrix $M(a_\ell)$ of order $n$ by
the following rule: $M(a_\ell)[i,j] = 1$ if $q_i\cdot a_{\ell} = q_j$, otherwise $M(a_\ell)[i,j] = 0$. We call the set $\langle M(a_1),
\ldots, M(a_{k})\rangle$ the \emph{matrix representation} of the automaton $\A$ and write $\A = \langle M(a_1), \ldots, M(a_{k})\rangle$.
We can uniquely extend the domain of mapping the $M(\cdot )$ from $\Sigma$ to $\Sigma^{*}$ in accordance with the equation $M(uv) =
M(u)M(v)$, where $u,v \in \Sigma^*$. It is not hard to see the following important property of the matrix representation: for every word
$w$ we have $M(w)[i,j] = 1$ if $q_i \cdot w = q_j$, otherwise $M(w)[i,j] = 0$. This immediately implies that a word $w$ is a reset word of
$\A$ if and only if $M(w)$ has a column of 1's. Throughout the paper we also make use of the following notation. Let $\mathrsfs{M}$ be a
set of square matrices. Then $\mathrsfs{M}^{k} = M_1 \cdot M_2 \cdot \ldots \cdot M_k$, where $M_i \in \mathrsfs{M}$ for $i \in
\overline{1,k}$, and $\mathrsfs{M}^* = \bigcup_{k = 1}^{\infty} \mathrsfs{M}^k$.

Let us introduce a pair of square matrices $A_n$ and $B_n$ of order $n$. We denote the $i$-th row of matrix $A_n$ by $A_n[i,.]$ and let
$e_i$ denote the row vector whose only non-zero entry is equal to 1 and is located in position $i$.
$$ A_n[i,.] =
\begin{cases}
e_2, &\text{if } i = 1\\
e_i, &\text{if } i \neq 1\\
\end{cases},\quad
B_n[i,.] =
\begin{cases}
e_{i+1}, &\text{if } i \neq n\\
e_1, &\text{if } i = n\\
\end{cases}.
$$
We refer to the automaton $\langle A_n, B_n \rangle$ as the \v{C}ern\'{y} automaton and denote it by $\C_n$. The matrix $A_n$ corresponds
to the action of the letter $a$, the matrix $B_n$ corresponds to the action of the letter $b$. Later on we will need the following
statement concerning reset words of $\C_n$ (Proposition 3 in~\cite{Gu}).

\begin{proposition}
\label{th:cerny} Every reset word of the automaton $\mathrsfs{C}_n$ contains at least $n^2 - 3n + 2$ occurrences of the letter $b$ and at
least $n - 1$ occurrences of the letter $a$.
\end{proposition}

\section{Main results}
For the sake of simplicity throughout this section we consider automata only over two letters. All definitions and propositions can be
easily generalized. Consider an $n$-state automaton $\A$ given by a pair of matrices $A$ and $B$. We say that the set of matrix pairs
$\sigma = \langle(X,Y), (\G,\D)\rangle$ is a \emph{decomposition} of the automaton $\A$ if:
\begin{enumerate}[(i)]
\item $X, \G$ are $(0,1)$-matrices of size $n$ by $r$
\item $Y, \D$ are $(0,1)$-matrices of size $r$ by $n$
\item $A = XY$, $B = \G\D$
\item every row of $X,Y,\G,\D$ has only one occurrence of 1.
\end{enumerate}
We say that decomposition is \emph{unavoidable} if every column of $X$ and $\G$ also contains at least one occurrence of 1.


It easily follows from the definition that $r \geq max\{\rk(A),$ $\rk(B)\}$. For any such $r$ we can easily construct some decomposition.
For example, matrix $Y$ may consist of different rows of $A$ in lexicographic order. If $r$ is bigger than the number of different rows in
$A$, then the lexicographically largest row appears several times. Let $j$ be the position of the $i$-th row of $A$ in lexicographic order.
Then the $i$-th row of $X$ is equal to $e_j$. It is not hard to see that defining in the same way matrices $\G$ and $\D$ gives us a
decomposition of $\A$. Moreover, if $r = \rk(A) = \rk(B)$, then this decomposition is unavoidable. We notice that even in case $r = \rk(A)
= \rk(B)$ there could exist several different decompositions.

Suppose now that we are given a decomposition $\sigma = \langle(X,Y), (\G,\D)\rangle$ of an automaton $\A$, where $X$ has size $n$ by $r$.
We are to define the \emph{reduced automaton} $\AD$ that is going to be a key object of this paper. The automaton $\AD$ has $\{1,\ldots,
r\}$ as the state set. The action of the input alphabet $\Sigma' = \{ y\g, yx, \dl x, \dl\g \}$ (these are letters, but expressed as two
symbols for convenience) is given by the corresponding $(0,1)$-matrices $\mathcal{M} =\{ Y\G , YX, \D X, \D\G \}$ of size $r$ by $r$. We
notice that every row of the matrices $X, Y, \G, \D$ contains only one occurrence of 1, so their products have the same property. Thus
automaton $\AD$ is deterministic and complete

\begin{example}
\label{ex} Let $\mathrsfs{D}$ be the automaton shown in Fig.~\ref{fig:d5} on the left (this automaton first appeared in \cite{AGV} as
$\mathrsfs{D}''_5$). We can define two decompositions of $\mathrsfs{D}$ as follows:
$$
\sigma_1:\;\;
A =
\left(
\begin{smallmatrix}
1&0&0&0\\
0&1&0&0\\
0&0&1&0\\
0&0&0&1\\
1&0&0&0
\end{smallmatrix}
\right)
\left(
\begin{smallmatrix}
0&1&0&0&0\\
0&0&1&0&0\\
0&0&0&1&0\\
0&0&0&0&1
\end{smallmatrix}
\right),\;
B =
\left(
\begin{smallmatrix}
1&0&0&0\\
1&0&0&0\\
0&1&0&0\\
0&0&1&0\\
0&0&0&1
\end{smallmatrix}
\right)
\left(
\begin{smallmatrix}
0&0&1&0&0\\
0&0&0&1&0\\
0&0&0&0&1\\
1&0&0&0&0
\end{smallmatrix}
\right);
$$


$$
\sigma_2:\;\;
A =
\left(
\begin{smallmatrix}
1&0&0&0\\
0&1&0&0\\
0&0&1&0\\
0&0&0&1\\
1&0&0&0
\end{smallmatrix}
\right)
\left(
\begin{smallmatrix}
0&1&0&0&0\\
0&0&1&0&0\\
0&0&0&1&0\\
0&0&0&0&1
\end{smallmatrix}
\right),\;
B =
\left(
\begin{smallmatrix}
0&1&0&0\\
0&1&0&0\\
0&0&1&0\\
0&0&0&1\\
1&0&0&0
\end{smallmatrix}
\right)
\left(
\begin{smallmatrix}
1&0&0&0&0\\
0&0&1&0&0\\
0&0&0&1&0\\
0&0&0&0&1
\end{smallmatrix}
\right).
$$
The reduced automaton $\mathrsfs{D}_{\sigma_1}$ is shown in the middle of Fig.~\ref{fig:d5}. The letter $y \g$ is omitted, since it acts as
identity mapping. The action of the letter $\dl \g$ coincides with the action of $yx$. The reduced automaton $\mathrsfs{D}_{\sigma_2}$ is
shown in Fig.~\ref{fig:d5}(right). Since the actions of letters $yx, y\g, \dl \g$ coincide, we keep only $yx$ on the picture.

\begin{figure}[ht]
 \begin{center}
  \unitlength=2.8pt
    \begin{picture}(18,26)(14,-24)
    \gasset{Nw=6,Nh=6,Nmr=3}
    \node(n0)(0,0){2}
    \node(n1)(-13,-9){1}
    \node(n2)(13,-9){3}
    \node(n3)(-8,-23){5}
    \node(n4)(8,-23){4}
    \drawedge(n1,n0){$a$} \drawedge(n2,n4){$a,b$}
    \drawedge(n0,n2){$a,b$} \drawedge[ELpos=35](n3,n0){$a$}
    \drawedge(n3,n1){$b$} \drawedge[ELpos=25](n1,n2){$b$}
    \drawedge(n4,n3){$a,b$}
    \end{picture}
 \begin{picture}(18,26)(-45,-2)
    \gasset{Nw=6,Nh=6,Nmr=3, loopdiam=5}
    \node(A1)(0,18){$1$}
    \node(A2)(18,18){$2$}
    \node(A3)(18,0){$3$}
    \node(A4)(0,0){$4$}
    %sigma 2 second on a picture
    \drawloop[loopangle=135](A1){$\dl x$}
    \drawedge(A1,A2){$yx$}
%    \drawedge(A2,A3){}
%    \drawedge(A3,A4){}
%    \drawedge(A4,A1){}
    \drawedge(A2,A3){$\dl x, yx$}
    \drawedge(A3,A4){$\dl x, yx$}
    \drawedge(A4,A1){$\dl x, yx$}
    \end{picture}
 \begin{picture}(18,26)(15,-2)
    \gasset{Nw=6,Nh=6,Nmr=3}
    \node(A1)(0,18){$1$}
    \node(A2)(18,18){$2$}
    \node(A3)(18,0){$3$}
    \node(A4)(0,0){$4$}
    %sigma 1 first on a picture
    \drawedge(A1,A2){$yx$}
    \drawedge(A2,A3){$yx$}
    \drawedge(A3,A4){$yx$}
    \drawedge[curvedepth=2.5, ELside=r, ELpos=35, ELdist=0.3](A1,A3){$\dl x$}
    \drawedge[ELpos=25, ELside=r](A2,A4){$\dl x$}
    \drawedge[curvedepth=2.5](A3,A1){}
    \drawedge(A4,A1){$\dl x, yx$}
    \end{picture}
 \end{center}
 \caption{Automata $\mathrsfs{D}$, $\mathrsfs{D}_{\sigma_1}$ and $\mathrsfs{D}_{\sigma_2}$}
 \label{fig:d5}
\end{figure}
This example shows that reduced automata may be significantly different.
\end{example}
The following simple proposition gives a way to obtain several decompositions of a given automaton. The proof of this fact is
straightforward.

\begin{proposition}
Let $\A$ be an automaton with a decomposition $\langle(X,Y),$ $(\G,\D)\rangle$. Then for any pair of permutation matrices $P$ and $Q$ of
appropriate size, $\langle(XP,P^{T}Y), (\G Q,Q^{T}\D)\rangle$ is also a decomposition of $\A$.
\end{proposition}
The next propositions show the reason of our interest in the automaton $\AD$. It turns out that the reduced automaton inherits important
properties of the original automaton $\A$.

\begin{proposition}
\label{th:1} Let $\A$ be a synchronizing automaton given by a pair of matrices $A$ and $B$. Then for every decomposition $\sigma =
\langle(X,Y), (\G,\D)\rangle$ the reduced automaton $\AD$ is synchronizing and $\rt(\AD) \leq \rt(\A) + 1$. Moreover, if the decomposition
$\sigma$ is unavoidable, then $\rt(\AD) \leq \rt(\A)$.
\end{proposition}

\begin{proof}
Since $\A$ is synchronizing, there is a matrix $W \in \{A, B\}^{\rt(\A)}$ such that it contains a column of 1's. We can represent $W$ as a
product of matrices $X,Y,\G,\D$ in accordance with $\sigma$. Thus, $W = SW'T$, where $S \in \{X,\G\}$, $T \in \{Y,\D\}$ and $W' \in
\mathcal{M}^{\rt(\A) - 1}$. Observe that the matrix $YWX$ is of order $r$ and also contains a column of 1's. Moreover, it can be
represented as a product of matrices in $\mathcal{M}$ of length $\rt(\A) + 1$. Hence, the automaton $\AD$ is synchronizing and its reset
threshold is at most $\rt(\A) + 1$. In case of an unavoidable decomposition the matrix $W'T$ has to contain a column of 1's. Thus $W'TX \in
\mathcal{M}^{\rt(\A)}$ can play a role of reset word. So, the inequality $\rt(\AD) \leq \rt(\A)$ holds true.
\end{proof}


\begin{proposition}
Let $\A$ be a strongly connected automaton given by a pair of matrices $A$ and $B$. Then for every unavoidable decomposition $\sigma =
\langle(X,Y),(\G,\D)\rangle$ the reduced automaton $\AD$ is also strongly connected.
\end{proposition}

\begin{proof}
Let us fix arbitrary $i,j \in \overline{1,r}$. Now we are to construct a path from $i$ to $j$ in automaton $\AD$. Since every row of $Y$
contains 1, there exists $i'$ such that $Y[i,i'] = 1$. The decomposition $\sigma$ is unavoidable, thus there exists $j'$ such that $X[j',j]
= 1$. Since $\A$ is strongly connected, we have a matrix $W \in \{A,B\}^*$ such that $W[i',j'] = 1$. It is clear that $YWX[i,j] = 1$.
Moreover, $YWX \in \mathcal{M}^*$. Therefore, $\AD$ is strongly connected.
\end{proof}

In order to obtain an upper bound on $\rt(\A)$ using $\rt(\AD)$, we introduce a new notion. Let $\langle \A , \mathrsfs{F} \rangle$ be a
pair of automata, where $\mathrsfs{F}$ has initial and final states. We say that $\A$ is \emph{synchronizing with constraint}
$\mathrsfs{F}$ if there is a word $w$ such that $w$ resets $\A$ and is also accepted by $\mathrsfs{F}$. The minimum length of such a word
$w$ is called the \emph{constrained reset threshold with respect to} $\mathrsfs{F}$ and is denoted by $\rtc(\A, \mathrsfs{F})$. We will
often omit the explicit reference to the automaton $\mathrsfs{F}$, since it will be known from context.

We would like to note that the constrained reset threshold can grow exponentially with the number of states in the automaton $\A$ even if
the automaton $\mathrsfs{F}$ is fixed. Such an example can be obtained by a slight modification of the 3-letter automaton
$\mathrsfs{A}^3_{pfa}(n)$ from \cite{Mart}. Letter $c$ in the modified new automaton will act as identity mapping on previously undefined
states. All defined actions of letters will remain the same. If we take the minimal automaton of the language $ba^*c$ as $\mathrsfs{F}$,
then constrained reset words coincide with carefully synchronizing words of the original automaton.

Let $\R$ be the automaton shown in Fig.~\ref{fig:r}. It has $\{S, YX, Y\G , \D X,$ $\D \G, Z\}$ the a set of states. For the sake of
readability, zero state $Z$ is not presented in the picture. For all letters $\ell \in \Sigma'$ we have $Z \cdot \ell = Z$. If the action
of a letter $\ell$ on a state $q$ is not shown in the picture then $q \cdot \ell = Z$. The state $S$ is the initial state of the automaton
$\R$ and the states $YX, Y\G , \D X, \D \G$ are final. From now on the automaton $\R$ will be the only constraint automaton we consider.

\begin{figure}[ht]
\begin{center}
 \unitlength=2.8pt
   \begin{picture}(18,26)(0,-2)
   \gasset{Nw=7,Nh=6,Nmr=3,loopdiam=5}
   \node[linewidth=0.3](A1)(0,18){$YX$}
   \node[linewidth=0.3](A2)(18,18){$Y\G$}
   \node[linewidth=0.3](A3)(18,0){$\D\G$}
   \node[linewidth=0.3](A4)(0,0){$\D X$}
   \node[Nmarks=i](AS)(-28,9){$S$}
   \drawedge(AS,A1){$yx$}
   \drawedge[ELpos=40,ELside=l,ELdist=0.5,curvedepth=-1.5](AS,A2){$y\g$}
   \drawedge[ELpos=40,ELside=r,ELdist=0,curvedepth=1.5](AS,A3){$\dl \g$}
   \drawedge[ELside=r](AS,A4){$\dl x$}
   \drawloop[loopangle=135](A1){$yx$}
   %\drawloop[loopangle=45](A2){$b$}
   \drawloop[loopangle=-45](A3){$\dl \g$}
   %\drawloop[loopangle=-135](A4){$b$}
   \drawedge[curvedepth=2,ELpos=45](A4,A2){$y \g$}
   \drawedge[curvedepth=2,ELpos=45](A2,A4){$\dl x$}
   \drawedge(A1,A2){$y\g$}
   \drawedge(A2,A3){$\dl \g$}
   \drawedge(A3,A4){$\dl x$}
   \drawedge(A4,A1){$yx$}
   \end{picture}
\end{center}
\caption{Automaton $\R$}
\label{fig:r}
\end{figure}

\begin{proposition}
\label{th:2} Let $\A$ be an automaton given by a pair of matrices $A$ and $B$. Then for every decomposition $\sigma = \langle(X,Y),
(\G,\D)\rangle$ automaton $\AD$ is synchronizing with constraint $\R$ if and only if $\A$ is synchronizing. Moreover, $\rtc(\AD) - 1 \leq
\rt(\A) \leq \rtc(\A) + 1$. If the decomposition $\sigma$ is unavoidable then $\rtc(\AD) \leq \rt(\A)$.
\end{proposition}

\begin{proof}
If $\A$ is synchronizing, then synchronizability of $\AD$ with constraint $\R$ and lower bounds on the reset threshold of $\A$ immediately
follow from the proof of Proposition~\ref{th:1}. Representations of the constructed matrices $YWX$ and $W'TX$ as a product of matrices in
$\mathcal{M}$ satisfy the constraint $\R$.

Now let the matrix $W = SW'P$ be a matrix corresponding to some constrained reset word $w$ of $\AD$, where $S \in \{Y, \D\}$ and $P \in \{
X, \G\}$. We notice that $W$ can be represented as a product of matrices in $\mathcal{M}$ that the automaton $\R$ accepts. Then it is not
hard to see that the matrix $S'WP'$ has a column of 1's and can be represented as a product of matrices $A$ and $B$, where $S',P'$ are
defined as follows:
$$
S' =
\begin{cases}
X,\;& \mbox{if } S = Y,\\
\G,\;& \mbox{if } S = \D;
\end{cases}\;\;\;
P' =
\begin{cases}
Y,\;& \mbox{if } P = X,\\
\D,\;& \mbox{if } P = \G.
\end{cases}
$$
Thus, $S'WP'$ can play the role of a reset word for $\A$ and we get $\rt(\A) \leq \rtc(\A) + 1$.
\end{proof}

Proposition~\ref{th:2} can be used in order to give a tight estimate for the \rl\ of an automaton of bounded rank. Let $\mathrsfs{E}_n$ be
defined as follows:
$$i\cdot a=\begin{cases}
2, &\text{if } i = 1,\\
3, &\text{if } i = 2,\\
i, &\text{if } i>2;
\end{cases}\quad
i \cdot b =\begin{cases}
i+1, &\text{if } i<n,\\
1, &\text{if } i=n.
\end{cases}$$
Automaton $\mathrsfs{E}_5$ is shown in Fig.~\ref{fig:e}.
\begin{figure}[hb]
\begin{center}
\unitlength .45mm
\begin{picture}(72,66)(25,-76)
\gasset{Nw=16,Nh=16,Nmr=8,loopdiam=10} \node(n0)(36.0,-16.0){2}
\node(n1)(4.0,-40.0){$1$} \node(n2)(68.0,-40.0){3} \node(n3)(16.0,-72.0){$5$}
\node(n4)(56.0,-72.0){4} \drawedge[ELdist=2.0](n1,n0){$a$}
\drawedge[ELdist=1.5](n2,n4){$b$} \drawedge[ELdist=1.7](n0,n2){$a,b$}
\drawedge[ELdist=1.7](n3,n1){$b$} \drawedge[ELdist=1.7](n1,n2){$b$}
\drawedge[ELdist=1.7](n4,n3){$b$}
\drawloop[ELdist=1.5,loopangle=30](n2){$a$}
\drawloop[ELdist=2.4,loopangle=-30](n4){$a$}
\drawloop[ELdist=1.5,loopangle=210](n3){$a$}
%\put(31,-73){$\dots$}
\end{picture}
\begin{picture}(72,66)(-25,-76)
\gasset{Nw=16,Nh=16,Nmr=8,loopdiam=10}
\node(n1)(10.0,-27.0){1} \node(n2)(60.0,-27.0){2} \node(n3)(10.0,-72.0){4}
\node(n4)(60.0,-72.0){3} \drawedge[ELdist=1.5](n2,n4){$\dl x$}
\drawedge[ELdist=1.7](n3,n1){$\dl x$} \drawedge[ELdist=1.7](n4,n3){$\dl x$}
\drawedge[ELdist=2.0](n1,n2){$\dl x, yx$} \drawloop[ELdist=1.5,loopangle=30](n2){$yx$}
\drawloop[ELdist=2.4,loopangle=-30](n4){$yx$}
\drawloop[ELdist=1.5,loopangle=210](n3){$yx$}
%\put(31,-73){$\dots$}
\end{picture}
\end{center}
\caption{The automaton $\mathrsfs{E}_5$ and its reduced automaton} \label{fig:e}
\end{figure}

Let $\langle A,B\rangle$ be the matrix representation of $\mathrsfs{E}_n$( in accordance with given numbering). We define the decomposition
$\sigma_n$ as follows: the matrix $Y$ is obtained from $A$ by deleting the second row and the matrix $\D$ is obtained from $B$ by deleting
the first row. The matrices $X, \G$ are uniquely determined by $Y$ and $\D$. For example for $n = 5$ we get:
$$\sigma_5: \;\;
A =
\left(
\begin{smallmatrix}
1&0&0&0\\
0&1&0&0\\
0&1&0&0\\
0&0&1&0\\
0&0&0&1\\
\end{smallmatrix}
\right)
\left(
\begin{smallmatrix}
0&1&0&0&0\\
0&0&1&0&0\\
0&0&0&1&0\\
0&0&0&0&1
\end{smallmatrix}
\right),\;
B =
\left(
\begin{smallmatrix}
1&0&0&0\\
1&0&0&0\\
0&1&0&0\\
0&0&1&0\\
0&0&0&1
\end{smallmatrix}
\right)
\left(
\begin{smallmatrix}
0&0&1&0&0\\
0&0&0&1&0\\
0&0&0&0&1\\
1&0&0&0&0
\end{smallmatrix}
\right).
$$
By a straightforward computation we get $YX = A_{n - 1}$, $Y\G = I_{n - 1}$, $\D X = \D\G = B_{n - 1}$, where $I_{n - 1}$ is the identity
matrix of order $n - 1$. The reduced automaton for $\mathrsfs{E}_5$ is shown on the right in Fig.~\ref{fig:e}. We have omitted the actions
of $y\g, \dl \g$ to improve readability. Notice that $\langle YX, \D\G\rangle$ is a matrix representation of the automaton $\mathrsfs{C}_{n
- 1}$. We can use this fact and Proposition~\ref{th:cerny} to estimate the constrained reset threshold. So, let $w \in \Sigma'$ be a
constrained reset word of minimal length for $\mathrsfs{E}_{\sigma_n}$. By Proposition~\ref{th:cerny} $w$ has at least $(n - 1)^2 - 3(n -
1) + 2 = n^2 - 5n + 6$ occurrences of $\dl x, \dl \g$ and $n - 2$ occurrences of $yx$. Notice that $(yx)^2$ is not a factor of $w$,
otherwise we could obtain a shorter constrained reset word by reducing $(yx)^2$ to just $yx$. Therefore, after every occurrence of $yx$,
except maybe the last one, there is an occurrence of $\dl x$, $\dl \g$ or $y \g$. Since $w$ is accepted by $\R$ we conclude that $yx$ is
followed by $y\g$. Hence, we get $|w| \geq n^2 - 5n + 6 + n - 2 + n - 3 = n^2 -3n + 1$. Proposition~\ref{th:2} implies $n^2 -3n + 1 \leq
\rt(\mathrsfs{E}_n) \leq n^2 -3n + 2$.

We notice that there is also another way to determine the reset threshold of $\mathrsfs{E}_n$. It is presented in an extended version
of~\cite{AGV} (submitted).

Now we use the series $\mathrsfs{E}_n$ as a base for constructing 2-letter automata of bounded rank $r$ with ``large'' reset threshold. The
automaton $\mathrsfs{E}_n$ has been chosen for the following reason. Computational experiments show that for small number of states $n$,
upper bound $n^2 -3n +2$ on reset threshold is optimal for 2-letter automata of bounded rank $n - 1$. Thus, we hope that generalizations of
the series $\mathrsfs{E}_n$ will have reset thresholds close to optimal.

\begin{proposition}
For every $r < n$ there is a synchronizing 2-letter $n$-state automaton of bounded rank $r$ whose \rl\ is at least $r^2 -r - 1$. If $r >
\frac{n}{2}$, then such an automaton is strongly connected.
\end{proposition}

Let us fix the number of states $n$, rank $r$ and defect $d = n - r$. The automaton $\mathrsfs{E}_n$ essentially was reduced to
$\mathrsfs{C}_{n - 1}$. Now we are going to reverse this procedure. We look for an automaton that can be reduced to $\mathrsfs{C}_r$. More
precisely, consider a solution $X,Y,\G,\D$ of the following system of matrix equations: $YX = A_{r}$, $Y\G = I_{r}$, $\D X = \D \G =
B_{r}$, such that:
\begin{enumerate}[(i)]
\item $X,\G$ are $(0,1)$-matrices of size $n$ by $r$
\item $Y,\D$ are $(0,1)$-matrices of size $r$ by $n$
\item every row of $X, Y, \G, \D$ has only one occurrence of 1
\item every column of $X$ and $\G$ contains at least one occurrence of 1.
\end{enumerate}
Arguing as before, we get $\rtc(YX, Y\G, \D X, \D \G) \geq r^2 - r - 1$. Consider the automaton $\A = \langle XY, \G\D \rangle$. It has $n$
states and the rank of each letter is bounded by $r$. Since $\langle (X,Y), (\G, \D) \rangle$ is an unavoidable decomposition of $\A$, then
by Proposition~\ref{th:2} we conclude that $\A$ is synchronizing. Moreover, its reset threshold is at least $r^2 - r - 1$. A simple
solution with required properties can be found in the following form:

$$
X =
\left(
\begin{array}{c}
X' \\
e_1 \\
A_r
\end{array}
\right),\;
Y =
\left(
\begin{array}{c}
0_{r,d} \mid I_r
\end{array}
\right),\;
\G =
\left(
\begin{array}{c}
\G' \\
e_1 \\
I_r
\end{array}
\right),\;
\D =
\left(
\begin{array}{c}
0_{r-1,d + 1} \mid I_{r-1}\\ \hline
e_d
\end{array}
\right),
$$
where $X', \G'$ are arbitrary $(0,1)$-matrices of size $d-1$ by $r$ with unique occurrence of 1
in each row. Here $0_{r,d}$ denotes zero matrix of size $r$ by $d$. This completes first part of the proof.

We need a more complicated solution in order to get a strongly connected automaton. If $r = n - 1$, then the automaton $\mathrsfs{E}_n$
satisfies the conditions of our proposition. Suppose now that $r \neq n - 1$. Then the desired solution is the following:
$$
X =
\left(
\begin{array}{c}
M \\
e_1 \\
A_r
\end{array}
\right),
Y =
\left(
\begin{array}{c}
0_{r,d} \mid I_r
\end{array}
\right),
\G =
\left(
\begin{array}{c}
M \\
e_1 \\
I_r
\end{array}
\right),
\D =
\left(
\begin{array}{c}
I_{d - 1} \mid 0_{d-1,r + 1} \\ \hline
\textbf{\,\,\,} 0_{r-d,2d} \mid I_{r - d} \\ \hline
e_d
\end{array}
\right),
$$
where $M =
\left(
\begin{smallmatrix}
e_2\\
e_3\\
\ldots\\
e_d
\end{smallmatrix}\right)$ is matrix of size $d - 1$ by $r$.
The proof that the automaton $\langle XY, \G\D\rangle$ is strongly connected is straightforward and technical (see Appendix). We omit it
due to space restrictions.

\section{Conclusion and discussion}
We suggest to study a relationship between the reset threshold of an automaton and ranks of its letters. We hope that presented techniques
will lead to a better understanding of this relationship. It may also be used to obtain new examples or in the study of existing ones. We
would like to mention several questions concerning the presented approach. First, it can be seen from example~\ref{ex} that the reduced
automaton is not uniquely defined. Is there a ``canonical'' reduction or a most ``convenient'' one? Second, different automata may have the
same reduced automaton. In other words, an automaton $\A$ is equivalent to an automaton $\mathrsfs{B}$ if $\mathrsfs{A}_{\sigma} =
\mathrsfs{B}_{\sigma'}$ for some decompositions $\sigma, \sigma'$. Essentially, this means that the problem of synchronization of $\A$
coincides with the problem of synchronization of $\mathrsfs{B}$. Is there a transparent characterization of such equivalence classes?


\begin{thebibliography}{99}
\bibitem{AS09}
Almeida, J., Steinberg, B.: Matrix mortality and the \v{C}ern\'{y}--Pin conjecture. Developments in Language Theory, Lect.\ Notes Comput.\
Sci. 5583, 67--80 (2009)

\bibitem{AGV}
Ananichev, D.S., Gusev, V.V., Volkov, M.V.: Slowly synchronizing automata and
digraphs. Mathematical Foundations of Computer Science, Lect.\ Notes Comput.\ Sci. 6281, 55--65 (2010)

\bibitem{AVZ}
Ananichev, D.S., Volkov, M.V., Zaks, Yu.I.: Synchronizing automata
with a letter of deficiency 2. Theor.\ Comput.\ Sci. 376, 30--41 (2007)

\bibitem{BBP}
B\'{e}al, M.-P., Berlinkov, M.V., Perrin, D.: A quadratic upper bound on the size of a synchronizing word
in one-cluster automata. Int. J. Found. Comput. Sci. 22(2), 277--288 (2011)

\bibitem{Ce64}
\v{C}ern\'{y}, J.: Pozn\'{a}mka k homog\'{e}nnym eksperimentom s
kone\v{c}n\'{y}mi automatami. Matematicko-fyzikalny \v{C}asopis
Slovensk.\ Akad.\ Vied 14(3) 208--216 (1964) (in Slovak)

\bibitem{Du98}
Dubuc, L.: Sur les automates circulaires et la conjecture de
\v{C}ern\'y. RAIRO Inform.\ Th\'eor.\ Appl. 32, 21--34 (1998) (in
French)

\bibitem{Gu}
Gusev, V.V.: Lower bounds for the length of reset words in eulerian automata.
Reachability Problems, Lect.\ Notes Comput.\ Sci. 6945, 180--190 (2011)

\bibitem{Ka03}
Kari, J.: Synchronizing finite automata on Eulerian digraphs.
Theoret.\ Comput.\ Sci. 295, 223--232 (2003)

\bibitem{Mart}
Martyugin, P.V.: Lower bounds for the length of the shortest carefully synchronizing
words for two- and three-letter partial automata. Diskretn. Anal. Issled. Oper., 15(4), 44--56 (2008)
(in Russian)

\bibitem{Pi83}
Pin, J.-E.: On two combinatorial problems arising from automata
theory. Ann.\ Discrete Math. 17, 535--548 (1983)

\bibitem{Rys}
Rystsov, I.K.: Estimation of the length of reset words for automata with simple idempotents.
Cybernetics and Systems Analysis 36(3), 339--344 (2000)

\bibitem{Sa05}
Sandberg, S.: Homing and synchronizing sequences. Model-Based Testing of Reactive Systems. Lect.\ Notes Comput.\ Sci. 3472, 5--33 (2005)

\bibitem{Tr07}
Trahtman, A.N.: The \v{C}ern\'y conjecture for aperiodic automata.
Discrete Math.\ Theor.\ Comput.\ Sci. 9(2), 3--10 (2007)

\bibitem{Tr11}
Trahtman, A.N.: Modifying the upper bound on the length of minimal synchronizing word. Fundamentals of Computation Theory, Lect.\ Notes
Comput.\ Sci. 6914, 173--180 (2011)

\bibitem{Vo08}
Volkov, M.V.: Synchronizing automata and the \v{C}ern\'{y}
conjecture.
%In: Mart\'\i{}n-Vide, C.; Otto, F.; Fernau, H. (eds.),
Languages and Automata: Theory and Applications, Lect.\ Notes
Comput.\ Sci. 5196, 11--27 (2008)

\bibitem{Vo09}
Volkov, M.V.: Synchronizing automata preserving a chain of partial
orders. Theoret.\ Comput.\ Sci. 410, 2992--2998 (2009)
\end{thebibliography}

\section*{Appendix}
If matrices $X,Y,\G,\D$ defined as follows:
$$
X =
\left(
\begin{array}{c}
M \\
e_1 \\
A_r
\end{array}
\right),
Y =
\left(
\begin{array}{c}
0_{r,d} \mid I_r
\end{array}
\right),
\G =
\left(
\begin{array}{c}
M \\
e_1 \\
I_r
\end{array}
\right),
\D =
\left(
\begin{array}{c}
I_{d - 1} \mid 0_{d-1,r + 1} \\ \hline
\textbf{\,\,\,} 0_{r-d,2d} \mid I_{r - d} \\ \hline
e_d
\end{array}
\right),
$$
where $M =
\left(
\begin{smallmatrix}
e_2\\
e_3\\
\ldots\\
e_d
\end{smallmatrix}\right)$ is matrix of size $d - 1$ by $r$,
then automaton $\A = \langle XY, \G\D\rangle$ is strongly connected. By direct computation we obtain:
$$
XY =
\bordermatrix{    ~  &    ~     \cr
                  _1 & e_{d+2}  \cr
                  _2 & e_{d+3}  \cr
                  _{\ldots} & _{\ldots} \cr
                  _{d - 1} & e_{2d}  \cr
                  _d & e_{d+1}  \cr
                  _{d+1} & e_{d+2}  \cr
                  _{d+2} & e_{d+2}  \cr
                  _{d+3} & e_{d+3}  \cr
                  _{d+4} & e_{d+4}  \cr
                  _{\ldots} & _{\ldots} \cr
                  _{n} & e_{n}  \cr
             }, \;\;\;\;
\G\D =
\bordermatrix{    ~  &    ~     \cr
                  _1 & e_{2}  \cr
                  _2 & e_{3}  \cr
                  _{\ldots} & _{\ldots} \cr
                  _{d - 2} & e_{d - 1}  \cr
                  _{d - 1} & e_{2d + 1}  \cr
                  _d & e_{1}  \cr
                  _{d+1} & e_{1}  \cr
                  _{d+2} & e_{2}  \cr
                  _{\ldots} & _{\ldots} \cr
                  _{2d-1} & e_{d-1}  \cr
                  _{2d} & e_{2d+1}  \cr
                  _{2d+1} & e_{2d+2}  \cr
                  _{\ldots} & _{\ldots} \cr
                  _{n-1} & e_{n}  \cr
                  _{n} & e_{d}  \cr
             }.
$$
In order to show that $\A$ is strongly connected we will present closed walks with common vertices.
$1\xrightarrow{\G\D}2\xrightarrow{\G\D}\ldots\xrightarrow{\G\D}d-1\xrightarrow{XY}2d\xrightarrow{\G\D}2d+1\xrightarrow{\G\D}\ldots
\xrightarrow{\G\D}n\xrightarrow{\G\D}d\xrightarrow{\G\D}1$. Therefore, we have that all states except maybe $\overline{d+1,2d -1}$ are in
the same strongly connected component.
$d+1\xrightarrow{XY}d+2\xrightarrow{\G\D}2\xrightarrow{XY}d+3\xrightarrow{\G\D}3\xrightarrow{XY}d+4\xrightarrow{\G\D}
4\xrightarrow{XY}\ldots\xrightarrow{\G\D}d-2\xrightarrow{XY}2d-1\xrightarrow{\G\D}d-1\leadsto d\xrightarrow{XY}d+1$. Therefore, automaton
$\A$ is strongly connected.
%For any $v \in \overline{d+1,2d -1}$ we have $v\xrightarrow{XY}d+v\xrightarrow{\G\D}v$.
\end{document}
