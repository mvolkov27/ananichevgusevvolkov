\documentclass[11pt]{llncs}
\usepackage{amsmath}
\usepackage{amssymb}
\usepackage{epic,gastex}
\usepackage{array}
\usepackage{enumerate}

\newcommand{\sa}{synchronizing automata}
\newcommand{\san}{synchronizing automaton}
\newcommand{\sw}{reset word}
\newcommand{\sws}{reset words}
\newcommand{\ssw}{reset word of minimum length}
\newcommand{\rl}{reset threshold}

\newcommand{\A}{\mathrsfs{A}}
\newcommand{\C}{\mathrsfs{C}}
\newcommand{\R}{\mathrsfs{R}}
\newcommand{\AD}{\mathrsfs{A_{\sigma}}}
\newcommand{\G}{\Gamma}
\newcommand{\D}{\Delta}
\newcommand{\g}{\gamma}
\newcommand{\dl}{\delta}
\newcommand{\rk}{rk}
\newcommand{\rt}{rt}
\newcommand{\rtc}{rt_{c}}

\DeclareMathOperator{\weight}{wg}

\DeclareSymbolFont{rsfscript}{OMS}{rsfs}{m}{n}
\DeclareSymbolFontAlphabet{\mathrsfs}{rsfscript}

\begin{document}

\title{Synchronizing automata of bounded rank\thanks{Supported
by the Russian Foundation for Basic Research, grants xx-xx-xxxxx
and xx-xx-xxxxx, and by the Federal Education Agency of Russia,
grant 2.1.1/3537.}}

\titlerunning{Synchronizing automata of bounded rank}

\author{Vladimir V. Gusev}

\authorrunning{V. V. Gusev}

\tocauthor{V. V. Gusev(Ekaterinburg, Russia)}

\institute{Institute of Mathematics and Computer Science,\\
Ural Federal University, Ekaterinburg, Russia\\
\email{vl.gusev@gmail.com}}

\maketitle

\begin{abstract}
We reduce a problem of synchronization of $n$-state automaton
whose letters have rank $r < n$ to synchronization of $r$-state 
automaton with constraints given by regular language.
As an illustration of our technique we present for every 
$r \geq \frac{n}{2}$ and $n$ strongly-connected $n$-state \sa\
of bounded rank $r$ with \rl equal to $r(r - 1)$.
\end{abstract}

\section{Introduction}
A complete deterministic finite automaton $\mathrsfs{A}$ is called
\emph{synchronizing} if the action of some word $w$ resets $\mathrsfs{A}$, that
is, leaves the automaton in one particular state no matter at which state it is
applied. Any such word $w$ is said to be a \emph{reset word} for the automaton.
The minimum length of reset words for $\mathrsfs{A}$ is called the \emph{\rl}
of $\mathrsfs{A}$ and denoted by $\rt(\mathrsfs{A})$. Synchronizing automata
constitute an interesting object and serve as natural models of error-resistant
systems. For a brief introduction to the theory of synchronizing automata we refer
the reader to surveys~\cite{Sa05,Vo08}. 
%WANT SOME LYRICS ABOUT DIFFICULT PROBLEMS HERE
The interest to the field is
also heated by the famous \v{C}ern\'{y} conjecture.

In~1964 Jan \v{C}ern\'{y}~\cite{Ce64} constructed for each $n>1$ a
synchronizing automaton $\mathrsfs{C}_n$ with $n$ states whose \rl\ is equal to
$(n-1)^2$. Soon after that he conjectured that these automata represent the
worst possible case, that is, every synchronizing automaton with $n$ states can
be reset by a word of length $(n-1)^2$. Despite intensive research, the best
upper bound on the \rl\ of synchronizing automata with $n$ states achieved so
far is $\frac{n^3-n}6$, see~\cite{Pi83}, so it is much larger than the
conjectured value. There is also bound by Trahtman that contains unclear place~\cite{Tr11}.
Though the \v{C}ern\'{y} conjecture is open in general, it
has been confirmed for some restricted classes of synchronizing automata.
We will mention only some of them.
Dubuc~\cite{Du98} has proved conjecture for automata that have a letter acting as cyclic permutation.
Kari~\cite{Ka03} confirmed conjecture for automata with eulerian undeliyng digraph.

For some classes there is quadratic upper bound. For example, Berlinkov, Beal, Perrin~\cite{BBP} obtained
such bound for automata that has one letter whoose underlying digraph is connected.
Rystsov~\cite{Rys} obtained quadratic upper bound in case when every letter is either permutation or
idemponent of defect 1.

In present paper we are interested in synchronization automata of bounded rank.
%cases of bounded rank in literature
We can ask following general question: Given a synchronizing $n$-state automaton $\A$
with $k$ letters. Let $r_1, r_2, \ldots, r_k$ be their corresponding ranks. What are the
bounds on \rt of $\A$ in terms of $n$ and ranks?
Of course upper bound is difficult, but sometimes possible(Rys).
We can rewrite result of Cerny and Rystsov from this point of view. Volkov Zaks also.
You result from mfcs also.
Here we present a bit different lower bound.

Natural desire in Cerny conjecture is to use induction by number of states. But most(all?) of the proofs 
does not do this. Our approach in some sence gives such opportunity.

\section{Preliminaries}
Let $\A = (Q, \Sigma)$ be a totally-defined deterministic finite automaton. Let us fix some orderings 
of $Q = \{q_1, q_2, \ldots q_n\}$ and $\Sigma = \{a_1, a_2, \ldots, a_k\}$.
We can associate with arbitrary letter $a_\ell$
square $(0,1)$-matrix $M(a_\ell)$ of order $n$ by the following rule:
$M(a_\ell)[i,j] = 1$ if $q_i\cdot a_{\ell} = q_j$, otherwise $M(a_\ell)[i,j] = 0$.
We will call the set $\langle M(a_1), \ldots, M(a_{k})\rangle$  \emph{matrix representation} of automaton $\A$,
and will denote this fact $\A = \langle M(a_1), \ldots, M(a_{k})\rangle$. We can uniquely extend domain of mapping $M(\cdot )$ 
from $\Sigma$ to $\Sigma^{*}$ in accordance with equation $M(uv) = M(u)M(v)$, where $u,v \in \Sigma^*$.
It is not hard to see following important property of matrix representation: for every word $w$ we have
$M(w)[i,j] = 1$ if $q_i \cdot w = q_j$, otherwise $M(w)[i,j] = 0$. It immediately implies that
a word $w$ is a reset word of $\A$ if and only if $M(w)$ has column of 1's.

Let us introduce pair of square matrices $A_n$ and $B_n$ of order $n$. We will denote $i$-th row of matrix
$A_n$ as $A_n[i,.]$, and $e_i$ denote row vector whose only non-zero entry equal to 1 located in position $i$.
$$ A_n[i,.] = 
\begin{cases}
e_2, &\text{if } i = 1\\
e_i, &\text{if } i \neq 1\\
\end{cases},\quad
B_n[i,.] = 
\begin{cases}
e_{i+1}, &\text{if } i \neq n\\
e_1, &\text{if } i = n\\
\end{cases}.
$$
We will refer to automaton $\langle A_n, B_n \rangle$ as \v{C}ern\'{y} automaton and denote it $\C_n$.
Matrix $A_n$ corresponds to action of letter $a$, matrix $B_n$ corresponds to action of letter $b$.
Later on we will require following statement concearning reset words of $\C_n$(Proposition 3 in~\cite{Gu}).

\begin{proposition}
\label{th:cerny} Every reset word of the automaton $\mathrsfs{C}_n$ contains at
least $n^2 - 3n + 2$ occurrences of the letter $b$ and at least $n - 1$
occurrence of the letter $a$.
\end{proposition}
%Consider a square $(0,1)$-matrix $M$ of order $n$ and of rank $r$ such that every row of $M$
%contains the only occurence of 1. Note, that matrices corresponding to actions of letters in
%an automaton have this property. We will call representation $M=XY$, where $X,Y$ are $(0,1)$-matrices of size 
%$n$ by $k$ and $k$ by $n$ respectively.
%
%
%Then it is well known that matrix $M$ can be
%represented as $M=XY$, where $X,Y$ are $(0,1)$-matrices of size $n$ by $k$ and $k$ by $n$
%respectively with the following property:
%every row of them contains exactly one 1.

\section{Main results}
For the sake of simplicity throughout this section we will consider automata only
over two letters. All definitions and propositions can be easily generalized.
Consider an $n$-state automaton $\A$ given by pair of matrices $A$ and $B$.
We say that set of matrix pairs $\sigma = \langle(X,Y), (\G,\D)\rangle$ is
a \emph{decomposition} of automaton $\A$ if:
\begin{enumerate}[(i)]
\item $X, \G$ are $(0,1)$-matrices of size $n$ by $k$
\item $Y, \D$ are $(0,1)$-matrices of size $k$ by $n$
\item $A = XY$, $B = \G\D$
\item every row of $X,Y,\G,\D$ has only one occurence of 1.
\end{enumerate}
We say that decomposition is \emph{unavoiding} if also every column of $X$ and 
$\G$ contains at least one occurence of 1.
%a \emph{decomposition} of automaton $\A$ if:\\
%$\bullet$ $X, \G$ are $(0,1)$-matrices of size $n$ by $k$ and $Y, \D$ are
%$(0,1)$-matrices of size $k$ by $n$.\\
%$\bullet$ $A = XY$, $B = \G\D$.\\
%$\bullet$ every row of $X,Y,\G,\D$ has only one occurence of 1.\\
%We say that decomposition is \emph{unavoiding} if also every column of $X$ and 
%$\G$ contains at least one occurence of 1.

It immediately follows from the definition that $k \geq max\{\rk(A),$ $\rk(B)\}$.
And for any such $k$ we can easily construct some decomposition. 
For example, matrix $Y$ may consist of different rows of $A$ in lexicographic order.
If $k$ is bigger than number of different rows in $A$ then lexicografically largest row 
appear several times. Let $j$ be the position of $i$th row of $A$ in lexicografic order.
Then $i$th row of $X$ is equal to $e_j$. It is not hard to see that defining in the 
same way matrices $\G$ and $\D$ gives us decomposition of $\A$. Moreover,
if $k = \rk(A) = \rk(B)$ then presented decomposition is unavoiding. Note, that even in case
$k = \rk(A) = \rk(B)$ there could exist several different decompositions.

%For example, 
%matrices $Y$ and $\D$ may consist of different rows of $A$ and $B$ respectively
%in lexicographic oreder. If $k$ is bigger than number of different rows in $A$ or $B$
%then lexicografically largest row may appear several times.
Suppose now that we are given decomposition $\sigma = \langle(X,Y), (\G,\D)\rangle$ of 
automaton $\A$, where $X$ has size $n$ by $k$.
%Suppose now that we have decomposition $A = XY$, $B = \G\D$, where
%$X$, $\G$ are $(0,1)$-matrices of size $n$ by $k$, and $Y$, $\D$ are 
%$(0,1)$-matrices of size $k$ by $n$ with the following property:
%every row of them contains exactly one 1.
%There are many such decompositions. Note, that rank of matrices $A$
%and $B$ can not be smaller than $k$. We will pay special attention
%to the following natural case. If $\rk(A) = \rk(B) = k$ then
%$Y, \D$ consist of different rows of $A$ and $B$ in some order.
%Matrices $X, \G$ are chosen accordingly.
%Let us define automaton $\AD$ that is going to be key object of current paper.
%Now we are ready to define automaton $\AD$ using decomposition
%of $A$ and $B$. 
We are to define \emph{reduced automaton} $\AD$ that is going to be key object of this paper.
Automaton $\AD$ has $\{1,\ldots, k\}$ as a state
set. Action of input alphabet $\Sigma' = \{ y\g, yx, \dl x, \dl\g \}$ (these are letters, 
but expressed as two symbols for convenience) is given
by corresponding $(0,1)$-matrices $\mathcal{M} =\{ Y\G , YX, \D X, \D\G \}$ of size $k$ by $k$.
Note, that every row of matrices $X, Y, \G, \D$ contains only one occurence of 1, 
so their products have the same property. Thus automaton $\AD$
is deterministic and totally-defined.

%It is not hard to see that action of
%$\Sigma'$ is deterministic and totally-defined. Since every row of matrices
%$X, Y, \G, \D$ contain exactly one occurence of 1, thus their products have the same property.

\begin{example}
Let $\mathrsfs{D}$ be an automaton depicted on fig.~\ref{fig:d5} on the left
(firstly appeared in \cite{AGW} as $\mathrsfs{D}''_5$).
%Consider automaton $\mathrsfs{D}$ from fig.~\ref{fig:d5}, where 
%Consider automaton $\mathrsfs{D''}_5 = \langle A, B \rangle$, where 
%$A = \left(\begin{smallmatrix}0&1&0&0&0\\0&0&1&0&0\\0&0&0&1&0\\0&0&0&0&1\\0&1&0&0&0\end{smallmatrix}\right)$
%and $B = \left(\begin{smallmatrix}0&0&1&0&0\\0&0&1&0&0\\0&0&0&1&0\\0&0&0&0&1\\1&0&0&0&0\end{smallmatrix}\right)$.
%Let us define two decomposition $\sigma_1$ as follows: 
We can define two decompositions of it as follows: 
$$
\sigma_1:\;\;
A = 
\left(
\begin{smallmatrix}
1&0&0&0\\
0&1&0&0\\
0&0&1&0\\
0&0&0&1\\
1&0&0&0
\end{smallmatrix}
\right)
\left(
\begin{smallmatrix}
0&1&0&0&0\\
0&0&1&0&0\\
0&0&0&1&0\\
0&0&0&0&1
\end{smallmatrix}
\right),\;
B = 
\left(
\begin{smallmatrix}
1&0&0&0\\
1&0&0&0\\
0&1&0&0\\
0&0&1&0\\
0&0&0&1
\end{smallmatrix}
\right)
\left(
\begin{smallmatrix}
0&0&1&0&0\\
0&0&0&1&0\\
0&0&0&0&1\\
1&0&0&0&0
\end{smallmatrix}
\right);
$$

%Decomposition $\sigma_2$ has the same representation for $A$, but
$$
\sigma_2:\;\;
A = 
\left(
\begin{smallmatrix}
1&0&0&0\\
0&1&0&0\\
0&0&1&0\\
0&0&0&1\\
1&0&0&0
\end{smallmatrix}
\right)
\left(
\begin{smallmatrix}
0&1&0&0&0\\
0&0&1&0&0\\
0&0&0&1&0\\
0&0&0&0&1
\end{smallmatrix}
\right),\;
B = 
\left(
\begin{smallmatrix}
0&1&0&0\\
0&1&0&0\\
0&0&1&0\\
0&0&0&1\\
1&0&0&0
\end{smallmatrix}
\right)
\left(
\begin{smallmatrix}
1&0&0&0&0\\
0&0&1&0&0\\
0&0&0&1&0\\
0&0&0&0&1
\end{smallmatrix}
\right).
$$
Reduced automaton $\mathrsfs{D}_{\sigma_1}$ depicted in the middle of fig.~\ref{fig:d5}. Letter
$y \g$ is omitted, since it acts as identitity mapping. Action of the letter $\dl \g$ coincides with
the action of $yx$.
%For $\sigma_1$ action of letters $yx = \dl \g$ is cyclic permutation and $y \g$ is identity
%mapping.
Reduced automaton $\mathrsfs{D}_{\sigma_2}$ depicted on the right of fig.~\ref{fig:d5}. Since actions
of letters $yx, y\g, \dl \g$ are the same, we left only $yx$ on the picture.
%Note that for $\sigma_2$ action of letters $yx = y\g = \dl \g$ is cyclic permutation.
\begin{figure}[ht]
 \begin{center}
  \unitlength=2.8pt
    \begin{picture}(18,26)(14,-24)
    \gasset{Nw=6,Nh=6,Nmr=3}
	\node(n0)(0,0){2}
	\node(n1)(-13,-9){1} 
	\node(n2)(13,-9){3}
	\node(n3)(-8,-23){5} 
	\node(n4)(8,-23){4}
	\drawedge(n1,n0){$a$} \drawedge(n2,n4){$a,b$}
	\drawedge(n0,n2){$a,b$} \drawedge[ELpos=35](n3,n0){$a$}
	\drawedge(n3,n1){$b$} \drawedge[ELpos=25](n1,n2){$b$}
	\drawedge(n4,n3){$a,b$}
    \end{picture}
 \begin{picture}(18,26)(-45,-2)
    \gasset{Nw=6,Nh=6,Nmr=3, loopdiam=5}
    \node(A1)(0,18){$1$}
    \node(A2)(18,18){$2$}
    \node(A3)(18,0){$3$}
    \node(A4)(0,0){$4$}
    %sigma 2 second on a picture
    \drawloop[loopangle=135](A1){$\dl x$}
    \drawedge(A1,A2){$yx$}
%    \drawedge(A2,A3){}
%    \drawedge(A3,A4){}
%    \drawedge(A4,A1){}
    \drawedge(A2,A3){$\dl x, yx$}
    \drawedge(A3,A4){$\dl x, yx$}
    \drawedge(A4,A1){$\dl x, yx$}
    \end{picture}
 \begin{picture}(18,26)(15,-2)
    \gasset{Nw=6,Nh=6,Nmr=3}
    \node(A1)(0,18){$1$}
    \node(A2)(18,18){$2$}
    \node(A3)(18,0){$3$}
    \node(A4)(0,0){$4$}
    %sigma 1 first on a picture
    \drawedge(A1,A2){$yx$}
    \drawedge(A2,A3){$yx$}
    \drawedge(A3,A4){$yx$}
    \drawedge[curvedepth=2.5, ELside=r, ELpos=35, ELdist=0.3](A1,A3){$\dl x$}
    \drawedge[ELpos=25, ELside=r](A2,A4){$\dl x$}
    \drawedge[curvedepth=2.5](A3,A1){}
    \drawedge(A4,A1){$\dl x, yx$}
    \end{picture}
 \end{center}
 \caption{Automata $\mathrsfs{D}$, $\mathrsfs{D}_{\sigma_1}$ and $\mathrsfs{D}_{\sigma_2}$}
 \label{fig:d5}
\end{figure}
This example shows that reduced automata may be significantly different.
%As example shows reductions for different matrix decompositions can be different. Later
%on we will describe some way of obtaining more decompositions. So, we have freedom of choise.
\end{example}
Following simple proposition gives a way to obtain several decompositions of automaton $\A$.
Proof of this fact is straightforward.

\begin{proposition}
Let $\A$ be an automaton with decomposition $\langle(X,Y), (\G,\D)\rangle$.
Then for any pair of permutation matrices $P$ and $Q$ of appropriate size $\langle(XP,P^{T}Y), (\G Q,Q^{T}\D)\rangle$ is
also a decomposition of $\A$.
\end{proposition}
Next propositions show the reason of our interest in 
automaton $\AD$. It appears, that reduced automaton inherits important properties of the original automaton $\A$.

\begin{proposition}
\label{th:1}
Let $\A$ be a synchronizing automaton given by pair of matrices $A$ and $B$.
Then for every decomposition $\sigma = \langle(X,Y), (\G,\D)\rangle$ reduced automaton $\AD$ is synchronizing
and $\rt(\AD) \leq \rt(\A) + 1$. Moreover, if decomposition $\sigma$ is unavoiding then $\rt(\AD) \leq \rt(\A)$.
%Given a synchronizing automaton $\A = \langle A,B \rangle$. For every decomposition $\sigma$
%automaton $\AD$ is synchronizing and $\rt(\AD) \leq \rt(\A) + 1$. If decomposition is indispensable
%(or unavoidable, compulsory, wholesome) then $\rt(\AD) \leq \rt(\A)$.
\end{proposition}

\begin{proof}
Since $\A$ is synchronizing then there is a matrix $W \in \{A, B\}^{\rt(\A)}$ such
that it contains a column of 1's. We can represent $W$ as product of matrices $X,Y,\G,\D$ in 
accordance with $\sigma$. Thus, $W = SW'T$, where $S \in \{X,\G\}$, $T \in \{Y,\D\}$ and $W' \in \mathcal{M}^{\rt(\A) - 1}$.
Note, that matrix $YWX$ is of order $k$ and also contains a column of 1's. Moreover, it can be represented
as a product of matrices in $\mathcal{M}$ of length $\rt(\A) + 1$. Hence, automaton $\AD$ is
synchronizing and its reset threshold is at most $\rt(\A) + 1$.
In case of unavoiding decomposition matrix $W'T$ has to contain a column of 1's.
%If decomposition is indispensable then matrix $W'T$ also contains column of 1's. 
Thus $W'TX \in \mathcal{M}^{\rt(\A)}$ can play a role of reset word. So, the inequality $\rt(\AD) \leq \rt(\A)$ holds true.
\end{proof}


%Introduced reduction also preserves strong connectivity of $\A$.
\begin{proposition}
Let $\A$ be a strongly-connected automaton given by pair of matrices $A$ and $B$.
Then for every unavoiding decomposition $\sigma = \langle(X,Y),(\G,\D)\rangle$ reduced automaton $\AD$
is also strongly-connected.
%Given a strongly-connected automaton $\A = \langle A,B \rangle$. For every indispensable decomposition
%$\sigma$ automaton $\AD$ is strongly-connected.
\end{proposition}

\begin{proof}
Let us fix arbitrary $i,j \in \overline{1,k}$. Now we are to construct a path from $i$ to $j$ 
in automaton $\AD$. Since every row of $Y$ contains 1 there
exists $i'$ such that $Y[i,i'] = 1$. Decomposition $\sigma$ is unavoiding, thus
there exists $j'$ such that $X[j',j] = 1$. Since $\A$ is strongly-connected we have matrix 
$W \in \{A,B\}^*$ such that $W[i',j'] = 1$. It is clear that $YWX[i,j] = 1$. Moreover,
$YWX \in \mathcal{M}^*$. Therefore, $\AD$ is strongly-connected.
%Now it is straightforward that $YWX$ can be represented
%as composition of $Y\G , YX, \D X, \D\G$ and $YWX[i,j] = 1$. Thus, $\AD$ is strongly-connected.
\end{proof}
In order to obtain upper bound on $\rt(\A)$ using $\rt(\AD)$ we will introduce a new notion.
%Theorem~\ref{th:1} gives only lower bound for the reset threshold. We need to modify problem itself
%in order to have also upper bound. Actually, we will suggest problem almost the same, as to sychronize
%original automaton.
Let $\langle \A , \mathrsfs{F} \rangle$ be a pair of automata, where $\mathrsfs{F}$ has initial and final states. 
We say that $\A$ is \emph{synchronizing with constraint} $\mathrsfs{F}$ if there is a word $w$ such that
$w$ resets $\A$ and is also accepted by $\mathrsfs{F}$. The length of the shortest such
word $w$ we will call \emph{constrained reset threshold with respect to} $\mathrsfs{F}$ and denote it as $\rtc(\A, \mathrsfs{F})$.
We will often omit explicit mention of automaton $\mathrsfs{F}$, since it will be known from context.

We would like to note that constrained reset threshold can grow exponentially on number of states in automaton $\A$, even
if automaton $\mathrsfs{F}$ is fixed. Such example can be obtained by a slight modification of 3-letter automaton $\mathrsfs{A}^3_{pfa}(n)$
from \cite{Mart}. Letter $c$ in a new automaton will act as identity mapping on previously undefined states. All defined actions
of letters will remain the same. %Minimal automaton of language $ba^*c$ is taken as an automaton $\mathrsfs{F}$.
If we take minimal automaton of language $ba^*c$ as an automaton $\mathrsfs{F}$ then constrained reset word will coincide 
with carefully synchronizing word of original automaton.

%During our consideration $\mathrsfs{F}$ usually will be known from context, thus often we will omit explicit mention
%of $\mathrsfs{F}$. We will say constrained reset threshold and denote it by $\rtc(\A)$. DISCSSION ON EXPONENTIAL
%UPPER BOUND.

Let $\R$ be an automaton shown in fig.~\ref{fig:r}. It has $\{S, YX, Y\G , \D X, \D \G, Z\}$ as a set of states.
For the sake of clarity zero state $Z$ is not present in figure. For all letters $\ell \in \Sigma'$ we have 
$Z \cdot \ell = Z$. If action of a letter $\ell$ on a state $q$ is not shown in picture then $q \cdot \ell = Z$.
State $S$ is initial state of automaton $\R$. States $YX, Y\G , \D X, \D \G$ are final.

\begin{figure}[ht]
\begin{center}
 \unitlength=2.8pt
   \begin{picture}(18,26)(0,-2)
   \gasset{Nw=7,Nh=6,Nmr=3,loopdiam=5}
   \node[linewidth=0.3](A1)(0,18){$YX$}
   \node[linewidth=0.3](A2)(18,18){$Y\G$}
   \node[linewidth=0.3](A3)(18,0){$\D\G$}
   \node[linewidth=0.3](A4)(0,0){$\D X$}
   \node[Nmarks=i](AS)(-28,9){$S$}
   \drawedge(AS,A1){$yx$}
   \drawedge[ELpos=40,ELside=l,ELdist=0.5,curvedepth=-1.5](AS,A2){$y\g$}
   \drawedge[ELpos=40,ELside=r,ELdist=0,curvedepth=1.5](AS,A3){$\dl \g$}
   \drawedge[ELside=r](AS,A4){$\dl x$}
   \drawloop[loopangle=135](A1){$yx$}
   %\drawloop[loopangle=45](A2){$b$}
   \drawloop[loopangle=-45](A3){$\dl \g$}
   %\drawloop[loopangle=-135](A4){$b$}
   \drawedge[curvedepth=2,ELpos=45](A4,A2){$y \g$}
   \drawedge[curvedepth=2,ELpos=45](A2,A4){$\dl x$}
   \drawedge(A1,A2){$y\g$}
   \drawedge(A2,A3){$\dl \g$}
   \drawedge(A3,A4){$\dl x$}
   \drawedge(A4,A1){$yx$}
   \end{picture}
\end{center}
\caption{Automaton $\R$}
\label{fig:r}
\end{figure}
\begin{proposition}
\label{th:2}
Let $\A$ be a synchronizing automaton given by pair of matrices $A$ and $B$.
Then for every decomposition $\sigma = \langle(X,Y), (\G,\D)\rangle$ automaton $\AD$ is synchronizing
with constraint $\R$ and $\rtc(\AD) - 1 \leq \rt(\A) \leq \rtc(\A) + 1$. 
Moreover, if decomposition $\sigma$ is unavoiding then $\rtc(\AD) \leq \rt(\A)$.
%Given a synchronizing automaton $\A = \langle A,B \rangle$. For every decomposition $\sigma$
%automaton $\AD$ is synchronizing with constraint $\R$ and $\rtc(\AD) - 1 \leq \rt(\A) \leq \rtc(\A) + 1$. 
%If decomposition is indispensable
%(or unavoidable, compulsory, wholesome) then lower bound can be improved $\rtc(\AD) \leq \rt(\A)$.
\end{proposition}

\begin{proof}
Synchronizability of $\AD$ with constraint $\R$ and lower bounds on reset threshold of $\A$ immediately 
follow from the proof of proposition~\ref{th:1}.
Representations of matrices $YWX$ and $W'TX$ as a product of matrices in $\mathcal{M}$ satisfy the
constraint $\R$.
%Lower bound easily follows from proof of theorem~\ref{th:1}. Matrices constructed there satisfy
%constraint $\R$.
Now let matrix $W = SW'P$ be a matrix corresponding to some constrained reset word $w$ of $\AD$, where $S \in \{Y, \D\}$
and $P \in \{ X, \G\}$. Note that $W$ can be represented as a product of matrices in $\mathcal{M}$ that automaton $\R$ 
accepts.
Then it is not hard to see that matrix $S'WP'$ has a column of 1's and can be represented as
a product of matrices $A$ and $B$, where $S',P'$ are defined as follows:
$$
S' = 
\begin{cases}
X,\;& \mbox{if } S = Y,\\
\G,\;& \mbox{if } S = \D;
\end{cases}\;\;\;
P' = 
\begin{cases}
Y,\;& \mbox{if } P = X,\\
\D,\;& \mbox{if } P = \G.
\end{cases}
$$
Thus, $S'WP'$ can play a role of reset word for $\A$ and we get $\rt(\A) \leq \rtc(\A) + 1$.
%Then matrix $\overline{S}W\overline{P}$ has column of 1's and can be represented as
%a product of matrices $A$ and $B$. Thus, $\rt(\A) \leq \rtc(\A) + 1$.
\end{proof}

Proposition~\ref{th:2} can be used in order to tightly estimate \rl\ of automaton of bounded rank.
Let $\mathrsfs{E}_n$ be defined as follows:
$$i\cdot a=\begin{cases}
2, &\text{if } i = 1,\\
3, &\text{if } i = 2,\\
i, &\text{if } i>2;
\end{cases}\quad
i \cdot b =\begin{cases}
i+1, &\text{if } i<n,\\
1, &\text{if } i=n.
\end{cases}$$
Automaton $\mathrsfs{E}_5$ is shown in fig.~\ref{fig:e}.
\begin{figure}[hb]
\begin{center}
\unitlength .45mm
\begin{picture}(72,66)(25,-76)
\gasset{Nw=16,Nh=16,Nmr=8,loopdiam=10} \node(n0)(36.0,-16.0){2}
\node(n1)(4.0,-40.0){$1$} \node(n2)(68.0,-40.0){3} \node(n3)(16.0,-72.0){$5$}
\node(n4)(56.0,-72.0){4} \drawedge[ELdist=2.0](n1,n0){$a$}
\drawedge[ELdist=1.5](n2,n4){$b$} \drawedge[ELdist=1.7](n0,n2){$a,b$}
\drawedge[ELdist=1.7](n3,n1){$b$} \drawedge[ELdist=1.7](n1,n2){$b$}
\drawedge[ELdist=1.7](n4,n3){$b$}
\drawloop[ELdist=1.5,loopangle=30](n2){$a$}
\drawloop[ELdist=2.4,loopangle=-30](n4){$a$}
\drawloop[ELdist=1.5,loopangle=210](n3){$a$} 
%\put(31,-73){$\dots$}
\end{picture}
\begin{picture}(72,66)(-25,-76)
\gasset{Nw=16,Nh=16,Nmr=8,loopdiam=10}
\node(n1)(10.0,-27.0){1} \node(n2)(60.0,-27.0){2} \node(n3)(10.0,-72.0){4}
\node(n4)(60.0,-72.0){3} \drawedge[ELdist=1.5](n2,n4){$\dl x$}
\drawedge[ELdist=1.7](n3,n1){$\dl x$} \drawedge[ELdist=1.7](n4,n3){$\dl x$}
\drawedge[ELdist=2.0](n1,n2){$\dl x, yx$} \drawloop[ELdist=1.5,loopangle=30](n2){$yx$}
\drawloop[ELdist=2.4,loopangle=-30](n4){$yx$}
\drawloop[ELdist=1.5,loopangle=210](n3){$yx$} 
%\put(31,-73){$\dots$}
\end{picture}
\end{center}
\caption{Automaton $\mathrsfs{E}_5$ and its reduced automaton}
\label{fig:e}
\end{figure}

Let $\langle A,B\rangle$ be matrix representation of $\mathrsfs{E}_n$(in accordance with given numbering).
We define decomposition $\sigma_n$ as follows: matrix $Y$ is obtained from $A$ by deleting second row and matrix $\D$
obtained from $B$ by deleting first row. %Since $\sigma_n$ is decomposition, 
Matrices $X, \G$ are uniquely 
determined by $Y$ and $\D$.
For example for $n = 5$ we get: 
$$\sigma_5: \;\;
A = 
\left(
\begin{smallmatrix}
1&0&0&0\\
0&1&0&0\\
0&1&0&0\\
0&0&1&0\\
0&0&0&1\\
\end{smallmatrix}
\right)
\left(
\begin{smallmatrix}
0&1&0&0&0\\
0&0&1&0&0\\
0&0&0&1&0\\
0&0&0&0&1
\end{smallmatrix}
\right),\;
B = 
\left(
\begin{smallmatrix}
1&0&0&0\\
1&0&0&0\\
0&1&0&0\\
0&0&1&0\\
0&0&0&1
\end{smallmatrix}
\right)
\left(
\begin{smallmatrix}
0&0&1&0&0\\
0&0&0&1&0\\
0&0&0&0&1\\
1&0&0&0&0
\end{smallmatrix}
\right).
$$
By straightforward computation we get $YX = A_{n - 1}$, $Y\G = I_{n - 1}$, $\D X = \D\G = B_{n - 1}$,
where $I_{n - 1}$ is idetity matrix of order $n - 1$.
Reduced automaton for $\mathrsfs{E}_5$ is shown on the right in fig.~\ref{fig:e}.
We have omitted actions of $y\g, \dl \g$ for the sake of clarity.
Note that $\langle YX, \D\G\rangle$ is a matrix representation of automaton $\mathrsfs{C}_{n - 1}$. We can use this fact
and proposition~\ref{th:cerny} to estimate constrained reset threshold. 
So, let $w \in \Sigma'$ be constrained reset word of minimal length for $\mathrsfs{E}_{\sigma_n}$.
By proposition~\ref{th:cerny} $w$ has at least $(n - 1)^2 - 3(n - 1) + 2 = n^2 - 5n + 6$ occurences of $\dl x, \dl \g$ and $n - 2$
occurences of $yx$. Note that $(yx)^2$ is not a factor of $w$. Otherwise we can obtain shorter constrained reset word by 
reducing $(yx)^2$ to just $yx$. Therefore, after every occurence of $yx$, except maybe the last one, there is occurence 
of $\dl x$, $\dl \g$ or $y \g$. Since $w$ is accepted by $\R$ we conclude that $yx$ is followed by $y\g$.
Hence, we get $|w| \geq n^2 - 5n + 6 + n - 2 + n - 3 = n^2 -3n + 1$.
%Note that after every merge maybe except the last one there is identity mapping due to constraints. Then theorem~\ref{th:cerny}
%implies	$\rtc(\mathrsfs{E}_{\sigma_n}) \geq (n - 1)^2 - 3(n - 1) + 2 + (n - 2) + (n - 3) = n^2 -3n + 1$. And this bound is tight since
%the word $(mic^{n - 2})^{n - 3}m$ is constraint reset word. 
And proposition~\ref{th:2} implies $n^2 -3n + 1 \leq \rt(\mathrsfs{E}_n) \leq n^2 -3n + 2$. 
There is another way to determine reset threshold of $\mathrsfs{E}_n$. Soon it will be presented in
extended version of~\cite{AGV}.

Now we will use series $\mathrsfs{E}_n$ as a base for constructing 2-letter automata of bounded
rank $r$ with "large" reset threshold. Automaton $\mathrsfs{E}_n$ was chosen for the following reason.
Computational experiments show that for small number of states $n$ upper bound $n^2 -3n +2$ on reset threshold is optimal
for 2-letter automata of bounded rank $n - 1$.
Thus, we hope that generalizations of series $\mathrsfs{E}_n$ will have reset thresholds close to optimal.
%Now we are going to generalize series $\mathrsfs{E}_n$ on other ranks. In other words, we will find series of automata
%of bounded rank with two letters that can be reduced to the automaton $\mathrsfs{C}_k$.

\begin{proposition}
For every $r < n$ there is a synchronizing 2-letter $n$-state automaton of bounded rank $r$ whose \rl\ is at least 
$r^2 -r - 1$. If $r \geq \frac{n}{2}$ then such automaton is strongly-connected.
\end{proposition}

Let us fix number of states $n$ and rank $r$. Automaton $\mathrsfs{E}_n$ essentially was reduced to $\mathrsfs{C}_{n - 1}$.
Now we are going to reverse this procedure. We will look for automaton that can be reduced to $\mathrsfs{C}_r$.
More precisely, consider a solution $X,Y,\G,\D$ of the following system of matrix equations:
$YX = A_{r}$, $Y\G = I_{r}$, $\D X = \D \G = B_{r}$, such that:
\begin{enumerate}[(i)]
\item $X,\G$ are $(0,1)$-matrices of size $n$ by $r$
\item $Y,\D$ are $(0,1)$-matrices of size $r$ by $n$
\item every row of $X, Y, \G, \D$ has only one occurence of 1
\item every column of $X$ and $\G$ contains at least one occurence of 1.
\end{enumerate}
Arguing as before we get $\rtc(YX, Y\G, \D X, \D \G) \geq r^2 - r - 1$.
Consider automaton $\A = \langle XY, \G\D \rangle$. It has $n$ states and rank of each letter is bounded by $r$.
Since $\langle (X,Y), (\G, \D) \rangle$ is unavoiding decomposition of $\A$ then by proposition~\ref{th:2} we 
conclude that $\A$ is synchronizing.
Moreover, its reset threshold is at least $r^2 - r - 1$.
Simple solution with required properties can be found in the following form:

$$
X =
\left(
\begin{array}{c}
X' \\
e_1 \\
A_r
\end{array}
\right),\;
Y = 
\left(
\begin{array}{c}
0_d \mid I_r
\end{array}
\right),\;
\G = 
\left(
\begin{array}{c}
\G' \\
e_1 \\
I_r
\end{array}
\right),\;
\D = 
\left(
\begin{array}{c}
0_{d + 1} \mid I_{r-1}\\ \hline
e_d
\end{array}
\right),
$$
where $X', \G'$ are arbitrary $(0,1)$-matrices of size $d-1$ by $r$ with unique occurence of 1
in each row. Here $0_{d}$ denotes zero matrix of order $d$. This completes first part of the proof.
%NOTE HERE, that we deduce from synch of reduced synch of original. ADD this to theorem~\ref{th:2}.

We need more complicated solution in order to get strongly-connected automaton.
%Presented solution do not provide us with strongly-connected automaton. 
If $r = n - 1$ then automaton $\mathrsfs{E}_n$ satisfies the conditions of proposition.
Suppose now that $r \neq n - 1$.
Then the desired solution is the following:
$$
X =
\left(
\begin{array}{c}
M \\
e_1 \\
A_r
\end{array}
\right),
Y = 
\left(
\begin{array}{c}
0_d \mid I_r
\end{array}
\right),
\G = 
\left(
\begin{array}{c}
M \\
e_1 \\
I_r
\end{array}
\right),
\D = 
\left(
\begin{array}{c}
I_{d - 1} \mid 0_{r + 1} \\ \hline
\textbf{\,\,\,} 0_{2d} \textbf{\,\,\,} \mid I_{r - d} \\ \hline
e_d
\end{array}
\right),
$$
where $M =
\left(
\begin{smallmatrix}
e_2\\
e_3\\
\ldots\\
e_d
\end{smallmatrix}\right)$ is matrix of size $d$ by $r$.
The proof of the fact that automaton reducable to this one is strongly-connected is
straightforward. The results can be summorized in the following theorem.


\section{Conclusion and discussion}
Firstly, find canocical reduction. Find classes that have same reduced automaton.

\begin{thebibliography}{99}
\bibitem{AGW}
Adler, R.L., Goodwyn, L.W., Weiss, B.: Equivalence of topological Markov shifts.
Israel J. Math. 27, 49--63 (1977)

\bibitem{AS09}
Almeida, J.; Steinberg, B.: Matrix mortality and the \v{C}ern\'{y}--Pin
conjecture. In:  Diekert, V.; Nowotka, D. (eds.), Developments in
Language Theory, Lect.\ Notes Comput.\ Sci., vol.\,5583, pp. 67--80.
Springer, Heidelberg (2009)

\bibitem{AMR}
Almeida, M.; Moreira, N.; Reis, R.: Enumeration and generation with a string
automata representation. Theor.\ Comput.\ Sci. 387, 93--102 (2007)

\bibitem{AVZ}
Ananichev, D.S., Volkov, M.V., Zaks, Yu.I.: Synchronizing automata
with a letter of deficiency 2. Theor.\ Comput.\ Sci. 376, 30--41 (2007)

\bibitem{Ce64}
\v{C}ern\'{y}, J.: Pozn\'{a}mka k homog\'{e}nnym eksperimentom s
kone\v{c}n\'{y}mi automatami. Matematicko-fyzikalny \v{C}asopis
Slovensk.\ Akad.\ Vied 14(3) 208--216 (1964) (in Slovak)

\bibitem{Du98}
Dubuc, L.: Sur les automates circulaires et la conjecture de
\v{C}ern\'y. RAIRO Inform.\ Th\'eor.\ Appl. 32, 21--34 (1998) (in
French)

\bibitem{DM62}
Dulmage, A.L., Mendelsohn, N.S.: The exponent of a primitive matrix.
Can.\ Math.\ Bull. 5, 241--244 (1962)

\bibitem{DM64}
Dulmage, A.L., Mendelsohn, N.S.: Gaps in the exponent set of primitive
matrices. Ill.\ J. Math. 8, 642--656 (1964)

\bibitem{Ep90}
Eppstein, D.: Reset sequences for monotonic automata. SIAM J.
Comput. 19, 500--510 (1990)

\bibitem{Hi88}
Higgins, P.M.: The range order of a product of $i$ transformations
from a finite full transformation semigroup, Semigroup Forum 37, 31--36
(1988)

\bibitem{Ka03}
Kari, J.: Synchronizing finite automata on Eulerian digraphs.
Theoret.\ Comput.\ Sci. 295, 223--232 (2003)

\bibitem{Pi78}
Pin, J.-E.: Le probl\`eme de la synchronization et la conjecture de
\v{C}ern\'y. Th\`ese de 3\`eme cycle. Universit\'e de Paris 6 (1978) (in
French)

\bibitem{Pi83}
Pin, J.-E.: On two combinatorial problems arising from automata
theory. Ann.\ Discrete Math. 17, 535--548 (1983)

\bibitem{RaAl05}
Ram\'{\i}rez Alfons\'{\i}n, J.L.: The diophantine Frobenius problem.
Oxford University Press (2005)

\bibitem{Sa05}
Sandberg, S.: Homing and synchronizing sequences. In: Broy, M.
et~al. (eds.), Model-Based Testing of Reactive Systems. Lect.\
Notes Comput.\ Sci., vol.\,3472, pp.\,5--33. Springer, Heidelberg
(2005)

\bibitem{SZ}
Skvortsov E., Zaks Yu.: Synchronizing random automata. Submitted;
proceedings version in: Rigo, M. (ed.), AutoMathA 2009, Universit\'e
de Li\`ege (2009)

\bibitem{Tr06}
Trahtman, A.N.: An efficient algorithm finds noticeable trends
and examples concerning the \v{C}ern\'y conjecture. In:
Kr\'alovi\v{c}, R.; Urzyczyn, P. (eds.), Mathematical Foundations
of Computer Science. Lect.\ Notes Comput.\ Sci., vol.\,4162, pp.\,789--800
Springer, Heidelberg (2006)

\bibitem{Tr06a}
Trahtman, A.N.: Notable trends concerning the synchronization
of graphs and automata. Electr.\ Notes Discrete Math. 25, 173--175 (2006)

\bibitem{Tr07}
Trahtman, A.N.: The \v{C}ern\'y conjecture for aperiodic automata.
Discrete Math.\ Theor.\ Comput.\ Sci. 9(2), 3--10 (2007)

\bibitem{Tr08}
Trahtman, A.N.: Some aspects of synchronization of DFA. J. Comput.\ Sci.\
Technol. 23, 719--727 (2008)

\bibitem{Tr09}
Trahtman, A.N.: The Road Coloring Problem. Israel J. Math. 172,
51--60 (2009)

\bibitem{Vo08}
Volkov, M.V.: Synchronizing automata and the \v{C}ern\'{y}
conjecture. In: Mart\'\i{}n-Vide, C.; Otto, F.; Fernau, H. (eds.),
Languages and Automata: Theory and Applications. Lect.\ Notes
Comput.\ Sci., vol.\,5196, pp.\,11--27.  Springer, Heidelberg (2008)

\bibitem{Vo09}
Volkov, M.V.: Synchronizing automata preserving a chain of partial
orders. Theoret.\ Comput.\ Sci. 410, 2992--2998 (2009)

\bibitem{Wi50}
Wielandt, H.: Unzerlegbare, nicht negative Matrizen. Math.\ Z.
52, 642--648 (1950) (in German)
\end{thebibliography}

\end{document}