\documentclass[11pt]{llncs}
\usepackage{amsmath}
\usepackage{amssymb}
\usepackage{epic,gastex}

\newcommand{\sa}{synchronizing automata}
\newcommand{\san}{synchronizing automaton}
\newcommand{\ssw}{a reset word of minimum length}


\DeclareMathOperator{\weight}{wg}

\DeclareSymbolFont{rsfscript}{OMS}{rsfs}{m}{n}
\DeclareSymbolFontAlphabet{\mathrsfs}{rsfscript}

\newcommand{\theoremtext}[1]{
For each $n$, there exists strongly connected \san\ with $n$ states,
such that it's shortest reset word is of length $#1$.
}

\newcommand{\lemmatext}[1]{
The word $#1$ is a reset word for the automaton $\mathrsfs{B}_n$.}


\begin{document}
\title{Slowly synchronizing automata and digraphs\thanks{Supported
by the Russian Foundation for Basic Research, grants 09-01-12142
and 10-01-00524, and by the Federal Education Agency of Russia,
grant 2.1.1/3537.}}

\titlerunning{Slowly synchronizing automata and digraphs}

\author{D. S. Ananichev \and V. V. Gusev \and M. V. Volkov}

\authorrunning{D. S. Ananichev, V. V. Gusev, M. V. Volkov}

\tocauthor{D. S. Ananichev, V. V. Gusev, M. V. Volkov
(Ekaterinburg, Russia)}

\institute{Department of Mathematics and Mechanics,\\
Ural State University, 620083 Ekaterinburg, RUSSIA\\
\email{Dmitry.Ananichev@usu.ru, vl.gusev@gmail.com,
Mikhail.Volkov@usu.ru}}

\maketitle

\begin{abstract}
We present several infinite series of \sa\ for which
the minimum length of reset words is close to the
square of the number of states. These automata are
closely related to primitive  matrices with large exponent.
\end{abstract}


\section{Background and overview}

A \emph{complete deterministic finite automaton} (DFA) is a triple
$\mathrsfs{A}=\langle Q,\Sigma,\delta\rangle$, where $Q$ and
$\Sigma$ are finite sets called the \emph{state set} and the
\emph{input alphabet} respectively, and $\delta:Q\times\Sigma\to
Q$ is a totally defined function called the \emph{transition
function}. Let $\Sigma^*$ stand for the collection of all finite
words over the alphabet $\Sigma$, including the empty word.
The function $\delta$ extends to a function $Q\times\Sigma^*\to Q$
(still denoted by $\delta$) in the following natural way: for every
$q\in Q$ and $w\in\Sigma^*$, we set
$$\delta(q,w)=\begin{cases}
q &\text{ if $w$ is empty},\\
\delta(\delta(q,v),a) &\text{ if $w=va$ for some $v\in\Sigma^*$ and $a\in\Sigma$}.
\end{cases}$$
Thus, via $\delta$, every word $w\in\Sigma^*$ acts on the set $Q$.

A DFA $\mathrsfs{A}=\langle Q,\Sigma,\delta\rangle$ is called
\emph{synchronizing} if the action of some word $w\in\Sigma^*$
resets $\mathrsfs{A}$, that is, leaves the automaton in one
particular state no matter at which state in $Q$ it is applied:
$\delta(q,w)=\delta(q',w)$ for all $q,q'\in Q$. Any such word $w$
is said to be a \emph{reset word} for the DFA.

Synchronizing automata serve as transparent and natural models of
error-resistant systems in many applications (coding theory, robotics,
testing of reactive systems) and also reveal interesting connections
with symbolic dynamics and other parts of mathematics. For a brief
introduction to the theory of \sa\ we refer the reader to the recent
surveys~\cite{Sa05,Vo08}. Here we focus on the so-called \v{C}ern\'{y}
conjecture that constitutes a major open problem in this area.

In~1964 \v{C}ern\'{y}~\cite{Ce64} constructed for each $n>1$ a \san\
$\mathrsfs{C}_n$ with $n$ states whose shortest reset word has length
$(n-1)^2$. Soon after that he conjectured that these automata
represent the worst possible case, that is, every \san\ with $n$ states
can be reset by a word of length $(n-1)^2$. This imply looking conjecture
resists researchers' efforts for more than 40 years. Even though the
conjecture has been confirmed for various restricted classes of \sa\
(cf., e.g., \cite{Ep90,Du98,Ka03,Tr07,AS09,Vo09}), no upper bound of magnitude
$O(n^2)$ for the minimum length of reset words for $n$-state \sa\ is known
in general---the best upper bound achieved so far is $\frac{n^3-n}6$,
see~\cite{Pi83}.

One of the difficulties that one encounters when approaching the
\v{C}ern\'{y} conjecture is that there are only very few examples
of \emph{extreme} \sa, that is, $n$-state \sa\ whose shortest reset
words have length $(n-1)^2$. In fact, the \v{C}ern\'{y} series
$\mathrsfs{C}_n$, $n=2,3,\dotsc$, is the only known infinite series
of extreme \sa. Besides that, we know only a few isolated examples
of such automata, see~\cite{Vo08} for a complete list. Moreover,
even \emph{slowly} \sa, that is, \sa\ whose shortest reset words have
length close to the \v{C}ern\'{y} bound are very rare. This empirical
observation is supported also by probabilistic arguments. For instance,
Higgins~\cite{Hi88} has shown that the probability that a composition of $2n$
random self-maps of a set of size $n$ is a constant map tends to~1 as $n$
goes to infinity. In terms of automata, Higgins's result means that
a random automaton with $n$ states and at least $2n$ input letters
has a reset word of length $2n$. For further results of the same flavor
see~\cite{SZ}. Thus, there is no hope to find new examples of slowly
\sa\ via a random sampling experiment.

We therefore have designed and performed a set of exhaustive search
experiments. Our experiments are briefly described in Section~\ref{experiments}
while the main body of the paper is devoted to a theoretical analysis
of their outcome. We concentrate on two principal issues. In Section~\ref{matrices}
we discuss a remarkable analogy between the distribution of lengths of shortest
reset words for \sa\ and the distribution of exponents of primitive matrices.
Section~\ref{sss} collects several new series of slowly \sa. The ``initial''
examples in the series were found in the course of the experiments; then
each example was expanded to a series of automata that have been proved
to be slowly synchronizing. In our opinion, the proof technique is also
of interest; in fact, we provide a transparent and uniform approach to all
sufficiently large slowly \sa\ with 2~input letters, both new and already
known ones.

\section{Preliminaries on automata, digraphs and matrices}
\label{preliminaries}

A \emph{directed graph} (digraph) is a pair $D=\langle V,E\rangle$ where $V$ is
a finite set and $E\subseteq V\times V$. We refer to elements of $V$ and $E$ as
\emph{vertices} and \emph{edges}. Observe that our definition allows loops but
excludes multiple edges. If $v,v'\in V$ and $e=(v,v')\in E$, the edge $e$ is said
to be \emph{outgoing} for $v$.

Given a DFA $\mathrsfs{A}=\langle Q,\Sigma,\delta\rangle$, its \emph{underlying
digraph} $D(\mathrsfs{A})$ has $Q$ as the vertex set and $(q,q')\in Q\times Q$ is
an edge of $D(\mathrsfs{A})$ if and only if $q'=\delta(q,a)$ for some $a\in\Sigma$.
It is easy to see that a digraph $D$ is isomorphic to the underlying digraph of some
DFA if and only if each vertex of $D$ has at least one outgoing edge. In the sequel,
we always consider only digraphs satisfying this property. Every DFA $\mathrsfs{A}$
such that $D\cong D(\mathrsfs{A})$ is called a \emph{coloring} of $D$. Thus, every
coloring of $D$ is defined by assigning non-empty sets of labels (colors) from some
alphabet $\Sigma$ to edges of $D$ such that the label sets assigned to the outgoing
edges of each vertex form a partition of $\Sigma$. Fig.\,\ref{fig:cerny} shows a
digraph and two of its colorings by $\Sigma=\{a,b\}$.

\begin{figure}[ht]
 \begin{center}
  \unitlength=3pt
    \begin{picture}(18,26)(-60,-4)
    \gasset{Nw=6,Nh=6,Nmr=3}
    \node(A1)(0,18){$1$}
    \node(A2)(18,18){$2$}
    \node(A3)(18,0){$3$}
    \node(A4)(0,0){$4$}
    \drawloop[loopangle=135](A1){$a$}
    \drawloop[loopangle=45](A2){$b$}
    \drawloop[loopangle=-45](A3){$b$}
    \drawedge(A1,A2){$b$}
    \drawedge(A2,A3){$a$}
    \drawedge(A3,A4){$a$}
    \drawedge(A4,A1){$a,b$}
    \end{picture}
 \begin{picture}(18,26)(0,-4)
    \gasset{Nw=6,Nh=6,Nmr=3}
    \node(A1)(0,18){$1$}
    \node(A2)(18,18){$2$}
    \node(A3)(18,0){$3$}
    \node(A4)(0,0){$4$}
    \drawloop[loopangle=135](A1){$b$}
    \drawloop[loopangle=45](A2){$b$}
    \drawloop[loopangle=-45](A3){$b$}
    \drawedge(A1,A2){$a$}
    \drawedge(A2,A3){$a$}
    \drawedge(A3,A4){$a$}
    \drawedge(A4,A1){$a,b$}
    \end{picture}
 \begin{picture}(18,26)(60,-4)
    \gasset{Nw=6,Nh=6,Nmr=3}
    \node(A1)(0,18){$1$}
    \node(A2)(18,18){$2$}
    \node(A3)(18,0){$3$}
    \node(A4)(0,0){$4$}
    \drawloop[loopangle=135](A1){}
    \drawloop[loopangle=45](A2){}
    \drawloop[loopangle=-45](A3){}
    \drawedge(A1,A2){}
    \drawedge(A2,A3){}
    \drawedge(A3,A4){}
    \drawedge(A4,A1){}
    \end{picture}
 \end{center}
 \caption{A digraph and two of its colorings}
 \label{fig:cerny}
\end{figure}


\section{Correlation between digraphs and automata}
\label{matrices}

\begin{theorem}\label{dulmage}
Apart from isomorphism, there is exactly one primitive graph $D$
on $n$ vertices for which $\gamma(D) = (n - 1)^2 + 1$, and exactly one
for which $\gamma(D) = (n - 1)^2$.
\end{theorem}

8 states:\\

\begin{tabular}{|c|c|c|c|c|c|c|c|c|c|}
49 & 48 & 47 & 46 & 45 & 44 & 43 & 42 & 41 & 40 \\
\hline
1  & 0  & 0  & 0  & 0  & 1  & 1  & 3  & 1  & 5  \\
\end{tabular}
\vspace{0.7cm}

9 states:\\
\begin{tabular}{|c|c|c|c|c|c|c|c|c|c|c|c|c|c|}
64 & 63 & 62 & 61 & 60 & 59 & 58 & 57 & 56 & 55 & 54 & 53 & 52 & 51 \\
\hline
1  & 0  & 0  & 0  & 0  & 0  & 1  & 2  & 3  & 0  & 0  & 0  & 4  & 4   \\
\end{tabular}


\section{Main results and slowly sinchronizing series}
\label{sss}

\begin{theorem}
For each $n$, there exists strongly connected \san\ with $n$ states,
such that it's shortest reset word is of length $n^2-3n+3$.
\end{theorem}

\begin{lemma}
\label{lemma:anan}
The word $(ab^{n - 2})^{n - 2}a$ is a reset word for the automaton $\mathrsfs{B}_n$.
\end{lemma}

\begin{figure}[ht]
\begin{center}
\unitlength .7mm
\begin{picture}(112,76)(0,-86)
\gasset{Nadjustdist=1.5} \node(n0)(56.0,-16.0){1}
\node(n1)(24.0,-40.0){0} \node(n2)(88.0,-40.0){2}
\node(n3)(36.0,-72.0){4} \node(n4)(76.0,-72.0){3}
\drawedge[ELdist=2.0](n1,n0){$b$} \drawedge[ELdist=1.5](n2,n4){$a, b$}
\drawedge[ELdist=1.7](n0,n2){$a, b$} \drawedge[ELdist=2.0](n4,n3){$a, b$}
\drawedge[ELdist=1.7](n3,n1){$a, b$} \drawedge[ELdist=2.0](n1,n2){$a$}
\end{picture}
\end{center}
\caption{The automaton $\mathrsfs{B}_5$}\label{fig:anan}
\end{figure}

The states of the automaton $\mathrsfs{B}_n$
are the residues modulo $n$ and its input letters $a$ and $b$ act
as follows:
$$
 \delta(m,a)=
 \begin{cases}
  m + 2\!\!\pmod{n} & \text{for $m = 0$}, \\
  m+1\!\!\pmod{n} & \text{for $m \neq 0$};
  \end{cases}
\qquad \delta(m,b)=m+1\!\!\pmod{n}.
$$
The smallest automaton in the series is shown in Fig.~\ref{fig:anan}.

\begin{proof}
Let $w$ be \ssw . It is clear that action of $w$ leaves automaton $\mathrsfs{B}_n$ in state 2.
For any $u \in \Sigma^*$ word $uw$ is synchronizing too and it also brings automaton to the state 2.
In other words, there is a path in underlying
digraph from any vertex to vertex 2 of length $p$, if $p \geq |w|$.
Let us consider paths from 2 to 2. Every one of them is the combination of cycles of length $n$ and $n - 1$.
Thus for any $p \geq |w|$ the following is true: $p = xn + y(n - 1)$, where $x$ and $y$ are nonnegative integers.
It is well known(See smth) that $n(n - 1) - n - (n - 1) = n^2 - 3n + 1$ is the greatest number that could not be represented
as a nonegative integer combination of $n$ and $n - 1$. Thus $|w| \geq n^2 - 3n + 2$.
Suppose that $|w| = n^2 - 3n + 2$.
%and  then there is a path of such length from state $1$ to state $2$.
If we read this word from state $1$ then we get a path of length $n^2 - 3n + 1$ from $2$ to $2$ which is not possible due to previous
statement. Thus $|w| \geq n^2 - 3n + 3$ but this bound is tight by lemma~\ref{lemma:anan}.
\end{proof}
\newpage

\begin{theorem}\label{theo}
\theoremtext{n^2-3n+3}
\end{theorem}

\begin{lemma}
\lemmatext{(ab^{n - 2})^{n - 2}a}
\end{lemma}

\begin{figure}[ht]
\begin{center}
\unitlength .7mm
\begin{picture}(112,76)(0,-86)
\gasset{Nadjustdist=1.5} \node(n0)(56.0,-16.0){0}
\node(n1)(24.0,-40.0){4} \node(n2)(88.0,-40.0){1}
\node(n3)(36.0,-72.0){3} \node(n4)(76.0,-72.0){2}
\drawloop[ELdist=1.5,loopangle=33.34](n2){$a$}
\drawloop[ELdist=2.4,loopangle=320.0](n4){$a$}
\drawedge[ELdist=2.0](n1,n0){$b$} \drawedge[ELdist=1.5](n2,n4){$b$}
\drawedge[ELdist=1.7](n0,n2){$a, b$} \drawedge[ELdist=2.0](n4,n3){$b$}
\drawedge[ELdist=1.7](n3,n1){$b$}
%\drawedge[ELdist=2.0](n1,n2){$a$}
%\drawedge[ELdist=2.0](n0,n4){$a$}
\drawloop[ELdist=1.5,loopangle=144.55](n1){$a$}
\drawloop[ELdist=1.5,loopangle=226.55](n3){$a$}
\end{picture}
\end{center}
\caption{The automaton $\mathrsfs{B}_5$}\label{B5}
\end{figure}

The states of the automaton $\mathrsfs{B}_n$
are the residues modulo $n$ and its input letters $a$ and $b$ act
as follows:
$$
 \delta(m,a)=
 \begin{cases}
  m + 2 \!\!\pmod{n} & \text{for $m = 0$}, \\
  m \!\!\pmod{n} & \text{for $m \neq 0$};
  \end{cases}
\qquad \delta(m,b)=m+1\!\!\pmod{n}.
$$
The smallest automaton in the series is shown in Fig.~\ref{B5}.


\newpage







\begin{lemma}
\lemmatext{(ab^{n - 2})^{n - 2}ba}
\end{lemma}

\begin{theorem}\label{theo}
\theoremtext{n^2-3n+4}
\end{theorem}


\begin{figure}[ht]
\begin{center}
\unitlength .7mm
\begin{picture}(112,76)(0,-86)
\gasset{Nadjustdist=1.5} \node(n0)(56.0,-16.0){1}
\node(n1)(24.0,-40.0){0} \node(n2)(88.0,-40.0){2}
\node(n3)(36.0,-72.0){4} \node(n4)(76.0,-72.0){3}
\drawedge[ELdist=2.0](n1,n0){$b$} \drawedge[ELdist=1.5](n2,n4){$a, b$}
\drawedge[ELdist=1.7](n0,n2){$b$} \drawedge[ELdist=2.0](n4,n3){$a, b$}
\drawedge[ELdist=1.7](n3,n1){$a, b$} \drawedge[ELpos=40, ELdist=2.0](n1,n2){$a$}
\drawedge[ELpos=60,ELdist=2.0](n0,n4){$a$}
\end{picture}
\end{center}
\caption{The automaton $\mathrsfs{B}_5$}\label{B5}
\end{figure}

The states of the automaton $\mathrsfs{B}_n$
are the residues modulo $n$ and its input letters $a$ and $b$ act
as follows:
$$
 \delta(m,a)=
 \begin{cases}
  m + 2\!\!\pmod{n} & \text{for $m = 0,1$}, \\
  m+1\!\!\pmod{n} & \text{for $1< m<n$};
  \end{cases}
\qquad \delta(m,b)=m+1\!\!\pmod{n}.
$$
The smallest automaton in the series is shown in Fig.~\ref{B5}.

\newpage

\begin{theorem}
\theoremtext{n^2-3n+2}
\end{theorem}

\begin{lemma}
\lemmatext{(ba^{n - 1})^{n - 3}ba}
\end{lemma}

\begin{figure}[ht]
\begin{center}
\unitlength .7mm
\begin{picture}(112,76)(0,-86)
\gasset{Nadjustdist=1.5} \node(n0)(56.0,-16.0){1}
\node(n1)(24.0,-40.0){0} \node(n2)(88.0,-40.0){2}
\node(n3)(36.0,-72.0){4} \node(n4)(76.0,-72.0){3}
\drawedge[ELdist=2.0](n1,n0){$b$} \drawedge[ELdist=1.5](n2,n4){$a, b$}
\drawedge[ELdist=1.7](n0,n2){$a$} \drawedge[ELdist=2.0](n4,n3){$a, b$}
\drawedge[ELdist=1.7](n3,n1){$a, b$} \drawedge[ELpos=40, ELdist=2.0](n1,n2){$a$}
\drawedge[ELpos=60,ELdist=2.0](n0,n4){$b$}
\end{picture}
\end{center}
\caption{The automaton $\mathrsfs{B}_5$}\label{B5}
\end{figure}

The states of the automaton $\mathrsfs{B}_n$
are the residues modulo $n$ and its input letters $a$ and $b$ act
as follows:
$$
 \delta(m,a)=
 \begin{cases}
  m+2\!\!\pmod{n} & \text{for $m = 1$}, \\
  m+1\!\!\pmod{n} & \text{for $m \neq 0$};
  \end{cases}
\qquad \delta(m,b)=
  \begin{cases}
  m + 2\!\!\pmod{n} & \text{for $m = 1$}, \\
  m+1\!\!\pmod{n} & \text{for $m \neq 1$};
  \end{cases}.
$$
The smallest automaton in the series is shown in Fig.~\ref{B5}.

\newpage


\begin{theorem}\label{theo}
\theoremtext{n^2-3n+3}
\end{theorem}

\begin{lemma}
\lemmatext{(ab^{n - 2})^{n - 2}a}
\end{lemma}

\begin{figure}[ht]
\begin{center}
\unitlength .7mm
\begin{picture}(112,76)(0,-86)
\gasset{Nadjustdist=1.5} \node(n0)(56.0,-16.0){0}
\node(n1)(24.0,-40.0){4} \node(n2)(88.0,-40.0){1}
\node(n3)(36.0,-72.0){3} \node(n4)(76.0,-72.0){2}
\drawloop[ELdist=1.5,loopangle=33.34](n2){$a$}
\drawloop[ELdist=2.4,loopangle=320.0](n4){$a$}
\drawedge[ELdist=2.0](n1,n0){$b$} \drawedge[ELdist=1.5](n2,n4){$b$}
\drawedge[ELdist=1.7](n0,n2){$b$} \drawedge[ELdist=2.0](n4,n3){$b$}
\drawedge[ELdist=1.7](n3,n1){$b$}
%\drawedge[ELdist=2.0](n1,n2){$a$}
\drawedge[ELdist=2.0](n0,n4){$a$}
\drawloop[ELdist=1.5,loopangle=144.55](n1){$a$}
\drawloop[ELdist=1.5,loopangle=226.55](n3){$a$}
\end{picture}
\end{center}
\caption{The automaton $\mathrsfs{B}_5$}\label{B5}
\end{figure}

The states of the automaton $\mathrsfs{B}_n$
are the residues modulo $n$ and its input letters $a$ and $b$ act
as follows:
$$
 \delta(m,a)=
 \begin{cases}
  m + 2 \!\!\pmod{n} & \text{for $m = 0$}, \\
  m \!\!\pmod{n} & \text{for $m \neq 0$};
  \end{cases}
\qquad \delta(m,b)=m+1\!\!\pmod{n}.
$$
The smallest automaton in the series is shown in Fig.~\ref{B5}.

\newpage


\begin{theorem}\label{theo}
\theoremtext{n^2-3n+2}
\end{theorem}

\begin{lemma}
\lemmatext{(aab^{n - 2})^{n - 3}aa}
\end{lemma}

\begin{figure}[ht]
\begin{center}
\unitlength .7mm
\begin{picture}(112,76)(0,-86)
\gasset{Nadjustdist=1.5} \node(n0)(56.0,-16.0){1}
\node(n1)(24.0,-40.0){0} \node(n2)(88.0,-40.0){2}
\node(n3)(36.0,-72.0){4} \node(n4)(76.0,-72.0){3}
\drawloop[ELdist=1.5,loopangle=33.34](n2){$a$}
\drawloop[ELdist=2.4,loopangle=320.0](n4){$a$}
\drawedge[ELdist=2.0](n1,n0){$a$} \drawedge[ELdist=1.5](n2,n4){$b$}
\drawedge[ELdist=1.7](n0,n2){$a, b$} \drawedge[ELdist=2.0](n4,n3){$b$}
\drawedge[ELdist=1.7](n3,n1){$b$}
\drawedge[ELdist=2.0](n1,n2){$b$}
%\drawedge[ELdist=2.0](n0,n4){$a$}
%\drawloop[ELdist=1.5,loopangle=144.55](n1){$a$}
\drawloop[ELdist=1.5,loopangle=226.55](n3){$a$}
\end{picture}
\end{center}
\caption{The automaton $\mathrsfs{B}_5$}\label{B5}
\end{figure}

The states of the automaton $\mathrsfs{B}_n$
are the residues modulo $n$ and its input letters $a$ and $b$ act
as follows:
$$
 \delta(m,a)=
 \begin{cases}
  m + 1 \!\!\pmod{n} & \text{for $m = 0, 1$}, \\
  m \!\!\pmod{n} & \text{for $1 < m < n$};
  \end{cases}
\qquad
 \delta(m,b)=\begin{cases}
  m + 2 \!\!\pmod{n} & \text{for $m = 0$}, \\
  m + 1 \!\!\pmod{n} & \text{for $m \neq 0$};
  \end{cases}.
$$
The smallest automaton in the series is shown in Fig.~\ref{B5}.

\newpage

\begin{theorem}\label{theo}
\theoremtext{n^2-4n+6}
\end{theorem}

\begin{lemma}
\lemmatext{b(ab^{n - 2})^{n - 3}ab}
\end{lemma}


\begin{figure}[th]
\unitlength=.7mm
\begin{center}
\begin{picture}(100,70)(0,-80)
\gasset{Nadjustdist=1.5} \node(n0)(56.0,-16.0){0}
\node(n1)(24.0,-40.0){4} \node(n2)(88.0,-40.0){1}
\node(n3)(36.0,-72.0){3} \node(n4)(76.0,-72.0){2}
\drawloop[ELdist=2.4,loopangle=320.0](n4){$a$}
\drawedge[ELdist=2.0](n1,n0){$b$} \drawedge[ELdist=1.5](n2,n4){$b$}
\drawedge[ELdist=1.7](n0,n2){$b$} \drawedge[ELdist=2.0](n4,n3){$b$}
\drawedge[ELdist=1.7](n3,n1){$b$}
\node[NLangle=0.0](n16)(56.0,-48.0){5}
%\drawedge[ELdist=1.6](n1,n16){$a$}
\drawedge[ELdist=1.7](n16,n4){$b$}

\drawedge[curvedepth=2](n0,n16){$a$}
\drawedge[curvedepth=2](n16,n0){$a$}

%\drawloop[ELdist=1.5,loopangle=90](n0){$a$}
%\drawloop[ELdist=1.5](n16){$b$}
%\drawedge[ELdist=1.5,curvedepth=33](n3,n0){$a$}
\drawloop[ELdist=1.5,loopangle=144.55](n1){$a$}
\drawloop[ELdist=1.5,loopangle=226.55](n3){$a$}
\drawloop[ELdist=1.5,loopangle=33.34](n2){$a$}
\end{picture}
\end{center}
\caption{The automaton $\mathrsfs{B}^{\star}_6$}\label{B-star-6}
\end{figure}

\newpage

\begin{theorem}\label{theo}
\theoremtext{n^2-4n+6}
\end{theorem}

\begin{lemma}
\lemmatext{b(ab^{n - 2})^{n - 3}ba}
\end{lemma}


\begin{figure}[th]
\unitlength=.7mm
\begin{center}
\begin{picture}(100,70)(0,-80)
\gasset{Nadjustdist=1.5} \node(n0)(56.0,-16.0){0}
\node(n1)(24.0,-40.0){4} \node(n2)(88.0,-40.0){1}
\node(n3)(36.0,-72.0){3} \node(n4)(76.0,-72.0){2}
\drawloop[ELdist=2.4,loopangle=320.0](n4){$a$}
\drawedge[ELdist=2.0](n1,n0){$b$} \drawedge[ELdist=1.5](n2,n4){$a, b$}
\drawedge[ELdist=1.7](n0,n2){$b$} \drawedge[ELdist=2.0](n4,n3){$b$}
\drawedge[ELdist=1.7](n3,n1){$b$}
\node[NLangle=0.0](n16)(56.0,-48.0){5}
\drawedge[ELdist=1.7](n16,n4){$b$}

\drawedge[curvedepth=2](n0,n16){$a$}
\drawedge[curvedepth=2](n16,n0){$a$}

\drawloop[ELdist=1.5,loopangle=144.55](n1){$a$}
\drawloop[ELdist=1.5,loopangle=226.55](n3){$a$}
\end{picture}
\end{center}
\caption{The automaton $\mathrsfs{B}^{\star}_6$}\label{B-star-6}
\end{figure}

\section{Experiments}
\label{experiments}

\begin{thebibliography}{99}
\bibitem{AS09}
Almeida, J.; Steinberg, B.: Matrix mortality and the \v{C}ern\'{y}--Pin
conjecture. In:  Diekert, V.; Nowotka, D. (eds.), Developments in
Language Theory, Lect.\ Notes Comput.\ Sci., vol.\,5583, pp. 67--80.
Springer, Heidelberg (2009)

\bibitem{AVZ}
Ananichev, D.S., Volkov, M.V., Zaks, Yu.I.: Synchronizing automata
with a letter of deficiency 2. Theor.\ Comput.\ Sci. 376, 30--41 (2007)

\bibitem{Ce64}
\v{C}ern\'{y}, J.: Pozn\'{a}mka k homog\'{e}nnym eksperimentom s
kone\v{c}n\'{y}mi automatami. Matematicko-fyzikalny \v{C}asopis
Slovensk.\ Akad.\ Vied 14(3) 208--216 (1964) (in Slovak)

\bibitem{Du98}
Dubuc, L.: Sur les automates circulaires et la conjecture de
\v{C}ern\'y. RAIRO Inform.\ Th\'eor.\ Appl. 32, 21--34 (1998) (in
French)

\bibitem{Ep90}
Eppstein, D.: Reset sequences for monotonic automata. SIAM J.
Comput. 19, 500--510 (1990)

\bibitem{Hi88}
Higgins, P.M.: The range order of a product of $i$ transformations
from a finite full transformation semigroup, Semigroup Forum 37, 31--36
(1988)

\bibitem{Ka03}
Kari, J.: Synchronizing finite automata on Eulerian digraphs.
Theoret.\ Comput.\ Sci. 295, 223--232 (2003)

\bibitem{Pi83}
Pin, J.-E.: On two combinatorial problems arising from automata
theory. Ann.\ Discrete Math. 17, 535--548 (1983)

\bibitem{Sa03}
A.\,Salomaa, Composition sequences for functions over a finite
domain, Theoret.\ Comput.\ Sci. 292 (2003) 263--281.

\bibitem{Sa05}
Sandberg, S.: Homing and synchronizing sequences. In: Broy, M.
et~al. (eds.), Model-Based Testing of Reactive Systems. Lect.\
Notes Comput.\ Sci., vol.\,3472, pp.\,5--33. Springer, Heidelberg
(2005)

\bibitem{SZ}
Skvortsov E., Zaks Yu.: Synchronizing random automata. Submitted;
proceedings version in: Rigo, M. (ed.), AutoMathA 2009, Universit\'e
de Li\`ege (2009)

\bibitem{Tr06}
Trahtman, A.: An efficient algorithm finds noticeable trends
and examples concerning the \v{C}ern\'y conjecture. In:
Kr\'alovi\v{c}, R.; Urzyczyn, P. (eds.), Mathematical Foundations
of Computer Science. Lect.\ Notes Comput.\ Sci., vol.\,4162, pp.\,789--800
Springer, Heidelberg (2006)

\bibitem{Tr07}
Trahtman, A.: The \v{C}ern\'y conjecture for aperiodic automata.
Discrete Math.\ Theor.\ Comput.\ Sci. 9(2), 3--10 (2007)

\bibitem{Vo08}
Volkov, M.V.: Synchronizing automata and the \v{C}ern\'{y}
conjecture. In: Mart\'\i{}n-Vide, C.; Otto, F.; Fernau, H. (eds.),
Languages and Automata: Theory and Applications. Lect.\ Notes
Comput.\ Sci., vol.\,5196, pp.\,11--27.  Springer, Heidelberg (2008)

\bibitem{Vo09}
Volkov, M.V.: Synchronizing automata preserving a chain of partial
orders. Theoret.\ Comput.\ Sci. 410, 2992--2998 (2009)
\end{thebibliography}


\section*{Appendix}

\end{document}
