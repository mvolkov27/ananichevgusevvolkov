\documentclass[11pt]{llncs}
\usepackage{amsmath}
\usepackage{amssymb}
\usepackage{epic,gastex}

\newcommand{\sa}{synchronizing automata}
\newcommand{\san}{synchronizing automaton}
\newcommand{\ssw}{the shortest synchronizing word}


\DeclareMathOperator{\weight}{wg}

\DeclareSymbolFont{rsfscript}{OMS}{rsfs}{m}{n}
\DeclareSymbolFontAlphabet{\mathrsfs}{rsfscript}

\newcommand{\theoremtext}[1]{
For each $n$, there exists strongly connected \san\ with $n$ states,
such that it's shortest reset word is of length $#1$.
}

\newcommand{\lemmatext}[1]{
The word $#1$ is a reset word for the automaton $\mathrsfs{B}_n$.}


\begin{document}
\title{Slowly synchronizing automata and digraphs\thanks{Supported
by the Russian Foundation for Basic Research, grants 09-01-12142
and 10-01-00524, and by the Federal Education Agency of Russia,
grant 2.1.1/3537.}}

\titlerunning{Slowly synchronizing automata and digraphs}

\author{D. S. Ananichev \and V. V. Gusev \and M. V. Volkov}

\authorrunning{D. S. Ananichev, V. V. Gusev, M. V. Volkov}

\tocauthor{D. S. Ananichev, V. V. Gusev, M. V. Volkov
(Ekaterinburg, Russia)}

\institute{Department of Mathematics and Mechanics,\\
Ural State University, 620083 Ekaterinburg, RUSSIA\\
\email{Dmitry.Ananichev@usu.ru, vl.gusev@gmail.com,
Mikhail.Volkov@usu.ru}}

\maketitle

\begin{abstract}
We present several infinite series of \sa\
whose shortest reset words are close to conjectured by \v{C}ern\'{y} upper bound $(n - 1)^2$.
%Connection with primitive digraphs with big exponent.
\end{abstract}


\section*{Background and motivation}

\newpage
\section{Correlation between digraphs and automata}

\begin{theorem}\label{dulmage}
Apart from isomorphism, there is exactly one primitive graph $D$
on $n$ vertices for which $\gamma(D) = (n - 1)^2 + 1$, and exactly one
for which $\gamma(D) = (n - 1)^2$.
\end{theorem}

8 states:\\

\begin{tabular}{|c|c|c|c|c|c|c|c|c|c|}
49 & 48 & 47 & 46 & 45 & 44 & 43 & 42 & 41 & 40 \\
\hline
1  & 0  & 0  & 0  & 0  & 1  & 1  & 3  & 1  & 5  \\
\end{tabular}
\vspace{0.7cm}

9 states:\\
\begin{tabular}{|c|c|c|c|c|c|c|c|c|c|c|c|c|c|}
64 & 63 & 62 & 61 & 60 & 59 & 58 & 57 & 56 & 55 & 54 & 53 & 52 & 51 \\
\hline
1  & 0  & 0  & 0  & 0  & 0  & 1  & 2  & 3  & 0  & 0  & 0  & 4  & 4   \\
\end{tabular}


\section{Main results and slowly sinchronizing series}

\begin{theorem}
\theoremtext{n^2-3n+3}
\end{theorem}

\begin{lemma}
\lemmatext{(ab^{n - 2})^{n - 2}a}
\end{lemma}

\begin{figure}[ht]
\begin{center}
\unitlength .7mm
\begin{picture}(112,76)(0,-86)
\gasset{Nadjustdist=1.5} \node(n0)(56.0,-16.0){1}
\node(n1)(24.0,-40.0){0} \node(n2)(88.0,-40.0){2}
\node(n3)(36.0,-72.0){4} \node(n4)(76.0,-72.0){3}
\drawedge[ELdist=2.0](n1,n0){$b$} \drawedge[ELdist=1.5](n2,n4){$a, b$}
\drawedge[ELdist=1.7](n0,n2){$a, b$} \drawedge[ELdist=2.0](n4,n3){$a, b$}
\drawedge[ELdist=1.7](n3,n1){$a, b$} \drawedge[ELdist=2.0](n1,n2){$a$}
\end{picture}
\end{center}
\caption{The automaton $\mathrsfs{B}_5$}\label{Anan}
\end{figure}

The states of the automaton $\mathrsfs{B}_n$
are the residues modulo $n$ and its input letters $a$ and $b$ act
as follows:
$$
 \delta(m,a)=
 \begin{cases}
  m + 2\!\!\pmod{n} & \text{for $m = 0$}, \\
  m+1\!\!\pmod{n} & \text{for $m \neq 0$};
  \end{cases}
\qquad \delta(m,b)=m+1\!\!\pmod{n}.
$$
The smallest automaton in the series is shown in Fig.~\ref{Anan}.

{\bf proof}:
Let $w$ be \ssw . It is clear that action of $w$ leaves automaton $\mathrsfs{B}_n$ in state 2.
For any $u \in \Sigma^*$ word $uw$ is synchronizing too and it also brings automaton to the state 2.
In other words, there is a path in underlying
digraph from any vertex to vertex 2 of length $p$, if $p \geq |w|$.
Let us consider paths from 2 to 2. Every one of them is the combination of cycles of length $n$ and $n - 1$.
Thus for any $p \ geq |w|$ the following is true: $p = xn + y(n - 1)$, where $x$ and $y$ are nonnegative integers.
It is well known(See smth) that $n(n - 1) - n - (n - 1) = n^2 - 3n + 1$ is the greatest number that could not be represented
as a nonegative integer combination of $n$ and $n - 1$. Thus $|w| \geq n^2 - 3n + 2$.
Suppose that $|w| = n^2 - 3n + 2$ then there is a path of such length from state $1$ to state $2$. Applying
it will immediately follow


\newpage











\begin{lemma}
\lemmatext{(ab^{n - 2})^{n - 2}ba}
\end{lemma}

\begin{theorem}\label{theo}
\theoremtext{n^2-3n+4}
\end{theorem}


\begin{figure}[ht]
\begin{center}
\unitlength .7mm
\begin{picture}(112,76)(0,-86)
\gasset{Nadjustdist=1.5} \node(n0)(56.0,-16.0){1}
\node(n1)(24.0,-40.0){0} \node(n2)(88.0,-40.0){2}
\node(n3)(36.0,-72.0){4} \node(n4)(76.0,-72.0){3}
\drawedge[ELdist=2.0](n1,n0){$b$} \drawedge[ELdist=1.5](n2,n4){$a, b$}
\drawedge[ELdist=1.7](n0,n2){$b$} \drawedge[ELdist=2.0](n4,n3){$a, b$}
\drawedge[ELdist=1.7](n3,n1){$a, b$} \drawedge[ELpos=40, ELdist=2.0](n1,n2){$a$}
\drawedge[ELpos=60,ELdist=2.0](n0,n4){$a$}
\end{picture}
\end{center}
\caption{The automaton $\mathrsfs{B}_5$}\label{B5}
\end{figure}

The states of the automaton $\mathrsfs{B}_n$
are the residues modulo $n$ and its input letters $a$ and $b$ act
as follows:
$$
 \delta(m,a)=
 \begin{cases}
  m + 2\!\!\pmod{n} & \text{for $m = 0,1$}, \\
  m+1\!\!\pmod{n} & \text{for $1< m<n$};
  \end{cases}
\qquad \delta(m,b)=m+1\!\!\pmod{n}.
$$
The smallest automaton in the series is shown in Fig.~\ref{B5}.

\newpage

\begin{theorem}
\theoremtext{n^2-3n+2}
\end{theorem}

\begin{lemma}
\lemmatext{(ba^{n - 1})^{n - 3}ba}
\end{lemma}

\begin{figure}[ht]
\begin{center}
\unitlength .7mm
\begin{picture}(112,76)(0,-86)
\gasset{Nadjustdist=1.5} \node(n0)(56.0,-16.0){1}
\node(n1)(24.0,-40.0){0} \node(n2)(88.0,-40.0){2}
\node(n3)(36.0,-72.0){4} \node(n4)(76.0,-72.0){3}
\drawedge[ELdist=2.0](n1,n0){$b$} \drawedge[ELdist=1.5](n2,n4){$a, b$}
\drawedge[ELdist=1.7](n0,n2){$a$} \drawedge[ELdist=2.0](n4,n3){$a, b$}
\drawedge[ELdist=1.7](n3,n1){$a, b$} \drawedge[ELpos=40, ELdist=2.0](n1,n2){$a$}
\drawedge[ELpos=60,ELdist=2.0](n0,n4){$b$}
\end{picture}
\end{center}
\caption{The automaton $\mathrsfs{B}_5$}\label{B5}
\end{figure}

The states of the automaton $\mathrsfs{B}_n$
are the residues modulo $n$ and its input letters $a$ and $b$ act
as follows:
$$
 \delta(m,a)=
 \begin{cases}
  m+2\!\!\pmod{n} & \text{for $m = 1$}, \\
  m+1\!\!\pmod{n} & \text{for $m \neq 0$};
  \end{cases}
\qquad \delta(m,b)=
  \begin{cases}
  m + 2\!\!\pmod{n} & \text{for $m = 1$}, \\
  m+1\!\!\pmod{n} & \text{for $m \neq 1$};
  \end{cases}.
$$
The smallest automaton in the series is shown in Fig.~\ref{B5}.

\newpage


\begin{theorem}\label{theo}
\theoremtext{n^2-3n+3}
\end{theorem}

\begin{lemma}
\lemmatext{(ab^{n - 2})^{n - 2}a}
\end{lemma}

\begin{figure}[ht]
\begin{center}
\unitlength .7mm
\begin{picture}(112,76)(0,-86)
\gasset{Nadjustdist=1.5} \node(n0)(56.0,-16.0){0}
\node(n1)(24.0,-40.0){4} \node(n2)(88.0,-40.0){1}
\node(n3)(36.0,-72.0){3} \node(n4)(76.0,-72.0){2}
\drawloop[ELdist=1.5,loopangle=33.34](n2){$a$}
\drawloop[ELdist=2.4,loopangle=320.0](n4){$a$}
\drawedge[ELdist=2.0](n1,n0){$b$} \drawedge[ELdist=1.5](n2,n4){$b$}
\drawedge[ELdist=1.7](n0,n2){$b$} \drawedge[ELdist=2.0](n4,n3){$b$}
\drawedge[ELdist=1.7](n3,n1){$b$}
%\drawedge[ELdist=2.0](n1,n2){$a$}
\drawedge[ELdist=2.0](n0,n4){$a$}
\drawloop[ELdist=1.5,loopangle=144.55](n1){$a$}
\drawloop[ELdist=1.5,loopangle=226.55](n3){$a$}
\end{picture}
\end{center}
\caption{The automaton $\mathrsfs{B}_5$}\label{B5}
\end{figure}

The states of the automaton $\mathrsfs{B}_n$
are the residues modulo $n$ and its input letters $a$ and $b$ act
as follows:
$$
 \delta(m,a)=
 \begin{cases}
  m + 2 \!\!\pmod{n} & \text{for $m = 0$}, \\
  m \!\!\pmod{n} & \text{for $m \neq 0$};
  \end{cases}
\qquad \delta(m,b)=m+1\!\!\pmod{n}.
$$
The smallest automaton in the series is shown in Fig.~\ref{B5}.

\newpage


\begin{theorem}\label{theo}
\theoremtext{n^2-3n+2}
\end{theorem}

\begin{lemma}
\lemmatext{(aab^{n - 2})^{n - 3}aa}
\end{lemma}

\begin{figure}[ht]
\begin{center}
\unitlength .7mm
\begin{picture}(112,76)(0,-86)
\gasset{Nadjustdist=1.5} \node(n0)(56.0,-16.0){1}
\node(n1)(24.0,-40.0){0} \node(n2)(88.0,-40.0){2}
\node(n3)(36.0,-72.0){4} \node(n4)(76.0,-72.0){3}
\drawloop[ELdist=1.5,loopangle=33.34](n2){$a$}
\drawloop[ELdist=2.4,loopangle=320.0](n4){$a$}
\drawedge[ELdist=2.0](n1,n0){$a$} \drawedge[ELdist=1.5](n2,n4){$b$}
\drawedge[ELdist=1.7](n0,n2){$a, b$} \drawedge[ELdist=2.0](n4,n3){$b$}
\drawedge[ELdist=1.7](n3,n1){$b$}
\drawedge[ELdist=2.0](n1,n2){$b$}
%\drawedge[ELdist=2.0](n0,n4){$a$}
%\drawloop[ELdist=1.5,loopangle=144.55](n1){$a$}
\drawloop[ELdist=1.5,loopangle=226.55](n3){$a$}
\end{picture}
\end{center}
\caption{The automaton $\mathrsfs{B}_5$}\label{B5}
\end{figure}

The states of the automaton $\mathrsfs{B}_n$
are the residues modulo $n$ and its input letters $a$ and $b$ act
as follows:
$$
 \delta(m,a)=
 \begin{cases}
  m + 1 \!\!\pmod{n} & \text{for $m = 0, 1$}, \\
  m \!\!\pmod{n} & \text{for $1 < m < n$};
  \end{cases}
\qquad
 \delta(m,b)=\begin{cases}
  m + 2 \!\!\pmod{n} & \text{for $m = 0$}, \\
  m + 1 \!\!\pmod{n} & \text{for $m \neq 0$};
  \end{cases}.
$$
The smallest automaton in the series is shown in Fig.~\ref{B5}.

\newpage

\begin{theorem}\label{theo}
\theoremtext{n^2-4n+6}
\end{theorem}

\begin{lemma}
\lemmatext{b(ab^{n - 2})^{n - 3}ab}
\end{lemma}


\begin{figure}[th]
\unitlength=.7mm
\begin{center}
\begin{picture}(100,70)(0,-80)
\gasset{Nadjustdist=1.5} \node(n0)(56.0,-16.0){0}
\node(n1)(24.0,-40.0){4} \node(n2)(88.0,-40.0){1}
\node(n3)(36.0,-72.0){3} \node(n4)(76.0,-72.0){2}
\drawloop[ELdist=2.4,loopangle=320.0](n4){$a$}
\drawedge[ELdist=2.0](n1,n0){$b$} \drawedge[ELdist=1.5](n2,n4){$b$}
\drawedge[ELdist=1.7](n0,n2){$b$} \drawedge[ELdist=2.0](n4,n3){$b$}
\drawedge[ELdist=1.7](n3,n1){$b$}
\node[NLangle=0.0](n16)(56.0,-48.0){5}
%\drawedge[ELdist=1.6](n1,n16){$a$}
\drawedge[ELdist=1.7](n16,n4){$b$}

\drawedge[curvedepth=2](n0,n16){$a$}
\drawedge[curvedepth=2](n16,n0){$a$}

%\drawloop[ELdist=1.5,loopangle=90](n0){$a$}
%\drawloop[ELdist=1.5](n16){$b$}
%\drawedge[ELdist=1.5,curvedepth=33](n3,n0){$a$}
\drawloop[ELdist=1.5,loopangle=144.55](n1){$a$}
\drawloop[ELdist=1.5,loopangle=226.55](n3){$a$}
\drawloop[ELdist=1.5,loopangle=33.34](n2){$a$}
\end{picture}
\end{center}
\caption{The automaton $\mathrsfs{B}^{\star}_6$}\label{B-star-6}
\end{figure}

\newpage

\begin{theorem}\label{theo}
\theoremtext{n^2-4n+6}
\end{theorem}

\begin{lemma}
\lemmatext{b(ab^{n - 2})^{n - 3}ba}
\end{lemma}


\begin{figure}[th]
\unitlength=.7mm
\begin{center}
\begin{picture}(100,70)(0,-80)
\gasset{Nadjustdist=1.5} \node(n0)(56.0,-16.0){0}
\node(n1)(24.0,-40.0){4} \node(n2)(88.0,-40.0){1}
\node(n3)(36.0,-72.0){3} \node(n4)(76.0,-72.0){2}
\drawloop[ELdist=2.4,loopangle=320.0](n4){$a$}
\drawedge[ELdist=2.0](n1,n0){$b$} \drawedge[ELdist=1.5](n2,n4){$a, b$}
\drawedge[ELdist=1.7](n0,n2){$b$} \drawedge[ELdist=2.0](n4,n3){$b$}
\drawedge[ELdist=1.7](n3,n1){$b$}
\node[NLangle=0.0](n16)(56.0,-48.0){5}
\drawedge[ELdist=1.7](n16,n4){$b$}

\drawedge[curvedepth=2](n0,n16){$a$}
\drawedge[curvedepth=2](n16,n0){$a$}

\drawloop[ELdist=1.5,loopangle=144.55](n1){$a$}
\drawloop[ELdist=1.5,loopangle=226.55](n3){$a$}
\end{picture}
\end{center}
\caption{The automaton $\mathrsfs{B}^{\star}_6$}\label{B-star-6}
\end{figure}

\newpage


\begin{thebibliography}{99}
\itemsep -2pt

\bibitem{AVZ}
Dimitry S. Ananichev, Mikhail V. Volkov, Yu. I. Zaks,
Synchronizing automata with a letter of deficiency 2. Theor.\ Comput.\ Sci.
376 (2007)  30--41.

\bibitem{AV05}
D.\,S.\,Ananichev, M.\,V.\,Volkov, Synchronizing generalized
monotonic automata, Theoret.\ Comput.\ Sci. 330 (2005) 3--13.

\bibitem{Ce64}
J.\,\v{C}ern\'{y}, Pozn\'amka k homog\'{e}nnym eksperimentom s
konecn\'{y}mi automatami, Mat.-Fyz.\ Cas.\ Slovensk.\ Akad.\ Vied.
14 (1964) 208--216 [in Slovak].

\bibitem{Du98}
L.\,Dubuc, Sur le automates circulaires et la conjecture de
\v{C}ern\'y, RAIRO Inform.\ Theor.\ Appl., 32 (1998) 21--34 [in
French].

\bibitem{Ep90}
D.\,Eppstein, Reset sequences for monotonic automata, SIAM J.
Comput. 19 (1990) 500--510.

\bibitem{Fr82}
P.\,Frankl, An extremal problem for two families of sets, Eur.\ J.
Comb. 3 (1982) 125--127.

\bibitem{Go93}
K.\,Goldberg, Orienting polygonal parts without sensors,
Algorithmica 10 (1993) 201--225.

\bibitem{Hi88}
P.\,M.\,Higgins, The range order of a product of $i$
transformations from a finite full transformation semigroup,
Semigroup Forum 37 (1988) 31--36.

\bibitem{Ka03}
J.\,Kari, Synchronizing finite automata on Eulerian digraphs,
Theoret.\ Comput.\ Sci. 295 (2003) 223--232.

\bibitem{MS99}
A.\,Mateescu, A.\,Salomaa, Many-valued truth functions,
\v{C}ern\'{y}'s conjecture and road coloring, EATCS Bull. 68
(1999) 134--150.

\bibitem{Pi81}
J.-E.\,Pin, Le probl\'eme de la synchronisation et la conjecture
de \v{C}ern\'{y}, in A.\,De\,Luca (ed.), Non-commutative
Structures in Algebra and Geometric Combinatorics [Quaderni de la
Ricerca Scientifica 109], CNR, Roma, 1981, 37--48 [in French].

\bibitem{Pi83}
J.-E.\,Pin, On two combinatorial problems arising from automata
theory, Ann.\ Discrete Math. 17 (1983) 535--548.

\bibitem{Sa03}
A.\,Salomaa, Composition sequences for functions over a finite
domain, Theoret.\ Comput.\ Sci. 292 (2003) 263--281.

\bibitem{Sa05}
S.\,Sandberg, Homing and synchronizing sequences, in M.\,Broy
et~al. (eds.), Model-Based Testing of Reactive Systems [Lect.\
Notes Comput.\ Sci. 3472], Springer-Verlag, Berlin--Heidelberg,
2005, 5--33.

\bibitem{Tr-JALC}
A.\,N.\,Trahtman, The \v{C}ern\'y conjecture for aperiodic
automata, J. Automata, Languages and Combinatorics 11 (2006).

\bibitem{Tr06}
A.\,N.\,Trahtman, An efficient algorithm finds noticeable trends
and examples concerning the \v{C}ern\'y conjecture, in
R.\,Kr\'alovi\v{c}, P.\,Urzyczyn (eds.), Mathematical Foundations
of Computer Science 2006 [Lect.\ Notes Comput.\ Sci. 4162],
Springer-Verlag, Berlin--Heidelberg, 2006, 789�-800.

\end{thebibliography}

\newpage
\section*{Appendix}
\end{document}
