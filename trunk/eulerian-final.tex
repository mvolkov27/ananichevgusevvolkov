\documentclass[11pt]{llncs}
\usepackage{amsmath}
\usepackage{amssymb}
\usepackage{epic,gastex}
\usepackage{array}

\newcommand{\sa}{synchronizing automata}
\newcommand{\sea}{synchronizing eulerian automata}
\newcommand{\san}{synchronizing automaton}
\newcommand{\sean}{synchronizing eulerian automaton}
\newcommand{\sw}{reset word}
\newcommand{\sws}{reset words}
\newcommand{\ssw}{reset word of minimum length}
\newcommand{\rl}{reset length}
\newcommand{\reth}{reset threshold}

%\newtheorem{theorem}{Theorem}
%\newtheorem{lemma}[theorem]{Lemma}
%\newtheorem{proposition}[theorem]{Proposition}
%\newtheorem{corollary}{Corollary}

\DeclareMathOperator{\rt}{rt}

\DeclareSymbolFont{rsfscript}{OMS}{rsfs}{m}{n}
\DeclareSymbolFontAlphabet{\mathrsfs}{rsfscript}



%\markboth{V. Gusev}{LOWER BOUNDS FOR THE RESET LENGTH IN EULERIAN AUTOMATA}


\begin{document}

\title{Lower Bounds for the Length of Reset Words in Eulerian Automata\thanks{Supported
by the Russian Foundation for Basic Research, grant 10-01-00524, and by the
Federal Education Agency of Russia, grant 2.1.1/13995.}}
\author{Vladimir V. Gusev}
\authorrunning{V. V. Gusev}
\institute{Ural State University, Ekaterinburg, Russia\\
\email{vl.gusev@gmail.com}}

\maketitle

\begin{abstract}
For each odd $n\ge 5$ we present a synchronizing Eulerian automaton with $n$
states for which the minimum length of reset words is equal to $\frac{n^2 - 3n
+ 4}{2}$. We also discuss various connections between the \reth\ of a \san\ and
a sequence of reachability properties in its underlying graph.
\end{abstract}

%\thispagestyle{empty}

\section{Background and overview}
\label{intro}


A complete deterministic finite automaton $\mathrsfs{A}$ is called
\emph{synchronizing} if the action of some word $w$ resets $\mathrsfs{A}$, that
is, leaves the automaton in one particular state no matter at which state it is
applied. Any such word $w$ is said to be a \emph{reset word} for the automaton.
The minimum length of reset words for $\mathrsfs{A}$ is called the \emph{\reth}
of $\mathrsfs{A}$ and denoted by $\rt(\mathrsfs{A})$. Synchronizing automata
constitute an interesting combinatorial object and naturally appear in many
applications such as coding theory, robotics and testing of reactive systems.
For a brief introduction to the theory of synchronizing automata we refer the
reader to the recent surveys~\cite{Sa05,Vo08}. The interest to the field is
also heated by the famous \v{C}ern\'{y} conjecture.

In~1964 Jan \v{C}ern\'{y}~\cite{Ce64} constructed for each $n>1$ a
synchronizing automaton $\mathrsfs{C}_n$ with $n$ states whose \reth\ is
$(n-1)^2$. Soon after that he conjectured that these automata represent the
worst possible case, that is, every synchronizing automaton with $n$ states can
be reset by a word of length $(n-1)^2$. Despite intensive research, the best
upper bound on the \reth\ of synchronizing automata with $n$ states achieved so
far is $\frac{n^3-n}6$, see~\cite{Pi83}, so it is much larger than the
conjectured value. Though the \v{C}ern\'{y} conjecture is open in general, it
has been confirmed for various restricted classes of synchronizing automata,
see, e.g., \cite{Ep90,Du98,Ka03,Tr07,Vo09}. We recall here a result by Jarkko
Kari from~\cite{Ka03} as it has served as a departure point for the present
paper.

Kari~\cite{Ka03} has shown that every synchronizing Eulerian automaton with $n$
states possesses a reset word of length at most $n^2 -3n + 3$. Even though this
result confirms the \v{C}ern\'{y} conjecture for Eulerian automata, it does not
close the synchronizability question for this class of automata since no
matching lower bound for the \reth\ of Eulerian automata has been found so far.
In order to find such a matching bound, we need a series of Eulerian automata
with large \reth. It is this problem that we address in the present paper.

Our first attempt was following an approach from~\cite{AGV}. In that paper,
several examples of slowly synchronizing automata, which had been discovered in
the course of a massive computational experiment, have been related to known
examples of primitive graphs with large exponent from~\cite{DM64} and then have
been expanded to infinite series. The idea was to apply a similar analysis to
Eulerian graphs with large exponent that have been characterized
in~\cite{Shen00}. However, it turns out that this way we cannot achieve results
close to what we get by computational experiments. Thus, a refinement of the
approach from~\cite{AGV} appears to necessary. Here we suggest such a
refinement, and this is the main novelty of the present paper. As a concrete
demonstration of our modified approach, we exhibit a series of slowly
synchronizing Eulerian automata whose \reth\ is twice as large as the \reth\ of
automata that can be obtained by a direct application of techniques
from~\cite{AGV}. We believe that the method suggested in this paper can find a
number of other applications and its further development may shed a new light
on the properties of synchronizing automata.


\section{Preliminaries}
\label{preliminaries}
A \emph{complete deterministic finite automaton} (DFA) is
a couple $\mathrsfs{A}=\langle Q,\Sigma\rangle$, where $Q$ stands for the
\emph{state set} and $\Sigma$ for the \emph{input alphabet} whose letters act
on $Q$ by totally defined transformations. The action of $\Sigma$ on $Q$
extends in natural way to an action of the set $\Sigma^*$ of all words over
$\Sigma$. The result of the action of a word $w \in \Sigma^*$ on the state $q
\in Q$ is denoted by $q \cdot w$. Triples of the from $(q,a,q\cdot a)$ where
$q\in Q$ and $a\in\Sigma$ are called \emph{transitions} of the DFA; $q$, $a$
and $q\cdot a$ are referred to as, respectively, the \emph{source}, the
\emph{label} and the \emph{target} of the transition $(q,a,q\cdot a)$.

By a \emph{graph} we mean a tuple of sets and maps: the set of \emph{vertices}
$V$, the set of \emph{edges} $E$, a map $t: E\to V$ that maps every edge to its
\emph{tail} vertex, and a map $h:E\to V$ that maps every edge to its
\emph{head} vertex. Notice that in a graph, there may be several edges with the
same tail and head. (Thus, our graphs are in fact directed multigraphs but
since no other graph species will show up in this paper, we use a short name.)
We assume the reader's acquaintance with basic notions of the theory of graphs
such as path, cycle, isomorphism etc.

Given a DFA $\mathrsfs{A}=\langle Q,\Sigma\rangle$, its \emph{underlying graph}
$D(\mathrsfs{A})$ has $Q$ as the vertex set and has an edge $e_\tau$ with
$t(e_\tau)=q$, $h(e_\tau)=q\cdot a$ for each transition $\tau=(q,a,q\cdot a)$
of $\mathrsfs{A}$. We stress that if two transitions have a common source and a
common target (but different labels), then they give rise to different edges
(with a common tail and a common head). It is easy to see that a graph $D$ is
isomorphic to the underlying graph of some DFA if and only if each vertex of
$D$ serves as the tail for the same number of edges (the number is called the
\emph{outdegree} of $D$). In the sequel, we always consider only graphs
satisfying this property. Every DFA $\mathrsfs{A}$ such that $D\cong
D(\mathrsfs{A})$ is called a \emph{coloring} of $D$. Thus, every coloring of
$D$ is labeling its edges of by letters from some alphabet whose cardinality is
equal to the outdegree of $D$ such that edges with a common tail get different
colors. Fig.\,\ref{fig:two colorings} shows a graph and two of its colorings by
$\Sigma=\{a,b\}$.
\begin{figure}[ht]
 \begin{center}
  \unitlength=2.4pt
    \begin{picture}(18,30)(-70,-4)
    \gasset{Nw=6,Nh=6,Nmr=3}
    \node(A1)(0,18){$1$}
    \node(A2)(18,18){$2$}
    \node(A3)(18,0){$3$}
    \node(A4)(0,0){$4$}
    \drawloop[loopangle=135](A1){$a$}
    \drawloop[loopangle=45](A2){$b$}
    \drawloop[loopangle=-45](A3){$b$}
    \drawedge(A1,A2){$b$}
    \drawedge(A2,A3){$a$}
    \drawedge(A3,A4){$a$}
    \drawedge[curvedepth=3](A4,A1){$a$}
    \drawedge[curvedepth=-3,ELside=r](A4,A1){$b$}
    \end{picture}
 \begin{picture}(18,30)(0,-4)
    \gasset{Nw=6,Nh=6,Nmr=3}
    \node(A1)(0,18){$1$}
    \node(A2)(18,18){$2$}
    \node(A3)(18,0){$3$}
    \node(A4)(0,0){$4$}
    \drawloop[loopangle=135](A1){$a$}
    \drawloop[loopangle=45](A2){$a$}
    \drawloop[loopangle=-45](A3){$a$}
    \drawedge(A1,A2){$b$}
    \drawedge(A2,A3){$b$}
    \drawedge(A3,A4){$b$}
    \drawedge[curvedepth=3](A4,A1){$a$}
    \drawedge[curvedepth=-3,ELside=r](A4,A1){$b$}
    \end{picture}
 \begin{picture}(18,30)(70,-4)
    \gasset{Nw=6,Nh=6,Nmr=3}
    \node(A1)(0,18){$1$}
    \node(A2)(18,18){$2$}
    \node(A3)(18,0){$3$}
    \node(A4)(0,0){$4$}
    \drawloop[loopangle=135](A1){}
    \drawloop[loopangle=45](A2){}
    \drawloop[loopangle=-45](A3){}
    \drawedge(A1,A2){}
    \drawedge(A2,A3){}
    \drawedge(A3,A4){}
    \drawedge[curvedepth=3](A4,A1){}
    \drawedge[curvedepth=-3](A4,A1){}
    \end{picture}
 \end{center}
 \caption{A graph and two of its colorings}
 \label{fig:two colorings}
\end{figure}

A graph $D=\langle V,E\rangle$ is said to be \emph{strongly connected} if for
every pair $(v,v')\in V\times V$, there exists a path from $v$ to $v'$. A graph
is \emph{Eulerian} if it is strongly connected and each of its vertices serves
as the tail and as the head for the same number of edges. A DFA is said to be
\emph{Eulerian} if so is its underlying graph. More generally, we will freely
transfer graph notions (such as path, cycle, etc) from graphs to automata they
underlie.

A graph $D=\langle V,E\rangle$ is called \emph{primitive} if there exists a
positive integer $t$ such that for every pair $(v,v')\in V\times V$, there
exists a path from $v$ to $v'$ of length precisely $t$. The least $t$ with this
property is called the \emph{exponent} of the digraph $D$ and is denoted by
$\exp(D)$.

Let $w$ be a word over the alphabet $\Sigma = \{a_1, a_2, \ldots, a_k\}$. We
say that a word $u\in\Sigma^*$ \emph{occurs $\ell$ times as a factor of $w$} if
there are exactly $\ell$ different words $x_1, \ldots, x_\ell \in \Sigma^*$
such that for each $i$, $1 \leq i \leq \ell$, there is a word $y_i \in
\Sigma^*$ for which $w$ decomposes as $w = x_iuy_i$. The number $\ell$ is
called the \emph{number of occurrences of $u$ in $w$} and is denoted by
$|w|_u$. The vector $(|w|_{a_1}, |w|_{a_2}, \ldots,
|w|_{a_k})\in\mathbb{N}^k_0$ is called the \emph{Parikh vector} of the word
$w$; here $\mathbb{N}_0$ stands for the set of non-negative integers.

Now suppose that $\mathrsfs{A}=\langle Q,\Sigma\rangle$ is a DFA and $\alpha$
is a path in $\mathrsfs{A}$ labelled by a word $w\in\Sigma^*$. If a vector
$\mathsf{v}\in\mathbb{N}^k_0$ is equal to the Parikh vector of $w$, then we say
that $\mathsf{v}$ is the \emph{Parikh vector} of the path $\alpha$. We refer to
any path that has $\mathsf{v}$ as its Parikh vector as a
$\mathsf{v}$-\emph{path}.

\section{Main results}
\label{sea}

We start with revisiting the technique used in \cite{AGV} to obtain lower
bounds for the \reth\ of certain \sa.

Consider an arbitrary synchronizing automaton $\mathrsfs{A} = \langle Q, \Sigma
\rangle$. Let $w$ be a \sw\ for $\mathrsfs{A}$ that leaves the automaton in
some state $r \in Q$, that is, $p\cdot w=r$ for every $p\in Q$. Then, for every
state $p \in Q$, the word $w$ labels a path from $p$ to $r$. Therefore, for
every state $p \in Q$ there is a path of length $|w|$ from $p$ to $r$ in the
underlying graph $D(\mathrsfs{A})$. This leads us to the following notion. We
say that a strongly connected graph $D=(V,E)$ is 0-\emph{primitive} if there
exists an integer $k > 0$ and a vertex $r \in V$ such that for every vertex
$p\in V$ there is a path of length exactly $k$ from $p$ to $r$. The minimal
integer $k$ with this property (over all possible choices of $r$) is called the
0-\emph{exponent} of $D$ and is denoted by $\exp_0(D)$. We write
$\exp_0(\mathrsfs{A})$ instead of $\exp_0(D(\mathrsfs{A}))$. Then we have that
every synchronizing automaton $\mathrsfs{A}$ is 0-primitive and
\begin{equation}
\label{first inequality} \rt(\mathrsfs{A})\geq \exp_0(\mathrsfs{A}).
\end{equation}

It is not hard to see that the notions of 0-primitivity and primitivity are
equivalent. Indeed, every primitive digraph is obviously 0-primitive.
Conversely, let $D$ be a 0-primitive digraph with $n$ vertices. By the
definition there are paths of length exactly $\exp_0(D)$ from every vertex to
some fixed vertex $r$. Consider two arbitrary vertices $p$ and $q$ of $D$.
Since $D$ is strongly connected, there is a path $\alpha$ of length at most $n
- 1$ from $r$ to $q$. Now take any path $\beta$ of length $n - 1 - |\alpha|$
starting at $p$ and let $s$ be the endpoint of $\beta$. There is a path
$\gamma$ of length $\exp_0(D)$ from $s$ to $r$. Now the path
$\beta\gamma\alpha$ leads from $p$ to $q$ (through $s$ and $r$) and
$|\beta\gamma\alpha|=n - 1 + \exp_0(D)$. Thus, the digraph $D$ is primitive,
and moreover, we have the following inequality:
\begin{equation}
\label{second inequality} \exp_0(D) + n - 1 \geq \exp(D).
\end{equation}
The reader who may wonder why we need such a slight variation of the standard
notion will see that this variation fits better into a more general framework
that we will present below.

First, however, we demonstrate how to construct Eulerian automata with a
relatively large \reth\ on the basis of the notion of 0-primitivity. For this,
we need Eulerian digraphs with the largest possible exponent (or 0-exponent)
among all primitive Eulerian digraphs with $n$ vertices. Such digraphs have
been classified by Shen~\cite{Shen00}.

For every odd $n \geq 5$, consider the automaton $\mathrsfs{D}_n$ with the
state set $Q=\{1,2,\dots,n\}$ and the input letters $a$ and $b$ acting on $Q$
as follows:

$ 1 \cdot a = 2$, $1 \cdot b = 3$; $(n - 1) \cdot a = 2$, $(n - 1) \cdot b =
1$; $n \cdot a = 1$, $n \cdot b = 3$; and for every $1 < k < n - 1$
$$k \cdot a=\begin{cases}
k + 2 &\text{if } k \text{ is even},\\
k + 1 &\text{if } k \text{ is odd};
\end{cases}\quad
k \cdot b=\begin{cases}
k + 3 &\text{if } k \text{ is even},\\
k + 2 &\text{if } k \text{ is odd}.
\end{cases}$$
The automaton $\mathrsfs{D}_n$ is shown in Fig.\,\ref{fig:d}. We denote the
underlying graph of $\mathrsfs{D}_n$ by $\mathcal{D}_n$.

\begin{figure}[h]
  \begin{center}
    \unitlength=4pt
    \begin{picture}(75, 35)(0,-7)
    \gasset{curvedepth=0}
    \thinlines
    \node(A0)(0,7){$1$}
    \node(A1)(12,15){$2$}
    \node(A2)(12,0){$3$}
    \node(A3)(27,15){$4$}
    \node(A4)(27,0){$5$}
    \node(A5)(43,15){$6$}
    \node(A6)(43,0){$7$}
    \node(An3)(60,15){$n{-}3$}
    \node(An2)(60,0){$n{-}2$}
    \node(An1)(75,15){$n{-}1$}
    \node(An)(75,0){$n$}
    \drawedge(A0,A1){$a$}
    \drawedge[ELpos=45](A0,A2){$b$}
    \drawedge[ELpos=45](A1,A3){$a$}
    \drawedge[ELpos=30](A1,A4){$b$}
    \drawedge[ELpos=30](A2,A3){$a$}
    \drawedge[ELpos=45](A2,A4){$b$}
    \drawedge[ELpos=45](A3,A5){$a$}
    \drawedge[ELpos=30](A3,A6){$b$}
    \drawedge[ELpos=30](A4,A5){$a$}
    \drawedge[ELpos=45](A4,A6){$b$}
     \put(50,0){$\ldots$}
     \put(50,15){$\ldots$}
    \drawedge[ELpos=45](An3,An1){$a$}
    \drawedge[ELpos=30](An3,An){$b$}
    \drawedge[ELpos=30](An2,An1){$a$}
    \drawedge[ELpos=45](An2,An){$b$}
    \drawcurve(72.7,17)(60,19)(27,19)(14.2, 17)
    \drawcurve(74.5,18)(60,22)(27,23)(10,20)(0, 10)
    \put(39,24){$b$}
    \put(45,20){$a$}
    \drawcurve(72.7,-2)(60,-4)(27,-4)(14.2, -2)
    \drawcurve(74.5,-3)(60,-7)(27,-7)(10,-5)(0, 4)
    \put(39,-9){$a$}
    \put(45,-6.5){$b$}
    \end{picture}
  \end{center}
  \caption{The automaton $\mathrsfs{D}_n$}
  \label{fig:d}
\end{figure}


\begin{proposition}
\label{shen} If $\mathcal{G}$ is a primitive Eulerian graph with outdegree $2$
and $n$ vertices, $n \geq 8$, then $\exp(\mathcal{G}) \leq \frac{(n - 1)^2}{4}
+ 1$. The equality holds only for the graph $\mathcal{D}_n$.
\end{proposition}

Proposition~\ref{shen} and the inequalities \eqref{first inequality} and
\eqref{second inequality} guarantee that every synchronizing coloring of the
graph $\mathcal{D}_n$ has \reth\ of magnitude $\frac{n^2}{4} + o(n^2)$. In
particular, we can prove the following result using the technique developed
in~\cite{AGV}.

\begin{proposition}
The \reth\ of the automaton $\mathrsfs{D}_n$ is equal to $\frac{n^2 - 4n +
11}{4}$.
\end{proposition}

\begin{proof}
We start with estimating $\exp_0(\mathrsfs{D}_n)$. Observe that for every $\ell
\geq \exp_0(\mathrsfs{D}_n)$ there is a cycle of length $\ell$ in
$\mathrsfs{D}_n$. Indeed, let $r$ be a state such that for every $p \in Q$
there is a path of length $\exp_0(\mathrsfs{D}_n)$ from $p$ to $r$. Now take an
arbitrary path $\alpha$ of length $\ell - \exp_0(\mathrsfs{D}_n)$ starting at
$r$ and let $s$ be the endpoint of $\alpha$. By the choice of $r$, there is a
path $\beta$ of length $\exp_0(\mathrsfs{D}_n)$ from $s$ to $r$. Thus, the path
$\alpha\beta$ is a cycle of length exactly $\ell$.

Now consider the partition $\pi$ of the set $Q$ into $\frac{n+1}{2}$ classes
$V_i$, $0\le i \le \frac{n - 1}{2}$, where $V_0 = {1}$ and $V_i = \{2i,2i+1\}$
for every $0 < i \le \frac{n - 1}{2}$. We define a graph $\mathcal{G}_n$ with
the quotient set $Q/\pi$ as the vertex set and with the edges induced by the
edges of  $\mathcal{D}_n$ as follows: there is an edge $e'$ in $\mathcal{G}_n$
with $t(e') = V_i$ and $h(e) = V_j$ if and only if there is an edge $e$ in
$\mathcal{D}_n$ with $t(e)\in V_i$ and $h(e) \in V_j$. Then every cycle in
$\mathcal{D}_n$ induces a cycle of the same length in $\mathcal{G}_n$. In
particular,  for every $\ell \ge \exp_0(\mathrsfs{D}_n)$ there is a cycle of
length $\ell$ in $\mathcal{G}_n$. It is easy to see that the graph
$\mathcal{G}_n$ has precisely two simple cycles: one of length $\frac{n -
1}{2}$ and one of length $\frac{n + 1}{2}$. We conclude that every $\ell\ge
\exp_0(\mathrsfs{D}_n)$ is expressible as a non-negative integer combination of
$\frac{n - 1}{2}$ and $\frac{n + 1}{2}$.

Here we invoke the following well-known and elementary result from arithmetics:
\begin{lemma}[{\mdseries\cite[Theorem 2.1.1]{RaAl05}}]
\label{sylvester} If $k_1,k_2$ are relatively prime positive integers, then
$k_1k_2-k_1-k_2$ is the largest integer that is not expressible as a
non-negative integer combination of $k_1$ and $k_2$.
\end{lemma}
Applying lemma~\ref{sylvester} we conclude that $\exp_0(\mathrsfs{D}_n) \geq
\frac{n^2 -4n + 3}{4}$ and there is no cycle of length $\frac{n^2 -4n - 1}{4}$
in $\mathcal{G}_n$. The inequality~\ref{first inequality} implies that
$\rt(\mathrsfs{D}_n) \geq \frac{n^2 -4n + 3}{4}$, and it remains to exclude two
cases: $\rt(\mathrsfs{D}_n)= \frac{n^2 -4n + 3}{4}$ and $\rt(\mathrsfs{D}_n) =
\frac{n^2 -4n + 7}{4}$. This is easy.

Suppose that $w$ is a shortest \sw\ for $\mathrsfs{D}_n$ which leaves
$\mathrsfs{D}_n$ in some state $r \in V_i$. Note that $i \neq 0$ (otherwise the
word obtained by removing the last letter from $w$ would be a shorter \sw, and
this is impossible).

If $|w|=\frac{n^2 -4n + 3}{4}$, we write $w = xw'$ for some letter $x$ and
apply the word $w$ to some state from $V_{i - 1}$. We conclude that $w'$
induces a cycle from $V_i$ to $V_i$ in $\mathcal{G}_n$. This cycle would be of
length $\frac{n^2 -4n - 1}{4}$, which is impossible.

Finally suppose that the length of $w$ is $\frac{n^2 -4n + 7}{4}$. If $i \neq
1$, then the same argument as in the previous paragraph leads to a
contradiction. (We just apply $w$ to a state from $V_{i - 2}$.)  If $i = 1$,
let $w = xyw'$ for some letters $x$ and $y$. Depending on $x$, either $n \cdot
xy \in V_1$ or $(n - 1) \cdot xy \in V_1$. In both cases $w'$ induces a cycle
from $V_1$ to $V_1$ in $\mathcal{G}_n$ of length $\frac{n^2 -4n - 1}{4}$, which
is impossible.

We thus see that the \reth\ of the automaton $\mathrsfs{D}_n$ is at least
$\frac{n^2 -4n + 11}{4}$. Since the word $aa(ba^{\frac{n - 1}{2}})^\frac{n -
5}{2}bb$ resets $\mathrsfs{D}_n$, we conclude that this bound is tight.
\end{proof}

Our computational experiments show that the largest \reth\ among all
synchronizing colorings of $\mathcal{D}_n$ is equal to $\frac{(n - 1)^2}{4} +
1$. Therefore, $\frac{n^2}{4} + o(n^2)$ is the best lower bound on the \reth\
of synchronizing Eulerian automata with $n$ states that can be obtained by a
direct encoding of Eulerian graphs with large exponent. However, our main
result (Theorem~\ref{th:matr} below) shows that for every odd $n$ there is a
synchronizing Eulerian automaton with $n$ states and \reth\ $\frac{n^2 -3n +
4}{2}$. This indicates that the notion of 0-exponent is too weak to be useful
for isolating synchronizing Eulerian automata with maximal \reth. The reason
for this is that we have discarded too much information when passing from
synchronizability to 0-primitivity---we forget everything about paths labelled
by \sws\ except their length. Thus, we try to use another notion in which more
information is preserved, namely, the Parikh vectors of the paths are taken
into account.

Consider a DFA $\mathrsfs{A} = \langle Q, \Sigma \rangle$ with $|\Sigma| = k$
and fix some ordering of the letters in $\Sigma$. We define a subset
$\mathrm{E}_1(\mathrsfs{A})$ of $\mathbb{N}^{k}_0$ as follows: a vector
$\mathsf{v} \in \mathbb{N}^{k}_0$ belongs to $\mathrm{E}_1(\mathrsfs{A})$ if
and only if there is state $r \in Q$ such that for every $p \in Q$, there
exists a $\mathsf{v}$-path from $p$ to $r$. If the set
$\mathrm{E}_1(\mathrsfs{A})$ is non-empty, then the automaton $\mathrsfs{A}$ is
called \emph{$1$-primitive}. The minimum value of the sum $i_1 + i_2 + \dots +
i_k$ over all $k$-tuples $(i_1, i_2, \ldots, i_k)$ from
$\mathrm{E}_1(\mathrsfs{A})$ is called the \emph{$1$-exponent} of
$\mathrsfs{A}$ and denoted by $\exp_1(\mathrsfs{A})$. We would like to note
that a very close concept for colored multigraphs has been studied in
\cite{ShSu03,OlShDr02}. Clearly, every synchronizing automaton $\mathrsfs{A}$
is 1-primitive and
\begin{equation}
\label{third inequality} \rt(\mathrsfs{A})\geq \exp_1(\mathrsfs{A}).
\end{equation}

In order to illustrate how the notion of $1$-exponent may be utilized, we prove
a statement concerning the \v{C}ern\'{y} automata $\mathrsfs{C}_n$ (this
statement will be used in the proof of our main result). Recall the definition
of $\mathrsfs{C}_n$. The state set of $\mathrsfs{C}_n$ is $Q=\{1,2,\dots,n\}$
and the letters $a$ and $b$ act on $Q$ as follows:
$$i \cdot a =\begin{cases}
2 &\text{if } i = 1,\\
i &\text{if } 1 < i;
\end{cases}\quad
i \cdot b =\begin{cases}
i+1 &\text{if } i<n,\\
1 &\text{if } i=n.
\end{cases}$$
The automaton $\mathrsfs{C}_n$ for $n = 7$ is shown in Fig.~\ref{fig:cerny}.
Here and below we adopt the convention that edges bearing multiple labels
represent bunches of edges sharing tails and heads. In particular, the edge
$1\xrightarrow{a,b}2$ in Fig.\,\ref{fig:cerny} represents the two parallel
edges $1\xrightarrow{a}2$ and $1\xrightarrow{b}2$.

\begin{figure}[ht]
\begin{center}
\unitlength .5mm
\begin{picture}(50,80)(-20,-80)
\gasset{Nw=14,Nh=14,Nmr=7,loopdiam=10}
\node(n0)(0,-5){1}
\node(n1)(-30,-20){7} \node(n2)(30,-20){2}
\node(n3)(-35,-50){6} \node(n4)(35,-50){3}
\node(n5)(-17,-75){5} \node(n6)(17,-75){4}
\drawedge[ELdist=2.0](n1,n0){$b$}
\drawedge[ELdist=1.5](n2,n4){$b$}
\drawedge[ELdist=1.7](n0,n2){$a, b$}
\drawedge[ELdist=1.7](n3,n1){$b$}
\drawedge[ELdist=1.7](n4,n6){$b$}
\drawedge[ELdist=1.7](n6,n5){$b$}
\drawedge[ELdist=1.7](n5,n3){$b$}
\drawloop[ELdist=1.5,loopangle=25](n2){$a$}
\drawloop[ELdist=1.5,loopangle=150](n1){$a$}
\drawloop[ELdist=2.4,loopangle=340](n4){$a$}
\drawloop[ELdist=1.5,loopangle=200](n3){$a$}
\drawloop[ELdist=1.5,loopangle=300](n6){$a$}
\drawloop[ELdist=1.5,loopangle=240](n5){$a$}
\end{picture}
\end{center}
\caption{The automaton $\mathrsfs{C}_n$ for $n = 7$} \label{fig:cerny}
\end{figure}


\begin{proposition}
\label{th:cerny} Every reset word of the automaton $\mathrsfs{C}_n$ contains at
least $n^2 - 3n + 2$ occurrences of the letter $b$ and at least $n - 1$
occurrence of the letter $a$.
\end{proposition}

\begin{proof}
Since the automaton $\mathrsfs{C}_n$ is synchronizing, the set
$\mathrm{E}_1(\mathrsfs{C}_n)$ is non-empty. We make use of the following
simple property of $\mathrm{E}_1(\mathrsfs{C}_n)$: if $\mathsf{v} = (\alpha,
\beta) \in \mathrm{E}_1(\mathrsfs{C}_n)$, then for every $t \in \mathbb{N}$ we
have $(\alpha, \beta + t) \in \mathrm{E}_1(\mathrsfs{C}_n)$. Indeed, let $r$ be
a state such that for every $p \in Q$ there is a $\mathsf{v}$-path from $p$ to
$r$. We aim to show that there is also an $(\alpha, \beta + t)$-path from an
arbitrary state $p$ to $r$. Let $q = p\cdot b^{t}$, then by definition of $r$
there is a $\mathsf{v}$-path from $q$ to $r$. Augmenting this path in the
beginning by the path staring at $p$ and labeled $b^t$, we obtain an $(\alpha,
\beta + t)$-path from $p$ to $r$.

Now observe that there is a $\mathsf{v}$-path from $r$ to $r$. This path is a
cycle and it can be decomposed into simple cycles of the automaton
$\mathrsfs{C}_n$. The simple paths in $\mathrsfs{C}_n$ are loops labeled $a$
with the Parikh vector $(1,0)$, the cycle
$$1\xrightarrow{a}2\xrightarrow{b}3\xrightarrow{b}\dots\xrightarrow{b}n-1\xrightarrow{b}n\xrightarrow{b}1$$
with the Parikh vector $(1,n-1)$ and the cycle
$$1\xrightarrow{b}2\xrightarrow{b}3\xrightarrow{b}\dots\xrightarrow{b}n-1\xrightarrow{b}n\xrightarrow{b}1$$
with the Parikh vector $(1,n-1)$. Thus, there are some $x, y, z \in
\mathbb{N}_0$ such that the following equality holds true:
$$(\alpha,\beta) = x (1,0) + y (1, n - 1) + z(0,n).$$
It readily implies that $\beta = y(n - 1) + zn$. Since for every $t \in
\mathbb{N}$ the vector $(\alpha, \beta + t)$ also belongs to
$\mathrm{E}_1(\mathrsfs{C}_n)$, we conclude that $\beta + t$ is also
expressible as a non-negative integer combination of $n$ and $n - 1$.
Lemma~\ref{sylvester} implies that $\beta\ge n(n-1)-n-(n-1)+1=n^2-3n+2$. If $w$
is a \sw\ of the automaton $\mathrsfs{C}_n$, then the Parikh vector of $w$
belongs to $\mathrm{E}_1(\mathrsfs{C}_n)$, whence $w$ contains at least $n^2 -
3n + 2$ occurrences of the letter $b$.

It remains to prove that $w$ contains at least $n - 1$ occurrences of the
letter $a$. Note that for every set $S$ of states, we have $|S\cdot b|=|S|$ and
$|S\cdot a|\ge|S|-1$. Hence, to decrease the cardinality from $n$ to 1, one has
to apply $a$ at least $n-1$ times, and any word $w$ such that $|Q\cdot w|=1$
must contain at least $n - 1$ occurrences of $a$.
\end{proof}

As a corollary we immediately obtain \v{C}ern\'{y}'s
result~\cite[Lemma~1]{Ce64} that $\rt(\mathrsfs{C}_n)=(n - 1)^2$. Indeed,
Proposition~\ref{th:cerny} implies that the \reth\ is at least $(n - 1)^2$, and
it is easy to check that the word $(ab^{n - 1})^{n - 2}a$ of length $(n - 1)^2$
resets $\mathrsfs{C}_n$. Also we see that a \sw\ $w$ of minimal length for
$\mathrsfs{C}_n$ is unique. Indeed, $w$ cannot start or end with $b$ because
$b$ acts as a cyclic permutation. Thus, $w = aua$ and the word $u$ has $n^2 -
3n + 2$ occurrences of $b$ and $n - 3$ occurrences of $a$. Note that $b^n$
cannot occur as a factor of $u$ since $b^n$ acts is an identity mapping.
Clearly, there is only one way to insert $n - 3$ letters $a$ in the word
$b^{n^2-3n+2}$ such that the resulting word contains no factor $b^n$. Though
the series $\mathrsfs{C}_n$ is very well studied, to the best of our knowledge
the uniqueness of the shortest \sw\ for $\mathrsfs{C}_n$ has not been
explicitly stated in the literature.

Observe that $\exp_0(\mathrsfs{C}_n)=n-1$ and we could not extract any strong
lower bound for $\rt(\mathrsfs{C}_n)$ from the inequality~\eqref{second
inequality}. In~\cite{AGV} a tight lower bound for $\rt(\mathrsfs{C}_n)$ has
been obtained in an indirect way, via relating $\mathrsfs{C}_n$ to graphs with
largest possible 0-exponent from \cite{Wi50}. In contrast,
Proposition~\ref{th:cerny} implies that $\exp_1(\mathrsfs{C}_n)$ is close to
$(n-1)^2$ so the inequality~\eqref{third inequality} gives a stronger lower
bound.

Now we are ready to present main result of this paper. We define the automaton
$\mathrsfs{M}_n$ (from Matricaria) on the state set $Q=\{1,2,\dots,n\}$, where
$n \geq 5$ is odd, in which the letters $a$ and $b$ act as follows:
$$k \cdot a=\begin{cases}
k &\text{if } k \text{ is odd},\\
k + 1 &\text{if } k \text{ is even};
\end{cases}\quad
k \cdot b=\begin{cases}
k + 1 &\text{if } k \neq n \text{ is odd},\\
k &\text{if } k \text{ is even},\\
1 &\text{if } k = n.
\end{cases}$$
Observe that $\mathrsfs{M}_n$ is Eulerian. The automaton $\mathrsfs{M}_n$ for
$n = 7$ is shown in Fig.~\ref{fig:matr} on the left.
\begin{figure}[ht]
\begin{center}
\unitlength .5mm
\begin{picture}(50,100)(40,-88)
\gasset{Nw=14,Nh=14,Nmr=7,loopdiam=10}
\node(n0)(0,-5){1}
\node(n1)(-30,-20){7} \node(n2)(30,-20){2}
\node(n3)(-35,-50){6} \node(n4)(35,-50){3}
\node(n5)(-17,-75){5} \node(n6)(17,-75){4}
\drawedge[ELdist=2.0](n1,n0){$b$}
\drawedge[ELdist=1.5](n2,n4){$a$}
\drawedge[ELdist=1.7](n0,n2){$b$}
\drawedge[ELdist=1.7](n3,n1){$a$}
\drawedge[ELdist=1.7](n4,n6){$b$}
\drawedge[ELdist=1.7](n6,n5){$a$}
\drawedge[ELdist=1.7](n5,n3){$b$}
\drawloop[ELdist=1.5,loopangle=90](n0){$a$}
\drawloop[ELdist=1.5,loopangle=25](n2){$b$}
\drawloop[ELdist=1.5,loopangle=150](n1){$a$}
\drawloop[ELdist=2.4,loopangle=340](n4){$a$}
\drawloop[ELdist=1.5,loopangle=200](n3){$b$}
\drawloop[ELdist=1.5,loopangle=300](n6){$b$}
\drawloop[ELdist=1.5,loopangle=240](n5){$a$}
\end{picture}

\begin{picture}(1,1)(-65,-96)
\gasset{Nw=14,Nh=14,Nmr=7,loopdiam=10}
\node(n0)(0,-5){1}
\node(n1)(-30,-20){7} \node(n2)(30,-20){2}
\node(n3)(-35,-50){6} \node(n4)(35,-50){3}
\node(n5)(-17,-75){5} \node(n6)(17,-75){4}
\drawedge[ELdist=2.0](n1,n0){$b,c$}
\drawedge[ELdist=1.5,ELside=r](n2,n6){$c$}
\drawedge[ELdist=1.5](n0,n2){$b,c$}
\drawedge[ELdist=1.7,ELside=r](n3,n0){$c$}
\drawedge[ELdist=1.7](n4,n6){$b,c$}
\drawedge[ELdist=1.7,ELside=r](n6,n3){$c$}
\drawedge[ELdist=1.7](n5,n3){$b,c$}
\drawloop[ELdist=1.5,loopangle=25](n2){$b$}
\drawloop[ELdist=1.5,loopangle=200](n3){$b$}
\drawloop[ELdist=1.5,loopangle=300](n6){$b$}
\end{picture}
\end{center}
\caption{The automaton $\mathrsfs{M}_n$ for $n=7$ and the automaton induced by
the actions of $b$ and $c=ab$} \label{fig:matr}
\end{figure}

\begin{theorem}
\label{th:matr} If $n \geq 5$ is odd, then the automaton $\mathrsfs{M}_n$ is
synchronizing and its \reth\ is equal to $\frac{n^2 -3n + 4}{2}$.
\end{theorem}

\begin{proof}
Let $w$ be a \sw\ of minimum length for $\mathrsfs{M}_n$. Note that the action
of $aa$ is the same as the action of $a$. Therefore $aa$ could not be a factor
of $w$. (Otherwise reducing this factor to just $a$ results in a shorter \sw.)
So every occurrence of $a$, maybe except the last one, is followed by $b$. If
we let $c = ab$, then either $w$ or $wb$ (if $w$ ends with $a$) could be
rewritten into a word $u$ over the alphabet $\{b,c\}$. The actions of $b$ and
$c$ induce a new automaton on the state set of $\mathrsfs{M}_n$ (this induced
automaton is shown in Fig.~\ref{fig:matr} on the right). It is not hard to see
that in both cases $u$ is a \sw\ for the induced automaton. After applying the
first letter of $u$ it remains to synchronize the subautomaton on the set of
states $S =\{1\} \cup \{2k \mid 1 \leq k \leq \frac{n - 1}{2}\}$, and this
subautomaton is isomorphic to $\mathrsfs{C}_{\frac{n + 1}{2}}$.

Suppose $u = u'c$ for some word $u'$ over $\{b,c\}$. Since the action of $c$ on
any subset of $S$ cannot decrease its cardinality, we conclude that $u'$ is
also a \sw\ for the induced automaton. But $c$ is the last letter of $u$ only
if $w = w'a$ and $w'$ was rewritten into $u'$. Thus, $w'$ also is a \sw\ for
$\mathrsfs{M}_n$, which is a contradiction.

If $u = xu'$ for some letter $x$, then by Proposition~\ref{th:cerny} we
conclude that $u'$ has at least $(\frac{n + 1}{2})^2 - 3(\frac{n + 1}{2}) + 2 =
\frac{n^2 - 4n + 3}{4}$ occurrences of $c$ and at least $\frac{n - 1}{2}$
occurrences of $b$. Since each occurrence of $c$ in $u'$ corresponds to an
occurrence of the factor $ab$ in $w$, we conclude that the length of $w$ is at
least $1 + 2\frac{n^2 - 4n + 3}{4} + \frac{n - 1}{2} = \frac{n^2 - 3n + 4}{2}$.

One can verify that the word $b(b(ab)^{\frac{n - 1}{2}})^{\frac{n - 3}{2}}b$ is
a \sw\ for $\mathrsfs{M}_n$ whence the above bound is tight.
\end{proof}

It is not hard to see that $\exp_0(\mathrsfs{M}_n)=n-1$ and also
$\exp_1(\mathrsfs{M}_n)$ is linear in $n$. Thus, both 0-exponent and 1-exponent
are far too weak to give a good lower bound for the \reth\ of $\mathrsfs{M}_n$.
That is why we have obtained a tight lower bound for $\rt(\mathrsfs{M}_n)$ in an
indirect way, via relating $\mathrsfs{M}_n$ to and automaton with a large
1-exponent (namely, to $\mathrsfs{C}_{\frac{n + 1}{2}}$). Can one develop a
notion that would give a good bound in a more direct way?


Observe that the most important part of the proof of Theorem~\ref{th:matr}
deals with estimating the number of occurrences of the factor $ab$ in a reset
word. In fact, a rough estimation can be done directly. Let $w$ be a \sw\ that
leaves $\mathrsfs{M}_n$ in the state 2 and $k=|w|_{ab}$. Consider a path from 2
to 2 in which the state 2 does not occur in the middle. Words labeling such
paths come from the language $L=b^*(a^+b^+)^{\frac{n - 1}{2}}ba^*b$. Thus, $w$
can be divided into several blocks from $L$. Since every block has either
$\frac{n - 1}{2}$ or $\frac{n+1}{2}$ occurrences of the factor $ab$, we
conclude that $k$ is expressible as a non-negative integer combination of the
numbers $\frac{n-1}{2}$ and $\frac{n+1}{2}$. Note that $(ab)^tw$, where $t \in
\mathbb{N}$, is a \sw\ that leaves $\mathrsfs{M}_n$ in the state 2. Since $ab$
occurs $k + t$ times as a factor in $(ab)^tw$, we see that $k + t$ also is
expressible as a non-negative integer combination of $\frac{n-1}{2}$ and
$\frac{n+1}{2}$. Applying lemma~\ref{sylvester} we conclude that $k \geq
\frac{n^2 - 4n + 3}{4}$. Thus, the length of $w$ is at least $\frac{n^2 - 4n +
3}{2}$.

The above reasoning  suggests the following generalization. Let $\mathrsfs{A} =
\langle Q, \Sigma \rangle$ be a DFA with $Q=\{1,2,\dots,n\}$ and let $k$ be a
non-negative integer. We say that the automaton $\mathrsfs{A}$ is
\emph{$k$-primitive} if there exist words $u_1,u_2,\dots,u_n$ such that $1\cdot
u_1=2\cdot u_2=\dots=n\cdot u_n$ and for every word $v$ of length at most $k$
we have $|u_1|_v = |u_2|_v = \ldots = |u_n|_v$. Note that the last condition
implies that all words $u_1,u_2,\dots,u_n$ have the same length. The minimal
length of words that witness $k$-primitivity of  $\mathrsfs{A}$ is called the
\emph{$k$-exponent} of $\mathrsfs{A}$ and is denoted by $\exp_k(\mathrsfs{A})$.
Observe that the rough estimation in the previous paragraph shows that
$\exp_2(\mathrsfs{M}_n)$ is close to $\rt(\mathrsfs{M}_n)$.

Consider now an arbitrary synchronizing automaton $\mathrsfs{A}$. It is clear
that $\mathrsfs{A}$ is $k$-primitive for every $k$  and $\rt(\mathrsfs{A}) \geq
\exp_k(\mathrsfs{A})$. Thus, we have the following non-decreasing sequence:
\begin{equation}
\label{sequence}
\exp_0(\mathrsfs{A}) \leq \exp_1(\mathrsfs{A}) \leq \dots \leq
\exp_k(\mathrsfs{A}) \leq \exp_{k + 1}(\mathrsfs{A}) \leq \dotsc.
\end{equation}
At every next step we require that words $u_1,u_2,\dots,u_n$ get more similar
to each other than they were in previous step. Thus, sooner or later these
words converge to a \sw\ and the sequence stabilizes at $\rt(\mathrsfs{A})$.
Thus, we think that studying the sequence~\eqref{sequence} may lead to a new
approach to the \v{C}ern\'{y} conjecture.

\begin{thebibliography}{99}

\bibitem{AGV}
Ananichev, D.S., Gusev, V.V., Volkov, M.V.: Slowly synchronizing automata and
digraphs. MFCS 2010, LNCS 6281,  55--65 (2010)


\bibitem{Ce64}
\v{C}ern\'{y}, J.: Pozn\'{a}mka k homog\'{e}nnym eksperimentom s
kone\v{c}n\'{y}mi automatami. Matem.-fyzikalny \v{C}asopis Slovensk.\ Akad.\
Vied 14(3), 208--216 (1964) (in Slovak)

\bibitem{Du98}
Dubuc, L.: Sur les automates circulaires et la conjecture de
\v{C}ern\'y. RAIRO Inform.\ Th\'eor.\ Appl. 32, 21--34 (1998) (in
French)

\bibitem{DM64}
Dulmage, A.L., Mendelsohn, N.S.: Gaps in the exponent set of primitive
matrices. Ill.\ J. Math. 8, 642--656 (1964)

\bibitem{Ep90}
Eppstein, D.: Reset sequences for monotonic automata. SIAM J.
Comput. 19, 500--510 (1990)


\bibitem{Ka03}
Kari, J.: Synchronizing finite automata on Eulerian digraphs.
Theoret.\ Comput.\ Sci. 295, 223--232 (2003)

\bibitem{OlShDr02}
Olesky, D.D., Shader, B., van den Driessche, P.:
Exponents of tuples of nonnegative matrices. Linear Algebra Appl. 356,
123--134 (2002)

\bibitem{Pi83}
Pin, J.-E.: On two combinatorial problems arising from automata
theory. Ann.\ Discrete Math. 17, 535--548 (1983)

\bibitem{RaAl05}
Ram\'{\i}rez Alfons\'{\i}n, J.L.: The diophantine Frobenius problem.
Oxford University Press (2005)

\bibitem{Sa05}
Sandberg, S.: Homing and synchronizing sequences,
Model-Based Testing of Reactive Systems 2004, LNCS 3472,
5--33 (2005)

\bibitem{ShSu03}
Shader, B.L., Suwilo, S.: Exponents of nonnegative matrix
pairs, Linear Algebra Appl. 363, 275--293 (2003)


\bibitem{Shen00}
Shen, J.: Exponents of 2-regular digraphs, Discrete Math. 214,
211--219 (2000)

\bibitem{Tr07}
Trahtman, A.N.: The \v{C}ern\'y conjecture for aperiodic automata. Discrete
Math.\ Theor.\ Comput.\ Sci. 9(2), 3--10 (2007)

\bibitem{Vo08}
Volkov, M.V.: Synchronizing automata and the \v{C}ern\'{y}
conjecture, LATA 2008, LNCS 5196, 11--27 (2008)

\bibitem{Vo09}
Volkov, M.V.: Synchronizing automata preserving a chain of partial
orders, Theoret. Comput. Sci. 410, 2992--2998 (2009)

\bibitem{Wi50}
Wielandt, H.: Unzerlegbare, nicht negative Matrizen. Math.\ Z.
52, 642--648 (1950) (in German)
\end{thebibliography}


\end{document}
