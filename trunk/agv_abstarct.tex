\documentclass[11pt]{llncs}
\usepackage{amsmath}
\usepackage{amssymb}
\usepackage{epic,gastex}

\newcommand{\sa}{synchronizing automata}
\newcommand{\san}{synchronizing automaton}
\newcommand{\ssw}{a reset word of minimum length}


\DeclareMathOperator{\weight}{wg}

\DeclareSymbolFont{rsfscript}{OMS}{rsfs}{m}{n}
\DeclareSymbolFontAlphabet{\mathrsfs}{rsfscript}

\newcommand{\theoremtext}[1]{
For each $n$, there exists strongly connected \san\ with $n$ states,
such that it's shortest reset word is of length $#1$.
}

\newcommand{\lemmatext}[1]{
The word $#1$ is a reset word for the automaton $\mathrsfs{B}_n$.}


\begin{document}
\title{Slowly synchronizing automata and digraphs\thanks{Supported
by the Russian Foundation for Basic Research, grants 09-01-12142
and 10-01-00524, and by the Federal Education Agency of Russia,
grant 2.1.1/3537.}}

\titlerunning{Slowly synchronizing automata and digraphs}

\author{D. S. Ananichev \and V. V. Gusev \and M. V. Volkov}

\authorrunning{D. S. Ananichev, V. V. Gusev, M. V. Volkov}

\tocauthor{D. S. Ananichev, V. V. Gusev, M. V. Volkov
(Ekaterinburg, Russia)}

\institute{Department of Mathematics and Mechanics,\\
Ural State University, 620083 Ekaterinburg, RUSSIA\\
\email{Dmitry.Ananichev@usu.ru, vl.gusev@gmail.com,
Mikhail.Volkov@usu.ru}}

\maketitle


\begin{center}
\textbf{\large Extended abstract}
\end{center}

A \emph{complete deterministic finite automaton} (DFA) is a triple
$\mathrsfs{A}=\langle Q,\Sigma,\delta\rangle$, where $Q$ and
$\Sigma$ are finite sets called the \emph{state set} and the
\emph{input alphabet} respectively, and $\delta:Q\times\Sigma\to
Q$ is a totally defined function called the \emph{transition
function}. Let $\Sigma^*$ stand for the collection of all finite
words over the alphabet $\Sigma$, including the empty word.
The function $\delta$ extends to a function $Q\times\Sigma^*\to Q$
(still denoted by $\delta$) in the following natural way: for every
$q\in Q$ and $w\in\Sigma^*$, we set
$$\delta(q,w)=\begin{cases}
q &\text{ if $w$ is empty},\\
\delta(\delta(q,v),a) &\text{ if $w=va$ for some $v\in\Sigma^*$ and $a\in\Sigma$}.
\end{cases}$$
Thus, via $\delta$, every word $w\in\Sigma^*$ acts on the set $Q$.

A DFA $\mathrsfs{A}=\langle Q,\Sigma,\delta\rangle$ is called
\emph{synchronizing} if the action of some word $w\in\Sigma^*$
resets $\mathrsfs{A}$, that is, leaves the automaton in one
particular state no matter at which state in $Q$ it is applied:
$\delta(q,w)=\delta(q',w)$ for all $q,q'\in Q$. Any such word $w$
is said to be a \emph{reset word} for the DFA.

Synchronizing automata serve as transparent and natural models of
error-resistant systems in many applications (coding theory, robotics,
testing of reactive systems) and also reveal interesting connections
with symbolic dynamics and other parts of mathematics. For a brief
introduction to the theory of \sa\ we refer the reader to the recent
surveys~\cite{Sa05,Vo08}. Here we focus on the so-called \v{C}ern\'{y}
conjecture that constitutes a major open problem in this area.

In~1964 \v{C}ern\'{y}~\cite{Ce64} constructed for each $n>1$ a \san\
$\mathrsfs{C}_n$ with $n$ states whose shortest reset word has length
$(n-1)^2$. Soon after that he conjectured that these automata
represent the worst possible case, that is, every \san\ with $n$ states
can be reset by a word of length $(n-1)^2$. This imply looking conjecture
resists researchers' efforts for more than 40 years. Even though the
conjecture has been confirmed for various restricted classes of \sa\
(cf., e.g., \cite{Ep90,Du98,Ka03,Tr07,AS09,Vo09}), no upper bound of magnitude
$O(n^2)$ for the minimum length of reset words for $n$-state \sa\ is known
in general---the best upper bound achieved so far is $\frac{n^3-n}6$,
see~\cite{Pi83}.

One of the difficulties that one encounters when approaching the
\v{C}ern\'{y} conjecture is that there are only very few examples
of \emph{extreme} \sa, that is, $n$-state \sa\ whose shortest reset
words have length $(n-1)^2$. In fact, the \v{C}ern\'{y} series
$\mathrsfs{C}_n$, $n=2,3,\dotsc$, is the only known infinite series
of extreme \sa. Besides that, we know only a few isolated examples
of such automata, see~\cite{Vo08} for a complete list. Moreover,
even \emph{slowly} \sa, that is, \sa\ whose shortest reset words have
length close to the \v{C}ern\'{y} bound are very rare. This empirical
observation is supported also by probabilistic arguments. For instance,
Higgins~\cite{Hi88} has shown that the probability that a composition of $2n$
random self-maps of a set of size $n$ is a constant map tends to~1 as $n$
goes to infinity. In terms of automata, Higgins's result means that
a random automaton with $n$ states and at least $2n$ input letters
has a reset word of length $2n$. For further results of the same flavor
see~\cite{SZ}. Thus, there is no hope to find new examples of slowly
\sa\ via a random sampling experiment.

We therefore have designed and performed a set of exhaustive search
experiments. Our experiments are briefly described in Section~5
while the main body of the paper is devoted to a theoretical analysis
of their outcome. We concentrate on two principal issues. In Section~3
we discuss a remarkable analogy between the distribution of lengths of shortest
reset words for \sa\ and the distribution of exponents of primitive matrices.
Section~4 collects several new series of slowly \sa. The ``initial''
examples in the series were found in the course of the experiments; then
each example was expanded to a series of automata that have been proved
to be slowly synchronizing. In our opinion, the proof technique is also
of interest; in fact, we provide a transparent and uniform approach to all
sufficiently large slowly \sa\ with 2~input letters, both new and already
known ones.

\begin{thebibliography}{99}
\bibitem{AS09}
Almeida, J.; Steinberg, B.: Matrix mortality and the \v{C}ern\'{y}--Pin
conjecture. In:  Diekert, V.; Nowotka, D. (eds.), Developments in
Language Theory, Lect.\ Notes Comput.\ Sci., vol.\,5583, pp. 67--80.
Springer, Heidelberg (2009)

\bibitem{AVZ}
Ananichev, D.S., Volkov, M.V., Zaks, Yu.I.: Synchronizing automata
with a letter of deficiency 2. Theor.\ Comput.\ Sci. 376, 30--41 (2007)

\bibitem{Ce64}
\v{C}ern\'{y}, J.: Pozn\'{a}mka k homog\'{e}nnym eksperimentom s
kone\v{c}n\'{y}mi automatami. Matematicko-fyzikalny \v{C}asopis
Slovensk.\ Akad.\ Vied 14(3) 208--216 (1964) (in Slovak)

\bibitem{Du98}
Dubuc, L.: Sur les automates circulaires et la conjecture de
\v{C}ern\'y. RAIRO Inform.\ Th\'eor.\ Appl. 32, 21--34 (1998) (in
French)

\bibitem{Ep90}
Eppstein, D.: Reset sequences for monotonic automata. SIAM J.
Comput. 19, 500--510 (1990)

\bibitem{Hi88}
Higgins, P.M.: The range order of a product of $i$ transformations
from a finite full transformation semigroup, Semigroup Forum 37, 31--36
(1988)

\bibitem{Ka03}
Kari, J.: Synchronizing finite automata on Eulerian digraphs.
Theoret.\ Comput.\ Sci. 295, 223--232 (2003)

\bibitem{Pi83}
Pin, J.-E.: On two combinatorial problems arising from automata
theory. Ann.\ Discrete Math. 17, 535--548 (1983)

\bibitem{Sa03}
A.\,Salomaa, Composition sequences for functions over a finite
domain, Theoret.\ Comput.\ Sci. 292 (2003) 263--281.

\bibitem{Sa05}
Sandberg, S.: Homing and synchronizing sequences. In: Broy, M.
et~al. (eds.), Model-Based Testing of Reactive Systems. Lect.\
Notes Comput.\ Sci., vol.\,3472, pp.\,5--33. Springer, Heidelberg
(2005)

\bibitem{SZ}
Skvortsov E., Zaks Yu.: Synchronizing random automata. Submitted;
proceedings version in: Rigo, M. (ed.), AutoMathA 2009, Universit\'e
de Li\`ege (2009)

\bibitem{Tr06}
Trahtman, A.: An efficient algorithm finds noticeable trends
and examples concerning the \v{C}ern\'y conjecture. In:
Kr\'alovi\v{c}, R.; Urzyczyn, P. (eds.), Mathematical Foundations
of Computer Science. Lect.\ Notes Comput.\ Sci., vol.\,4162, pp.\,789--800
Springer, Heidelberg (2006)

\bibitem{Tr07}
Trahtman, A.: The \v{C}ern\'y conjecture for aperiodic automata.
Discrete Math.\ Theor.\ Comput.\ Sci. 9(2), 3--10 (2007)

\bibitem{Vo08}
Volkov, M.V.: Synchronizing automata and the \v{C}ern\'{y}
conjecture. In: Mart\'\i{}n-Vide, C.; Otto, F.; Fernau, H. (eds.),
Languages and Automata: Theory and Applications. Lect.\ Notes
Comput.\ Sci., vol.\,5196, pp.\,11--27.  Springer, Heidelberg (2008)

\bibitem{Vo09}
Volkov, M.V.: Synchronizing automata preserving a chain of partial
orders. Theoret.\ Comput.\ Sci. 410, 2992--2998 (2009)
\end{thebibliography}




\end{document}
