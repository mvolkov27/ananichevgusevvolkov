\documentclass[11pt]{llncs}
\usepackage{amsmath}
\usepackage{amssymb}
\usepackage{epic,gastex}
\usepackage{array}

\newcommand{\sa}{synchronizing automata}
\newcommand{\san}{synchronizing automaton}
\newcommand{\sw}{reset word}
\newcommand{\sws}{reset words}
\newcommand{\ssw}{reset word of minimum length}

\DeclareMathOperator{\weight}{wg}

\DeclareSymbolFont{rsfscript}{OMS}{rsfs}{m}{n}
\DeclareSymbolFontAlphabet{\mathrsfs}{rsfscript}

\begin{document}
\title{Slowly synchronizing automata and digraphs\thanks{Supported
by the Russian Foundation for Basic Research, grants 09-01-12142
and 10-01-00524, and by the Federal Education Agency of Russia,
grant 2.1.1/3537.}}

\titlerunning{Slowly synchronizing automata and digraphs}

\author{D. S. Ananichev \and V. V. Gusev \and M. V. Volkov}

\authorrunning{D. S. Ananichev, V. V. Gusev, M. V. Volkov}

\tocauthor{D. S. Ananichev, V. V. Gusev, M. V. Volkov
(Ekaterinburg, Russia)}

\institute{Department of Mathematics and Mechanics,\\
Ural State University, 620083 Ekaterinburg, RUSSIA\\
\email{Dmitry.Ananichev@usu.ru, vl.gusev@gmail.com,
Mikhail.Volkov@usu.ru}}

\maketitle

\begin{abstract}
We present several infinite series of \sa\ for which
the minimum length of reset words is close to the
square of the number of states. These automata are
closely related to primitive  matrices with large exponent.
\end{abstract}


\section{Background and overview}
\label{intro}

A \emph{complete deterministic finite automaton} (DFA) is a triple
$\mathrsfs{A}=\langle Q,\Sigma,\delta\rangle$, where $Q$ and
$\Sigma$ are finite sets called the \emph{state set} and the
\emph{input alphabet} respectively, and $\delta:Q\times\Sigma\to
Q$ is a totally defined function called the \emph{transition
function}. Let $\Sigma^*$ stand for the collection of all finite
words over the alphabet $\Sigma$, including the empty word.
The function $\delta$ extends to a function $Q\times\Sigma^*\to Q$
(still denoted by $\delta$) in the following natural way: for every
$q\in Q$ and $w\in\Sigma^*$, we set
$$\delta(q,w)=\begin{cases}
q &\text{ if $w$ is empty},\\
\delta(\delta(q,v),a) &\text{ if $w=va$ for some $v\in\Sigma^*$ and $a\in\Sigma$}.
\end{cases}$$
Thus, via $\delta$, every word $w\in\Sigma^*$ acts on the set $Q$.

A DFA $\mathrsfs{A}=\langle Q,\Sigma,\delta\rangle$ is called
\emph{synchronizing} if the action of some word $w\in\Sigma^*$
resets $\mathrsfs{A}$, that is, leaves the automaton in one
particular state no matter at which state in $Q$ it is applied:
$\delta(q,w)=\delta(q',w)$ for all $q,q'\in Q$. Any such word $w$
is said to be a \emph{reset word} for the DFA.

Synchronizing automata serve as transparent and natural models of
error-resistant systems in many applications (coding theory, robotics,
testing of reactive systems) and also reveal interesting connections
with symbolic dynamics and other parts of mathematics. For a brief
introduction to the theory of \sa\ we refer the reader to the recent
surveys~\cite{Sa05,Vo08}. Here we focus on the so-called \v{C}ern\'{y}
conjecture that constitutes a major open problem in this area.

In~1964 \v{C}ern\'{y}~\cite{Ce64} constructed for each $n>1$ a \san\
$\mathrsfs{C}_n$ with $n$ states whose shortest reset word has length
$(n-1)^2$. Soon after that he conjectured that these automata
represent the worst possible case, that is, every \san\ with $n$ states
can be reset by a word of length $(n-1)^2$. This simply looking conjecture
resists researchers' efforts for more than 40 years. Even though the
conjecture has been confirmed for various restricted classes of \sa\
(cf., e.g., \cite{Ep90,Du98,Ka03,Tr07,AS09,Vo09}), no upper bound of magnitude
$O(n^2)$ for the minimum length of reset words for $n$-state \sa\ is known
in general---the best upper bound achieved so far is $\frac{n^3-n}6$,
see~\cite{Pi83}.

One of the difficulties that one encounters when approaching the
\v{C}ern\'{y} conjecture is that there are only very few examples
of \emph{extreme} \sa, that is, $n$-state \sa\ whose shortest reset
words have length $(n-1)^2$. In fact, the \v{C}ern\'{y} series
$\mathrsfs{C}_n$, $n=2,3,\dotsc$, is the only known infinite series
of extreme \sa. Besides that, only a few isolated examples
of such automata have been found, see~\cite{Vo08} for a complete list.
Moreover, even \emph{slowly} \sa, that is, \sa\ whose shortest reset words
have length close to the \v{C}ern\'{y} bound are very rare. This empirical
observation is supported also by probabilistic arguments. For instance,
Higgins~\cite{Hi88} has shown that the probability that a composition of $2n$
random self-maps of a set of size $n$ is a constant map tends to~1 as $n$
goes to infinity. In terms of automata, Higgins's result means that
a random automaton with $n$ states and at least $2n$ input letters
has a reset word of length $2n$. For further results of the same flavor
see~\cite{SZ}. Thus, there is no hope to find new examples of slowly
\sa\ by a lucky chance or via a random sampling experiment.

We therefore have designed and performed a set of exhaustive search
experiments. Our experiments are briefly described in Section~\ref{experiments}
while the main body of the paper is devoted to a theoretical analysis
of their outcome. We concentrate on two principal issues. In Section~\ref{matrices}
we discuss a similarity between the distribution of lengths of shortest
reset words for \sa\ and the distribution of exponents of primitive digraphs.
Section~\ref{sss} collects several new series of slowly \sa. The ``initial''
examples in the series were found in the course of the experiments; then
each example was expanded to a series of automata that have been proved
to be slowly synchronizing. In our opinion, the proof technique is also
of interest; in fact, we provide a transparent and uniform approach to all
sufficiently large slowly \sa\ with 2~input letters, both new and already
known ones.

\section{Preliminaries}
\label{preliminaries}

We start with recalling two elementary and well-known number-theoretic results.
\begin{lemma}[{\mdseries\cite[Theorem 1.0.1]{RaAl05}}]
\label{schur}
If $k_1,\dots,k_m$ are positive integers whose greatest common divisor is equal
to\/ $1$, then there exists an integer $N$ such that every integer larger than $N$
is expressible as a non-negative integer combination of $k_1,\dots,k_m$.
\end{lemma}
The question of how the least $N$ with the property stated in Lemma~\ref{schur}
depends on the integers $k_1,\dots,k_m$ is known as the \emph{diophantine Frobenius
problem} and in general is highly non-trivial, see~\cite{RaAl05}. There is, however,
a simple special case which we frequently use in Section~\ref{sss}.
\begin{lemma}[{\mdseries\cite[Theorem 2.1.1]{RaAl05}}]
\label{sylvester}
If $k_1,k_2$ are relatively prime positive integers, then $k_1k_2-k_1-k_2$ is
the largest integer that is not expressible as a non-negative integer combination
of $k_1$ and $k_2$.
\end{lemma}

A \emph{directed graph} (digraph) is a pair $D=\langle V,E\rangle$ where $V$ is
a finite set and $E\subseteq V\times V$. We refer to elements of $V$ and $E$ as
\emph{vertices} and \emph{edges}. Observe that our definition allows loops but
excludes multiple edges. If $v,v'\in V$ and $e=(v,v')\in E$, the edge $e$ is said
to be \emph{outgoing} for $v$. We assume the reader's acquaintance with basic
notions of the theory of directed graphs such as (directed) path,
cycle, isomorphism etc.

Given a DFA $\mathrsfs{A}=\langle Q,\Sigma,\delta\rangle$, its \emph{underlying
digraph} $D(\mathrsfs{A})$ has $Q$ as the vertex set and $(q,q')\in Q\times Q$ is
an edge of $D(\mathrsfs{A})$ if and only if $q'=\delta(q,a)$ for some $a\in\Sigma$.
It is easy to see that a digraph $D$ is isomorphic to the underlying digraph of some
DFA if and only if each vertex of $D$ has at least one outgoing edge. In the sequel,
we always consider only digraphs satisfying this property. Every DFA $\mathrsfs{A}$
such that $D\cong D(\mathrsfs{A})$ is called a \emph{coloring} of $D$. Thus, every
coloring of $D$ is defined by assigning non-empty sets of labels (colors) from some
alphabet $\Sigma$ to edges of $D$ such that the label sets assigned to the outgoing
edges of each vertex form a partition of $\Sigma$. Fig.\,\ref{fig:cerny} shows a
digraph and two of its colorings by $\Sigma=\{a,b\}$.
\begin{figure}[ht]
 \begin{center}
  \unitlength=2.8pt
    \begin{picture}(18,26)(-60,-4)
    \gasset{Nw=6,Nh=6,Nmr=3}
    \node(A1)(0,18){$1$}
    \node(A2)(18,18){$2$}
    \node(A3)(18,0){$3$}
    \node(A4)(0,0){$4$}
    \drawloop[loopangle=135](A1){$a$}
    \drawloop[loopangle=45](A2){$b$}
    \drawloop[loopangle=-45](A3){$b$}
    \drawedge(A1,A2){$b$}
    \drawedge(A2,A3){$a$}
    \drawedge(A3,A4){$a$}
    \drawedge(A4,A1){$a,b$}
    \end{picture}
 \begin{picture}(18,26)(0,-4)
    \gasset{Nw=6,Nh=6,Nmr=3}
    \node(A1)(0,18){$1$}
    \node(A2)(18,18){$2$}
    \node(A3)(18,0){$3$}
    \node(A4)(0,0){$4$}
    \drawloop[loopangle=135](A1){$b$}
    \drawloop[loopangle=45](A2){$b$}
    \drawloop[loopangle=-45](A3){$b$}
    \drawedge(A1,A2){$a$}
    \drawedge(A2,A3){$a$}
    \drawedge(A3,A4){$a$}
    \drawedge(A4,A1){$a,b$}
    \end{picture}
 \begin{picture}(18,26)(60,-4)
    \gasset{Nw=6,Nh=6,Nmr=3}
    \node(A1)(0,18){$1$}
    \node(A2)(18,18){$2$}
    \node(A3)(18,0){$3$}
    \node(A4)(0,0){$4$}
    \drawloop[loopangle=135](A1){}
    \drawloop[loopangle=45](A2){}
    \drawloop[loopangle=-45](A3){}
    \drawedge(A1,A2){}
    \drawedge(A2,A3){}
    \drawedge(A3,A4){}
    \drawedge(A4,A1){}
    \end{picture}
 \end{center}
 \caption{A digraph and two of its colorings}
 \label{fig:cerny}
\end{figure}

The \emph{matrix} of a digraph $D=\langle V,E\rangle$ is just the incidence
matrix of the edge relation, that is, a $V\times V$-matrix whose entry in
the row $v$ and the column $v'$ is 1 if $(v,v')\in E$ and 0 otherwise. For
instance, the matrix of the digraph in Fig.~1 (with respect to the chosen
numbering of its vertices) is $\left(\begin{smallmatrix}1&1&0&0\\
0&1&1&0\\ 0&0&1&1\\ 1&0&0&0 \end{smallmatrix}\right)$. Conversely, given
an $n\times n$-matrix $P=(p_{ij})$ with non-negative real entries, we assign
to it a digraph $D(P)$ on the set $\{1,2,\dots,n\}$ as follows: $(i,j)$ is an
edge of $D(P)$ if and only if $p_{ij}>0$. This ``two-way'' correspondence
allows us to formulate in terms of digraphs several important for the sequel
notions and results which originated in the classical Perron--Frobenius theory
of non-negative matrices.

Recall that a digraph $D=\langle V,E\rangle$ is said to be \emph{strongly
connected} if for every pair $(v,v')\in V\times V$, there exists a path
from $v$ to $v'$. By the $t^{th}$ \emph{power}
of $D$ we mean the digraph $D^t$ with the same vertex set $V$, such that
$(v,v')\in V\times V$ is an edge of $D^t$ if and only if there is a path
in $D$ from $v$ to $v'$ of length precisely $t$. If $M$ is the matrix of $D$,
then the digraph $D^t$ can be equivalently defined as $D(M^t)$, where
$M^t$ is the usual $t^{th}$ power of $M$.

A strongly connected digraph $D$ is called \emph{primitive} if the greatest
common divisor of the lengths of all cycles in $D$ is equal to~1. (In the
literature such graphs are sometimes called \emph{aperiodic}.)
Lemma~\ref{schur} readily implies that if $D$ is a primitive digraph, then
in some power $D^t$ of $D$ every pair of vertices constitutes an edge, i.e.,
$D^t$ is a complete digraph with loops. (This is equivalent to saying that
every entry of the matrix $M^t$, where $M$ is the matrix of $D$, is positive.)
The least $t$ with this property is called the \emph{exponent} of the digraph $D$
and is denoted by $\gamma(D)$. We need some results on exponents of digraphs
summarized as follows.

\begin{theorem}
\label{dulmage}
\emph{(a) (Wielandt's theorem, see \cite{Wi50,DM62}, \cite[Theorem~1]{DM64})}
If a primitive graph $D$ has $n$~vertices, then  $\gamma(D)\le(n-1)^2+1$.

\emph{(b) \cite[Theorem~6 and Corollary 4]{DM64}} Up to isomorphism, there is
exactly one primitive graph $D$ on $n>2$ vertices for which $\gamma(D)=(n-1)^2+1$,
and exactly one for which $\gamma(D)=(n-1)^2$. The matrices of these digraphs
are
\begin{equation}
\label{wielandt}
\begin{pmatrix}
0 & 1 & 0 & \dots & 0 & 0\\
0 & 0 & 1 & \dots & 0 & 0\\
\hdotsfor{6}\\
0 & 0 & 0 & \dots & 0 & 1\\
1 & 1 & 0 & \dots & 0 & 0
\end{pmatrix} \text{ and }
\begin{pmatrix}
0 & 1 & 0 & \dots & 0 & 0\\
0 & 0 & 1 & \dots & 0 & 0\\
\hdotsfor{6}\\
1 & 0 & 0 & \dots & 0 & 1\\
1 & 1 & 0 & \dots & 0 & 0
\end{pmatrix}
\text{ respectively.}
\end{equation}

\emph{(c) \cite[Theorem~7]{DM64}} If $n>4$ is even, then there is no
primitive digraph $D$ on $n$ vertices such that
\begin{equation}
\label{even gap}
n^2-4n+6<\gamma(D)<(n-1)^2,
\end{equation}
and, up to isomorphism, there are exactly $3$ or exactly $4$ primitive
digraphs $D$ on $n$ vertices with $\gamma(D)=n^2-4n+6$, according
as $n$ is or is not a~multiple of $3$.

\emph{(d) \cite[Theorem~8]{DM64}} If $n>3$ is odd, then there is no
primitive digraph $D$ on $n$ vertices such that
\begin{equation}
\label{first odd gap}
n^2-3n+4<\gamma(D)<(n-1)^2,
\end{equation}
and, up to isomorphism, there is exactly one primitive graph $D$ on $n$ vertices
for which $\gamma(D)=n^2-3n+4$, exactly one for which $\gamma(D)=n^2-3n+3$, and
exactly two for which $\gamma(D)=n^2-3n+2$. The matrices of these digraphs
are:
\begin{equation}
\label{odd island}
\begin{pmatrix}
0 & 1 & 0 & \dots & 0 & 0\\
0 & 0 & 1 & \dots & 0 & 0\\
\hdotsfor{6}\\
0 & 0 & 0 & \dots & 1 & 0\\
0 & 0 & 0 & \dots & 0 & 1\\
1 & 0 & 1 & \dots & 0 & 0
\end{pmatrix}\!,
\begin{pmatrix}
0 & 1 & 0 & \dots & 0 & 0\\
0 & 0 & 1 & \dots & 0 & 0\\
\hdotsfor{6}\\
0 & 0 & 0 & \dots & 1 & 0\\
0 & 1 & 0 & \dots & 0 & 1\\
1 & 0 & 1 & \dots & 0 & 0
\end{pmatrix}\!,
\begin{pmatrix}
0 & 1 & 0 & \dots & 0 & 0\\
0 & 0 & 1 & \dots & 0 & 0\\
\hdotsfor{6}\\
1 & 0 & 0 & \dots & 1 & 0\\
0 & 1 & 0 & \dots & 0 & 1\\
1 & 0 & 1 & \dots & 0 & 0
\end{pmatrix}\!,
\begin{pmatrix}
0 & 1 & 0 & \dots & 0 & 0\\
0 & 0 & 1 & \dots & 0 & 0\\
\hdotsfor{6}\\
1 & 0 & 0 & \dots & 1 & 0\\
0 & 0 & 0 & \dots & 0 & 1\\
1 & 0 & 1 & \dots & 0 & 0
\end{pmatrix}\!.
\end{equation}

\emph{(e) \cite[Theorem~8]{DM64}} If $n>3$ is odd, then there is no
primitive digraph $D$ on $n$ vertices such that
\begin{equation}
\label{second odd gap}
n^2-4n+6<\gamma(D)<n^2-3n+2,
\end{equation}
and, up to isomorphism, there are exactly $3$ or exactly $4$ primitive digraphs $D$
on $n$ vertices with $\gamma(D)=n^2-4n+6$, according as $n$ is or is not
a~multiple of $3$.
\end{theorem}

\section{Exponents of digraphs vs lengths of reset words}
\label{matrices}

As mentioned in Section~\ref{intro}, this paper has grown
from certain observations made when we analyzed experimental
results. One such observation has been a similarity between
the ``upper parts'' of two sequences: the sequence of possible
lengths of shortest reset words for 2-letter \sa\ with $n$ states
and the sequence of possible exponents of primitive digraphs with
$n$ vertices. As it is clear from Theorem~\ref{dulmage}, the
upper part of the latter sequence has certain gaps whose sizes
and positions depend on the parity of $n$; our experiments have
revealed quite a similar pattern of gaps in the upper part
of the former sequence. Table~\ref{9 states} illustrates this
observation for $n=9$.

\begin{table}[h]
\extrarowheight=1pt
\caption{Exponents of primitive digraphs with $9$ vertices
vs lengths of shortest reset words for 2-letter \sa\ with $9$ states}\label{9 states}
\begin{tabular}{|p{5.4cm}||c|c|c|c|c|c|c|c|c|c|c|c|c|c|c|}
\hline
\centering{$N$} & 65 & 64 & 63 & 62 & 61 & 60 & 59 & 58 & 57 & 56 & 55 & 54 & 53 & 52 & 51 \\
\hline
\raggedright{Number of non-isomorphic primitive digraphs
with exponent} $N$
& \raisebox{-6pt}{1} & \raisebox{-6pt}{1} & \raisebox{-6pt}{0} & \raisebox{-6pt}{0} & \raisebox{-6pt}{0}
& \raisebox{-6pt}{0} & \raisebox{-6pt}{0} & \raisebox{-6pt}{1} & \raisebox{-6pt}{1} & \raisebox{-6pt}{2}
& \raisebox{-6pt}{0} & \raisebox{-6pt}{0} & \raisebox{-6pt}{0} & \raisebox{-6pt}{0} & \raisebox{-6pt}{4} \\
\hline
\raggedright{Number of non-isomorphic 2-letter \sa\ whose
shortest reset words have length $N$}
&\raisebox{-11pt}{0} &\raisebox{-11pt}{1} &\raisebox{-11pt}{0} &\raisebox{-11pt}{0} &\raisebox{-11pt}{0}
&\raisebox{-11pt}{0} &\raisebox{-11pt}{0} &\raisebox{-11pt}{1} &\raisebox{-11pt}{2} &\raisebox{-11pt}{3}
&\raisebox{-11pt}{0} &\raisebox{-11pt}{0} &\raisebox{-11pt}{0} &\raisebox{-11pt}{4} &\raisebox{-11pt}{4} \\
\hline
\end{tabular}
\end{table}
The data in the second row of Table~\ref{9 states} are calculated
from Theorem~\ref{dulmage}, while the data in the third row come from
our experiments.

Concerning gaps in the upper part of the sequence of possible lengths of shortest
reset words for 2-letter \sa\ with a given number of states, we notice that
the first gap has been registered in earlier experiments in the area. (Namely, according
to~\cite{Tr06,Tr06a}, for $n=7,8,9,10$ there exists no 2-letter \sa\ with $n$ states
whose shortest reset words have lengths between $n^2-2n$ and $n^2-3n+5$.) However,
to the best of our knowledge, the second gap as seen in Table~\ref{9 states} has not
been reported in the literature up to now.

We strongly believe that the observed similarity is more than a coincidence.
Clearly, there are deep connections between primitive digraphs and \sa. Indeed,
it is well known (see~\cite{AGW}) that if the underlying digraph of a \san\
is strongly connected that the digraph must be primitive; on the other hand,
as follows from Trahtman's proof~\cite{Tr09} of the so-called Road Coloring
conjecture by Adler, Goodwyn, and Weiss~\cite{AGW}, every primitive digraph
admits a synchronizing coloring. This, however, does not suffice to explain
similarities such as in Table~\ref{9 states} because many of slowly \sa\
``responsible'' for non-zero entries in the third row cannot be obtained
as colorings of primitive digraphs with large exponents corresponding
to non-zero entries in the second row. In the next section we demonstrate
some new connections between primitive digraphs with large exponents and
slowly \sa\ with two input letters. In this way, we derive all known
series of such automata and construct many new other ones.

\section{Series of slowly \sa}
\label{sss}

Due to space limitations, we present here only a part of our results on slowly \sa.
Namely, we restrict ourselves to series derived from primitive digraphs whose matrices
are listed in Theorem~\ref{dulmage}. We start with the digraph $W_n$ corresponding
to the first matrix in \eqref{wielandt}. The digraph (more precisely, its matrix)
first appeared in Wielandt's seminal paper~\cite{Wi50}. It has $n$ vertices
$1,2,\dots,n$, say, and the following $n+1$ edges: $(i,i+1)$ for $i=1,\dots,n-1$,
$(n,1)$, and $(n,2)$.

It is easy to see that, up to isomorphism and renaming of letters, there exists
a unique coloring of the digraph $W_n$ by two letters. Let $\mathrsfs{W}_n$ denote
this coloring. Fig.\,\ref{fig:anan} shows the digraph $W_n$ and the DFA $\mathrsfs{W}_n$.
\begin{figure}[ht]
\begin{center}
\unitlength .45mm
\begin{picture}(72,56)(20,-72)
\gasset{Nw=16,Nh=16,Nmr=8}
\node(n0)(36.0,-16.0){1}
\node(n1)(4.0,-40.0){$n$} \node(n2)(68.0,-40.0){2}
\node(n3)(16.0,-72.0){$n{-}1$} \node(n4)(56.0,-72.0){3}
\drawedge[ELdist=2.0](n1,n0){} \drawedge[ELdist=1.5](n2,n4){}
\drawedge[ELdist=1.7](n0,n2){}
\drawedge[ELdist=1.7](n3,n1){} \drawedge[ELdist=2.0](n1,n2){}
\put(31,-73){$\dots$}
\end{picture}
\begin{picture}(72,56)(-20,-72)
\gasset{Nw=16,Nh=16,Nmr=8}
\node(n0)(36.0,-16.0){1}
\node(n1)(4.0,-40.0){$n$} \node(n2)(68.0,-40.0){2}
\node(n3)(16.0,-72.0){$n{-}1$} \node(n4)(56.0,-72.0){3}
\drawedge[ELdist=2.0](n1,n0){$b$} \drawedge[ELdist=1.5](n2,n4){$a, b$}
\drawedge[ELdist=1.7](n0,n2){$a, b$}
\drawedge[ELdist=1.7](n3,n1){$a, b$} \drawedge[ELdist=2.0](n1,n2){$a$}
\put(31,-73){$\dots$}
\end{picture}
\end{center}
\caption{The digraph $W_n$ and its unique coloring $\mathrsfs{W}_n$}\label{fig:anan}
\end{figure}

\begin{theorem}
\label{theorem:anan}
The automaton $\mathrsfs{W}_n$ is synchronizing and the minimum length
of its reset words is $n^2-3n+3$.
\end{theorem}

\begin{proof}
It is routine to verify that the word $(ab^{n-2})^{n-2}a$, whose length
is $(n-1)(n-2)+1=n^2-3n+3$, is a reset word for $\mathrsfs{W}_n$.

Now let $w$ be a \sw\ for $\mathrsfs{W}_n$ and assume that the length
of $w$ (denoted $|w|$) is minimal. Let $j\in Q=\{1,2,\dots,n\}$ be the
state to which the action of $w$ brings $\mathrsfs{W}_n$. Then from
every state in $Q$ there is a path to $j$ labelled $w$. It is clear
that for each $j\ne 2$ all paths ending at $j$ share the last edge.
Therefore, if $j\ne 2$, removing the last letter from the word $w$
produces a word that still would be a \sw\ for $\mathrsfs{W}_n$.
We conclude that $j=2$ because $|w|$ is minimal.

If $u\in\Sigma^*$, the word $uw$ also is a reset word and it also
brings the automaton to the state~2. Hence, for every $\ell\ge|w|$,
there is a path of length $\ell$ in $W_n$ from any given vertex $i$
to~2. In particular, setting $i=2$, we conclude that for every
$\ell\ge|w|$ there is a cycle of length $\ell$ in $W_n$. The digraph
$W_n$ has only two simple cycles: one of length $n$ and one of length $n-1$.
Each cycle of $W_n$ must consist of these two cycles traversed several times
whence each number $\ell\ge|w|$ must be expressible as a non-negative integer
combination of $n$ and $n-1$. Here we invoke Lemma~\ref{sylvester} which
implies that $|w|>n(n-1)-n-(n-1)=n^2-3n+1$. Suppose that $|w|=n^2-3n+2$.
Then there should be a path of this length from the vertex~1 to the vertex~2.
The only outgoing edge of~1 is $(1,2)$, and thus, in the path it must be
followed by a cycle of length $n^2-3n+1$. No cycle of such length may
exist by  Lemma~\ref{sylvester}. Hence $|w|\ge n^2-3n+3$.
\end{proof}

The series $\mathrsfs{W}_n$ was discovered by the first author in 2008
(unpublished). He gave a (rather involved) proof of Theorem~\ref{theorem:anan}
using a game-theoretic technique developed in~\cite{AVZ}.

As mentioned in Section~\ref{matrices}, Trahtman's recent result~\cite{Tr09}
implies that every primitive digraph admits a synchronizing coloring. This
gives rise to the following natural question: given a primitive digraph on
$n$ vertices, what is the minimum length of \sws\ for its synchronizing
colorings? Observe that in general underlying digraphs of slowly \sa\ may
admit colorings with rather short \sws. Fig.~\ref{fig:cerny} illustrates
this phenomenon: the first coloring of the 4-vertex digraph in Fig.~\ref{fig:cerny}
is the \v{C}ern\'{y} automaton $\mathrsfs{C}_4$ with shortest reset word of
length~9 while the second coloring can be reset of the word $a^3$ of length~3.
Wielandt's digraphs $W_n$, however, can be colored in a essentially unique
way, whence Theorem~\ref{theorem:anan} gives the lower bound $n^2-3n+3$
for the value in question. We strongly believe that this lower bound in fact
tight, in other words, we suggest a conjecture that is in a sense parallel
to the \v{C}ern\'{y} one.
\begin{conjecture}
\label{hybrid}
Every primitive digraph on $n$ vertices admits a synchronizing coloring
that can be reset by a word of length $n^2-3n+3$.
\end{conjecture}
We observe that while there is a clear analogy between Conjecture~\ref{hybrid}
and the \v{C}ern\'{y} conjecture, none of them immediately implies the other.

\begin{theorem}
\label{theorem:cherny}
The automaton $\mathrsfs{C}_n$ is synchronizing and the minimum length
of its reset words is $n^2-3n+3$.
\end{theorem}

\begin{figure}[ht]
\begin{center}
\unitlength .45mm
\begin{picture}(112,76)(0,-86)
\gasset{Nadjustdist=1.5} \node(n0)(56.0,-16.0){0}
\node(n1)(24.0,-40.0){4} \node(n2)(88.0,-40.0){1}
\node(n3)(36.0,-72.0){3} \node(n4)(76.0,-72.0){2}
\drawloop[ELdist=1.5,loopangle=33.34](n2){$a$}
\drawloop[ELdist=2.4,loopangle=320.0](n4){$a$}
\drawedge[ELdist=2.0](n1,n0){$b$} \drawedge[ELdist=1.5](n2,n4){$b$}
\drawedge[ELdist=1.7](n0,n2){$a, b$} \drawedge[ELdist=2.0](n4,n3){$b$}
\drawedge[ELdist=1.7](n3,n1){$b$}
%\drawedge[ELdist=2.0](n1,n2){$a$}
%\drawedge[ELdist=2.0](n0,n4){$a$}
\drawloop[ELdist=1.5,loopangle=144.55](n1){$a$}
\drawloop[ELdist=1.5,loopangle=226.55](n3){$a$}
\end{picture}
\end{center}
\caption{The automaton $\mathrsfs{C}_5$}\label{fig:cerny-n}
\end{figure}

\begin{proof}
It is routine to verify that the word $(ab^{n - 1})^{n - 2}a$, whose length
is $n(n-2)+1=n^2-2n+1$, is a reset word for $\mathrsfs{C}_n$.

Now let $w$ be a \sw\ for $\mathrsfs{C}_n$ and assume that the length
of $w$ (denoted $|w|$) is minimal.
Let us note that action of $b$ is cyclic permutation of states. 
Therefore, if $w$ ends on $b$, we could simply remove it, and modified word
still would be a \sw\ for $\mathrsfs{C}_n$.
We conclude that $w = w'a$ because $|w|$ is minimal.

Actions of $a$ and $a^2$ in $\mathrsfs{C}_n$ are equal. Therefore, every
subword $a^l$ in $w$ could be reduced to just $a$ and the word $w$ would still be
synchronizing. In other words, every occurence of $a$ in $w$ is
followed by $b$ except the last one.

Let's introduce new letter(and transition) $d = ab$ then $w'$ could be
transformed into $v \in \{b,d\}^*$.

\end{proof}




\begin{lemma}
$(ab^{n - 2})^{n - 2}ba$
\end{lemma}

\begin{theorem}\label{theo}
$n^2-3n+4$
\end{theorem}


\begin{figure}[ht]
\begin{center}
\unitlength .7mm
\begin{picture}(112,76)(0,-86)
\gasset{Nadjustdist=1.5} \node(n0)(56.0,-16.0){1}
\node(n1)(24.0,-40.0){0} \node(n2)(88.0,-40.0){2}
\node(n3)(36.0,-72.0){4} \node(n4)(76.0,-72.0){3}
\drawedge[ELdist=2.0](n1,n0){$b$} \drawedge[ELdist=1.5](n2,n4){$a, b$}
\drawedge[ELdist=1.7](n0,n2){$b$} \drawedge[ELdist=2.0](n4,n3){$a, b$}
\drawedge[ELdist=1.7](n3,n1){$a, b$} \drawedge[ELpos=40, ELdist=2.0](n1,n2){$a$}
\drawedge[ELpos=60,ELdist=2.0](n0,n4){$a$}
\end{picture}
\end{center}
\caption{The automaton $\mathrsfs{B}_5$}\label{B5}
\end{figure}

The states of the automaton $\mathrsfs{B}_n$
are the residues modulo $n$ and its input letters $a$ and $b$ act
as follows:
$$
 \delta(m,a)=
 \begin{cases}
  m + 2\!\!\pmod{n} & \text{for $m = 0,1$}, \\
  m+1\!\!\pmod{n} & \text{for $1< m<n$};
  \end{cases}
\qquad \delta(m,b)=m+1\!\!\pmod{n}.
$$
The smallest automaton in the series is shown in Fig.~\ref{B5}.



\begin{theorem}
$n^2-3n+2$
\end{theorem}

\begin{lemma}
$(ba^{n - 1})^{n - 3}ba$
\end{lemma}

\begin{figure}[ht]
\begin{center}
\unitlength .7mm
\begin{picture}(112,76)(0,-86)
\gasset{Nadjustdist=1.5} \node(n0)(56.0,-16.0){1}
\node(n1)(24.0,-40.0){0} \node(n2)(88.0,-40.0){2}
\node(n3)(36.0,-72.0){4} \node(n4)(76.0,-72.0){3}
\drawedge[ELdist=2.0](n1,n0){$b$} \drawedge[ELdist=1.5](n2,n4){$a, b$}
\drawedge[ELdist=1.7](n0,n2){$a$} \drawedge[ELdist=2.0](n4,n3){$a, b$}
\drawedge[ELdist=1.7](n3,n1){$a, b$} \drawedge[ELpos=40, ELdist=2.0](n1,n2){$a$}
\drawedge[ELpos=60,ELdist=2.0](n0,n4){$b$}
\end{picture}
\end{center}
\caption{The automaton $\mathrsfs{B}_5$}\label{B5}
\end{figure}

The states of the automaton $\mathrsfs{B}_n$
are the residues modulo $n$ and its input letters $a$ and $b$ act
as follows:
$$
 \delta(m,a)=
 \begin{cases}
  m+2\!\!\pmod{n} & \text{for $m = 1$}, \\
  m+1\!\!\pmod{n} & \text{for $m \neq 0$};
  \end{cases}
\qquad \delta(m,b)=
  \begin{cases}
  m + 2\!\!\pmod{n} & \text{for $m = 1$}, \\
  m+1\!\!\pmod{n} & \text{for $m \neq 1$};
  \end{cases}.
$$
The smallest automaton in the series is shown in Fig.~\ref{B5}.


\begin{theorem}\label{theo}
$n^2-3n+3$
\end{theorem}

\begin{lemma}
$(ab^{n - 2})^{n - 2}a$
\end{lemma}

\begin{figure}[ht]
\begin{center}
\unitlength .7mm
\begin{picture}(112,76)(0,-86)
\gasset{Nadjustdist=1.5} \node(n0)(56.0,-16.0){0}
\node(n1)(24.0,-40.0){4} \node(n2)(88.0,-40.0){1}
\node(n3)(36.0,-72.0){3} \node(n4)(76.0,-72.0){2}
\drawloop[ELdist=1.5,loopangle=33.34](n2){$a$}
\drawloop[ELdist=2.4,loopangle=320.0](n4){$a$}
\drawedge[ELdist=2.0](n1,n0){$b$} \drawedge[ELdist=1.5](n2,n4){$b$}
\drawedge[ELdist=1.7](n0,n2){$b$} \drawedge[ELdist=2.0](n4,n3){$b$}
\drawedge[ELdist=1.7](n3,n1){$b$}
%\drawedge[ELdist=2.0](n1,n2){$a$}
\drawedge[ELdist=2.0](n0,n4){$a$}
\drawloop[ELdist=1.5,loopangle=144.55](n1){$a$}
\drawloop[ELdist=1.5,loopangle=226.55](n3){$a$}
\end{picture}
\end{center}
\caption{The automaton $\mathrsfs{B}_5$}\label{B5}
\end{figure}

The states of the automaton $\mathrsfs{B}_n$
are the residues modulo $n$ and its input letters $a$ and $b$ act
as follows:
$$
 \delta(m,a)=
 \begin{cases}
  m + 2 \!\!\pmod{n} & \text{for $m = 0$}, \\
  m \!\!\pmod{n} & \text{for $m \neq 0$};
  \end{cases}
\qquad \delta(m,b)=m+1\!\!\pmod{n}.
$$
The smallest automaton in the series is shown in Fig.~\ref{B5}.



\begin{theorem}\label{theo}
$n^2-3n+2$
\end{theorem}

\begin{lemma}
$aab^{n - 2})^{n - 3}aa$
\end{lemma}

\begin{figure}[ht]
\begin{center}
\unitlength .7mm
\begin{picture}(112,76)(0,-86)
\gasset{Nadjustdist=1.5} \node(n0)(56.0,-16.0){1}
\node(n1)(24.0,-40.0){0} \node(n2)(88.0,-40.0){2}
\node(n3)(36.0,-72.0){4} \node(n4)(76.0,-72.0){3}
\drawloop[ELdist=1.5,loopangle=33.34](n2){$a$}
\drawloop[ELdist=2.4,loopangle=320.0](n4){$a$}
\drawedge[ELdist=2.0](n1,n0){$a$} \drawedge[ELdist=1.5](n2,n4){$b$}
\drawedge[ELdist=1.7](n0,n2){$a, b$} \drawedge[ELdist=2.0](n4,n3){$b$}
\drawedge[ELdist=1.7](n3,n1){$b$}
\drawedge[ELdist=2.0](n1,n2){$b$}
%\drawedge[ELdist=2.0](n0,n4){$a$}
%\drawloop[ELdist=1.5,loopangle=144.55](n1){$a$}
\drawloop[ELdist=1.5,loopangle=226.55](n3){$a$}
\end{picture}
\end{center}
\caption{The automaton $\mathrsfs{B}_5$}\label{B5}
\end{figure}

The states of the automaton $\mathrsfs{B}_n$
are the residues modulo $n$ and its input letters $a$ and $b$ act
as follows:
$$
 \delta(m,a)=
 \begin{cases}
  m + 1 \!\!\pmod{n} & \text{for $m = 0, 1$}, \\
  m \!\!\pmod{n} & \text{for $1 < m < n$};
  \end{cases}
\qquad
 \delta(m,b)=\begin{cases}
  m + 2 \!\!\pmod{n} & \text{for $m = 0$}, \\
  m + 1 \!\!\pmod{n} & \text{for $m \neq 0$};
  \end{cases}.
$$
The smallest automaton in the series is shown in Fig.~\ref{B5}.


\begin{theorem}\label{theo}
$n^2-4n+6$
\end{theorem}

\begin{lemma}
$b(ab^{n - 2})^{n - 3}ab$
\end{lemma}


\begin{figure}[th]
\unitlength=.7mm
\begin{center}
\begin{picture}(100,70)(0,-80)
\gasset{Nadjustdist=1.5} \node(n0)(56.0,-16.0){0}
\node(n1)(24.0,-40.0){4} \node(n2)(88.0,-40.0){1}
\node(n3)(36.0,-72.0){3} \node(n4)(76.0,-72.0){2}
\drawloop[ELdist=2.4,loopangle=320.0](n4){$a$}
\drawedge[ELdist=2.0](n1,n0){$b$} \drawedge[ELdist=1.5](n2,n4){$b$}
\drawedge[ELdist=1.7](n0,n2){$b$} \drawedge[ELdist=2.0](n4,n3){$b$}
\drawedge[ELdist=1.7](n3,n1){$b$}
\node[NLangle=0.0](n16)(56.0,-48.0){5}
%\drawedge[ELdist=1.6](n1,n16){$a$}
\drawedge[ELdist=1.7](n16,n4){$b$}

\drawedge[curvedepth=2](n0,n16){$a$}
\drawedge[curvedepth=2](n16,n0){$a$}

%\drawloop[ELdist=1.5,loopangle=90](n0){$a$}
%\drawloop[ELdist=1.5](n16){$b$}
%\drawedge[ELdist=1.5,curvedepth=33](n3,n0){$a$}
\drawloop[ELdist=1.5,loopangle=144.55](n1){$a$}
\drawloop[ELdist=1.5,loopangle=226.55](n3){$a$}
\drawloop[ELdist=1.5,loopangle=33.34](n2){$a$}
\end{picture}
\end{center}
\caption{The automaton $\mathrsfs{B}^{\star}_6$}\label{B-star-6}
\end{figure}



\begin{theorem}\label{theo}
$n^2-4n+6$
\end{theorem}

\begin{lemma}
$b(ab^{n - 2})^{n - 3}ba$
\end{lemma}


\begin{figure}[th]
\unitlength=.7mm
\begin{center}
\begin{picture}(100,70)(0,-80)
\gasset{Nadjustdist=1.5} \node(n0)(56.0,-16.0){0}
\node(n1)(24.0,-40.0){4} \node(n2)(88.0,-40.0){1}
\node(n3)(36.0,-72.0){3} \node(n4)(76.0,-72.0){2}
\drawloop[ELdist=2.4,loopangle=320.0](n4){$a$}
\drawedge[ELdist=2.0](n1,n0){$b$} \drawedge[ELdist=1.5](n2,n4){$a, b$}
\drawedge[ELdist=1.7](n0,n2){$b$} \drawedge[ELdist=2.0](n4,n3){$b$}
\drawedge[ELdist=1.7](n3,n1){$b$}
\node[NLangle=0.0](n16)(56.0,-48.0){5}
\drawedge[ELdist=1.7](n16,n4){$b$}

\drawedge[curvedepth=2](n0,n16){$a$}
\drawedge[curvedepth=2](n16,n0){$a$}

\drawloop[ELdist=1.5,loopangle=144.55](n1){$a$}
\drawloop[ELdist=1.5,loopangle=226.55](n3){$a$}
\end{picture}
\end{center}
\caption{The automaton $\mathrsfs{B}^{\star}_6$}\label{B-star-6}
\end{figure}

\section{Experiments}
\label{experiments}
During the experiments every initially-connected deterministic automata was checked on synchronizability and it's shortest reset word was
calculated. This class of automata was deceided to use because it contains every strongly connected automata and has very simple and convenient
way of enumeration which is presented and thoroughly described in [Almeida]. There are about 700 billions of initially-connected deterministic automata
with 9 states and it is hard task for a single computer to do. That's why this check was distributed and done within a day on 60 cores of
amd opteron 2.6 Ghz processors. Program was written in C with mpi. Every obtained slowly synchronizing automaton was rechecked
with package TESTAS[trahtman's site]
developed by A.Trahtman.
It is also worth to mention that experiments show that deterministic automata indeed tends to be synchronizible with short reset words.
Almost $90\%$ of initially-connected deterministic automata with 9 states could be synchronized with word of length not greater than 9.
And as shows random sampling experiment this part becomes bigger with growth of number of states.

\begin{thebibliography}{99}
\bibitem{AGW}
Adler, R.L., Goodwyn, L.W., Weiss, B.: Equivalence of topological Markov shifts.
Israel J. Math. 27, 49--63 (1977)

\bibitem{AS09}
Almeida, J.; Steinberg, B.: Matrix mortality and the \v{C}ern\'{y}--Pin
conjecture. In:  Diekert, V.; Nowotka, D. (eds.), Developments in
Language Theory, Lect.\ Notes Comput.\ Sci., vol.\,5583, pp. 67--80.
Springer, Heidelberg (2009)

\bibitem{AVZ}
Ananichev, D.S., Volkov, M.V., Zaks, Yu.I.: Synchronizing automata
with a letter of deficiency 2. Theor.\ Comput.\ Sci. 376, 30--41 (2007)

\bibitem{Ce64}
\v{C}ern\'{y}, J.: Pozn\'{a}mka k homog\'{e}nnym eksperimentom s
kone\v{c}n\'{y}mi automatami. Matematicko-fyzikalny \v{C}asopis
Slovensk.\ Akad.\ Vied 14(3) 208--216 (1964) (in Slovak)

\bibitem{Du98}
Dubuc, L.: Sur les automates circulaires et la conjecture de
\v{C}ern\'y. RAIRO Inform.\ Th\'eor.\ Appl. 32, 21--34 (1998) (in
French)

\bibitem{DM62}
Dulmage, A.L., Mendelsohn, N.S.: The exponent of a primitive matrix.
Can.\ Math.\ Bull. 5, 241--244 (1962)

\bibitem{DM64}
Dulmage, A.L., Mendelsohn, N.S.: Gaps in the exponent set of primitive
matrices. Ill.\ J. Math. 8, 642--656 (1964)

\bibitem{Ep90}
Eppstein, D.: Reset sequences for monotonic automata. SIAM J.
Comput. 19, 500--510 (1990)

\bibitem{Hi88}
Higgins, P.M.: The range order of a product of $i$ transformations
from a finite full transformation semigroup, Semigroup Forum 37, 31--36
(1988)

\bibitem{Ka03}
Kari, J.: Synchronizing finite automata on Eulerian digraphs.
Theoret.\ Comput.\ Sci. 295, 223--232 (2003)

\bibitem{Pi83}
Pin, J.-E.: On two combinatorial problems arising from automata
theory. Ann.\ Discrete Math. 17, 535--548 (1983)

\bibitem{RaAl05}
Ram\'{\i}rez Alfons\'{\i}n, J.L.: The diophantine Frobenius problem.
Oxford University Press (2005)

\bibitem{Sa05}
Sandberg, S.: Homing and synchronizing sequences. In: Broy, M.
et~al. (eds.), Model-Based Testing of Reactive Systems. Lect.\
Notes Comput.\ Sci., vol.\,3472, pp.\,5--33. Springer, Heidelberg
(2005)

\bibitem{SZ}
Skvortsov E., Zaks Yu.: Synchronizing random automata. Submitted;
proceedings version in: Rigo, M. (ed.), AutoMathA 2009, Universit\'e
de Li\`ege (2009)

\bibitem{Tr06}
Trahtman, A.N.: An efficient algorithm finds noticeable trends
and examples concerning the \v{C}ern\'y conjecture. In:
Kr\'alovi\v{c}, R.; Urzyczyn, P. (eds.), Mathematical Foundations
of Computer Science. Lect.\ Notes Comput.\ Sci., vol.\,4162, pp.\,789--800
Springer, Heidelberg (2006)

\bibitem{Tr06a}
Trahtman, A.N.: Notable trends concerning the synchronization
of graphs and automata. Electr.\ Notes Discrete Math. 25, 173--175 (2006)

\bibitem{Tr07}
Trahtman, A.N.: The \v{C}ern\'y conjecture for aperiodic automata.
Discrete Math.\ Theor.\ Comput.\ Sci. 9(2), 3--10 (2007)

\bibitem{Tr09}
Trahtman, A.N.: The Road Coloring Problem. Israel J. Math.
\textbf{172}, 51--60 (2009)

\bibitem{Vo08}
Volkov, M.V.: Synchronizing automata and the \v{C}ern\'{y}
conjecture. In: Mart\'\i{}n-Vide, C.; Otto, F.; Fernau, H. (eds.),
Languages and Automata: Theory and Applications. Lect.\ Notes
Comput.\ Sci., vol.\,5196, pp.\,11--27.  Springer, Heidelberg (2008)

\bibitem{Vo09}
Volkov, M.V.: Synchronizing automata preserving a chain of partial
orders. Theoret.\ Comput.\ Sci. 410, 2992--2998 (2009)

\bibitem{Wi50}
Wielandt, H.: Unzerlegbare, nicht negative Matrizen. Math.\ Z.
52, 642--648 (1950) (in German)
\end{thebibliography}


\section*{Appendix}

\end{document}
